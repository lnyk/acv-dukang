\cvsection{编译使用详细说明}
\cvsubsection{Makefile及编译命令详解}
\dk~使用Makefile\footnote{前半部分的指令都有注释进行说明,make的用法也不在本文的讨论范围之内,感兴趣的朋友可以自行学习。}将几个最常用的{\LaTeX}编译相关操作定义成了对应的控制台命令,用以对编译、输出、清理等过程进行简化处理,以及联动resource文件夹下的Makefile文件。

\dkcodefile*{../Makefile}{makefile}[Makefile]

\begin{dkcomment}
  以下所有make命令,都必须在项目根目录(就是你能看到\dkbutton{README.md}的地方)下执行。
\end{dkcomment}

\begin{cvhonors}*
  \cvhonor
  {make}
  {自动编译输出main.tex,等同于make main}
  {编译输出}
  \cvhonor
  {make main}
  {同上}
  {编译输出}
  \cvhonor
  {make doc}
  {自动编译输出\dk~文档,也就是本文档的输出命令}
  {编译输出}
  \cvhonor
  {make resource}
  {使用resource文件夹下的Makefile编译输出所有支持的资源文件,等同于进入resource文件夹下进行make或make all}
  {关联编译}
  \cvhonor
  {make all}
  {一次性编译main.tex和\dk~文档,等同于\dkbutton{make main \&\& make doc}}
  {编译输出}
  \cvhonor
  {make clean}
  {自动清理所有临时文件和文件夹,包括主目录和所有子目录}
  {自动清理}
  \cvhonor
  {make cleanall}
  {在自动清理所有临时文件和文件夹的基础上,还会删除掉所有主目录下的PDF文件,并联动resource文件夹下所有生成的资源PDF}
  {自动清理}
\end{cvhonors}

\begin{dkcomment}
  \dkbutton{make cleanall}只会删除符合资源文件命名规则的PDF文件,其他文件不受影响。
\end{dkcomment}

\cvsubsection{选项设定部分}
\dk~提供了用于立刻开始创作的初始文件,位置是src/main.tex,该文件导言区\footnote{相信你知道什么是导言区,如果不是太清楚,还是抓紧自学一下吧,或者简单的认为,在大多数情况下,\\\textbackslash begin\{document\}前面的部分,就是导言区。}部分的设定与本文档源码src/dukang-doc.tex完全相同,只需要根据需要修改一些文档名称、作者姓名、首页需要哪些字段、是否需要首行缩进等信息和选项,就可以正式开始为你的大作添加正文了。

打开main.tex之后,首先会看到导言区一堆设定和备注信息,大多数情况下,阅读这些备注信息就足够掌握如何修改了,这里就挑其中几个比较重要的选项着重说明一下。

待补全

假如不小心把main.tex搞乱了也没关系,可以随时打开dukang-doc.tex查看正确的配置,或者干脆把导言区的所有内容复制回来,重新简单设定一下,就又可以开始创作了。这也是为什么我推荐用\dkbutton{\textbackslash input\{...\}}把文档正文章节和main.tex主文件分开的原因。

\begin{dkcomment}*[温馨提示]
  \hspace{2em}无论何时,dukang-doc.tex都是你值得参考的示例文档,文档本身和其内容章节文件(特别是源代码)尽量涵盖到了{\dk}的全部功能,包括设定和功能模块等,随时可以回来查看。\faKissWinkHeart
\end{dkcomment}

\cvsubsection{编译使用部分}
\dk~的编译控制文件(Makefile)提供了适合下面几种场景的编译流程,相信总有一个适合你。首先,对于一般使用来说,需要做的步骤很简单:

\begin{center}
  \dkcodebox{修改main.tex}~\faArrowCircleRight~\dkcodebox{添加内容}~\faArrowCircleRight~\dkcodebox*{make}~\faCheckCircle
\end{center}

这样在src文件夹下就得到了main.pdf,同时在resource文件夹下,如果有符合命名规则的资源文件,也会被联动编译,并生成对应的PDF文件。{\dk}当前版本资源文件的命名规则是\dkbutton{r-*.tex},符合这个规则的.tex文件都会被自动编译和控制。

如果想要保留编译之后的PDF,同时把项目目录清理干净的话:

\begin{center}
  \dkcodebox{修改main.tex}~\faArrowCircleRight~\dkcodebox{添加内容}~\faArrowCircleRight~\dkcodebox*{make}~\faArrowCircleRight~\dkcodebox*{make clean}~\faCheckCircle
\end{center}

最后,如果希望只保留源代码\footnote{比如用于提交源代码,归档,或者你就是个纯粹的代码强迫症患者{\color{awesome}\faHeart}},把其他临时文件连同编译出来的东西一同干掉的话:

\begin{center}
  \dkcodebox{修改main.tex}~\faArrowCircleRight~\dkcodebox{添加内容}~\faArrowCircleRight~\dkcodebox*{make}~\faArrowCircleRight~\dkcodebox*{make cleanall}~\faCheckCircle
\end{center}

\clearpage
