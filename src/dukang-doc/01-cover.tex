% 信件环境的开头部分
% 可以注释掉,不显示该部分
\makelettertitle

%-------------------------------------------------------------------------------
%	信件环境正文
%-------------------------------------------------------------------------------
\begin{cvletter}

\lettersection{一副图看懂总体结构}
\dkresource[htb]{resource/r-arch}{0.9}{}

\begin{dkcomment}{测试dkcomment}{\faTree}
这里是dkcomment的正文,这里是dkcomment的正文,这里是dkcomment的正文,这里是dkcomment的正文,这里是dkcomment的正文,这里是dkcomment的正文,这里是dkcomment的正文,这里是dkcomment的正文,这里是dkcomment的正文,这里是dkcomment的正文,这里是dkcomment的正文,这里是dkcomment的正文,这里是dkcomment的正文,这里是dkcomment的正文,这里是dkcomment的正文,这里是dkcomment的正文,这里是dkcomment的正文,这里是dkcomment的正文,这里是dkcomment的正文,这里是dkcomment的正文。
\end{dkcomment}

\begin{dkcodeh}{python}{tango}{Test脚本}
import os
print("仍然没有成功解决CTeX环境下中英文之间的多余空格!")
print("Sigh... Need to solve the bug...")
pass
\end{dkcodeh}


\lettersection{Why Google?}
Suspendisse commodo, massa eu congue tincidunt, elit mauris pellentesque orci, cursus tempor odio nisl euismod augue. Aliquam adipiscing nibh ut odio sodales et pulvinar tortor laoreet. Mauris a accumsan ligula. Class aptent taciti sociosqu ad litora torquent per conubia nostra, per inceptos himenaeos. Suspendisse vulputate sem vehicula ipsum varius nec tempus dui dapibus. Phasellus et est urna, ut auctor erat. Sed tincidunt odio id odio aliquam mattis. Donec sapien nulla, feugiat eget adipiscing sit amet, lacinia ut dolor. Phasellus tincidunt, leo a fringilla consectetur, felis diam aliquam urna, vitae aliquet lectus orci nec velit. Vivamus dapibus varius blandit.

\lettersection{Why Me?}
Duis sit amet magna ante, at sodales diam. Aenean consectetur porta risus et sagittis. Ut interdum, enim varius pellentesque tincidunt, magna libero sodales tortor, ut fermentum nunc metus a ante. Vivamus odio leo, tincidunt eu luctus ut, sollicitudin sit amet metus. Nunc sed orci lectus. Ut sodales magna sed velit volutpat sit amet pulvinar diam venenatis.

\end{cvletter}


%-------------------------------------------------------------------------------
% 信件环境结尾部分
% 可以注释掉,不显示该部分
% \makeletterclosing

% 信件环境结束后开新页
\clearpage
