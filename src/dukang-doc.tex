%!TEX TS-program = xelatex
%!TEX encoding = UTF-8 Unicode

%-------------------------------------------------------------------------------
% 基本配置
%-------------------------------------------------------------------------------
% 默认为A4纸张,12pt字号
\documentclass[12pt, a4paper, final]{awesome-cv}

% 使用geometry定义纸张边距
\geometry{left=1.4cm, top=.8cm, right=1.4cm, bottom=1.8cm, footskip=.5cm}

% 高亮颜色配置
% Awesome Colors:
%   awesome-emerald, awesome-skyblue, awesome-red, awesome-pink, awesome-orange
%   awesome-nephritis, awesome-concrete, awesome-darknight
\colorlet{awesome}{awesome-red}
% 下面部分可启用自定义颜色
% \definecolor{awesome}{HTML}{CA63A8}

% Colors for text
% Uncomment if you would like to specify your own color
% \definecolor{darktext}{HTML}{414141}
% \definecolor{text}{HTML}{333333}
% \definecolor{graytext}{HTML}{5D5D5D}
% \definecolor{lighttext}{HTML}{999999}
% \definecolor{sectiondivider}{HTML}{5D5D5D}

% 全局开关变量定义
% 是否使用高亮颜色配置来突出section标题
\setbool{acvSectionColorHighlight}{true}

% Header中社交媒体行的分隔符
% 目前指定为管道符 |
\renewcommand{\acvHeaderSocialSep}{\quad\textbar\quad}


%-------------------------------------------------------------------------------
%	个人信息
%	不需要的部分可以注释掉
%-------------------------------------------------------------------------------
% 首页的图片Logo,默认为./src/resource/profile<.png>
% 可用选项为circle|rectangle,edge/noedge,left/right
\photo[circle,noedge,left]{./src/resource/profile}
\name{\LaTeX{} Dukang}{模板手册}
\position{%
  \hyperref{https://github.com/posquit0/Awesome-CV}{}{}{Awesome-CV}{\enskip\faAdversal\enskip}\hyperref{http://williamyao.com}{}{}{Dukang \LaTeX{} Template}{\enskip\faWineBottle\enskip}
}
\address{基于\hyperref{https://github.com/posquit0}{}{}{Byungjin Park}的开源项目\hyperref{https://github.com/posquit0/Awesome-CV}{}{}{Awesome-CV}进行中文化适应及无侵入修改的友好\LaTeX 启动模板}

%-------------------------------------------------------------------------------
% 以下部分至少有一条要保留
% \mobile{(+82) 10-9030-1843}
% \email{posquit0.bj@gmail.com}
% \dateofbirth{January 1st, 1970}
\homepage{WilliamYao.com}
\github{lnyk}
% \linkedin{posquit0}
% \gitlab{gitlab-id}
% \stackoverflow{SO-id}{SO-name}
% \twitter{@twit}
% \skype{skype-id}
% \reddit{reddit-id}
% \medium{madium-id}
% \kaggle{kaggle-id}
% \googlescholar{googlescholar-id}{name-to-display}
%% \firstname and \lastname will be used
% \googlescholar{googlescholar-id}{}
\extrainfo{{\faWineBottle}衷心感谢\hyperref{https://github.com/posquit0}{}{}{Byungjin Park}同学的开源奉献}

\quote{“执着而不计成本,不为索取只为陶醉”\hspace{1em}——\hspace{1em}Carl Zeiss}


%-------------------------------------------------------------------------------
%	信件环境基本信息
%	所有内容必须不能缺少
%-------------------------------------------------------------------------------
% 收件方
\recipient
  {亲爱的\LaTeX 用户}
  {无论你在哪里工作\\以及你在什么地方生活\\或者\\为了理想,你有过多么努力的拼搏!\\请听我说……}
% 信件日期
\letterdate{\today}
% 信件标题
\lettertitle{未来的你,一定会感谢今天拼搏的自己!}
% 信件称谓
\letteropening{来自未来的一本书}
% 信件结尾词
\letterclosing{感谢最努力的自己吧!}
% 信件结尾词附言
\letterenclosure[亲爱的]{所有我爱的,以及爱我的人}


%-------------------------------------------------------------------------------
% dukang导言区设定
%-------------------------------------------------------------------------------
% 引入dukang宏包
\usepackage{dukang}
%-------------------------------------------------------------------------------
% 打开该开关会启用某些正文内容的首行缩进
\setbool{dukangParIndent}{true}
% 是否为PDF书签添加标题前的标号
\setbool{dukangBookmarkLeadingNumber}{true}
%-------------------------------------------------------------------------------
% 定义编译之后的PDF相关属性
\hypersetup{%
  pdftitle={\LaTeX{} Dukang模板手册},
  pdfauthor={William Yao},
  pdfcreator={William Yao},
  pdfsubject={引用Awesome-CV模板,继承并加强},
  pdfkeywords={tex,latex,pdf,awesome,cv,resume,book,article,william,yao,dukang}
}
%-------------------------------------------------------------------------------


%-------------------------------------------------------------------------------
\begin{document}
%-------------------------------------------------------------------------------
% dukang文档区设定
%-------------------------------------------------------------------------------
% 由于添加了生成书签的功能,需要在文档区调用下面的命令生成首页的书签
% 默认使用导言区的\name自动生成
% 如果缺少该项,PDF书签将没有顶层(0级)标题
\dukangSetTitleRef
%-------------------------------------------------------------------------------
% 首页的抬头
% 可选
% 可用对齐选项为C: 居中,L: 左对齐,R: 右对齐
\makecvheader[R]
%-------------------------------------------------------------------------------
% 每页的页脚定义,分为左中右三个部分,分别对应每个{}
% 各部分都可留空,但不能删掉{}
\makecvfooter
  {\today}% 左边部分
  {\hyperref{http://williamyao.com}{}{}{William Yao}~~~\faTree~~~\hyperref{http://williamyao.com}{}{}{\LaTeX Dukang模板}}% 中间部分
  {\thepage}% 右边部分


%-------------------------------------------------------------------------------
% 正文内容部分
% 独立引用每个文件,或者直接书写正文
%-------------------------------------------------------------------------------
% doc的首页
% 信件环境的开头部分
% 可以注释掉,不显示该部分
\makelettertitle

%-------------------------------------------------------------------------------
%	信件环境正文
%-------------------------------------------------------------------------------
\begin{cvletter}

\lettersection{它们就在这些地方}
\dkresource[htb]{resource/r-arch}{0.9}{}

\end{cvletter}


%-------------------------------------------------------------------------------
% 信件环境结尾部分
% 可以注释掉,不显示该部分
% \makeletterclosing

% 信件环境结束后开新页
\clearpage


\input{cv/education.tex}
\input{cv/skills.tex}
\input{cv/experience.tex}
\input{cv/extracurricular.tex}
\input{cv/honors.tex}
\input{cv/presentation.tex}
\input{cv/writing.tex}
\input{cv/committees.tex}


%-------------------------------------------------------------------------------
%	CV/RESUME CONTENT
%	Each section is imported separately, open each file in turn to modify content
%-------------------------------------------------------------------------------
\input{resume/summary.tex}
\input{resume/experience.tex}
\input{resume/honors.tex}
\input{resume/presentation.tex}
\input{resume/writing.tex}
\input{resume/committees.tex}
\input{resume/education.tex}
\input{resume/extracurricular.tex}


\end{document}
