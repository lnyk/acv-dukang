\cvsection{总体介绍}
Awesome-CV本身设计非常优秀,但由于语言习惯等差别,其对中文环境的支持和适配需要用户自行调整很多东西,同时其原作者的本意主要是聚焦在构建“简历”模板,鄙人以为,如此漂亮的设计,理应延伸至文章甚至书籍创作领域。在这个层面上,需要更加宏观的对项目中每个部件的细节进行调整,并引入适合文章或书籍创作的功能模块,而且相对于漂亮的设计和丰富的功能,{\LaTeX}过高的使用门槛却一直阻挡在追求严谨排版和高质量输出,却不是特别习惯看似古怪的语法、复杂难记的命令环境和众多宏包的朋友面前,“劝退”很多有志之士踏进{\LaTeX}的神奇世界。

尝试使用ctex宏包对Awesome-CV进行中文环境适配,对相应的部件和元素进行调整,对一些代码上的设计进行改进,并增加一些封装好的开箱即用的实用部件,是{\dk}项目的主要目的,将{\LaTeX}方面的一些知识表述清楚,将维护该项目学习到的一些技巧分享出去,也是我的初衷。

罗嗦这么多,还是赶快让我们进入主题吧。

\cvsubsection{主体结构}
{\dk}的结构大致上分为三个逻辑部分,第一部分是来自Awesome-CV项目的文档类,定义了所生成文件的最重要的内容,第二部分是本项目新增的宏包、扩展内容、资源文件夹以及编译控制文件,第三部分是提供给使用者作为快速开始模板的文件,当然还有一些项目维护所涉及的例如README、LICENSE、Git相关文件等。

下图列举了项目主体结构和主要的文件所在位置,其中来自本项目的所有文本类文件,都有详细的备注和说明信息,如果想要{\color{awesome}快速开始},不妨按照下图找到这些文件,阅读里面的源代码,寻找自己感兴趣的部分。

\dkresource{resource/r-arch}[0.9]

\begin{dkcomment}
  项目其他子文件夹和一些辅助性的文件,比如LOGO等没有在上图中体现。
\end{dkcomment}

\cvsubsection{快速开始}

如果你是{\LaTeX}资深玩家,或者之前有过使用经验,很有可能已经拥有了一个随时可用的本地环境。确认你已经安装好了合适的{\LaTeX}套件(比如TexLive 2022版本),对应操作系统的GNU Make,Python 3和Pygments库以及所需字体\footnote{如果你是资深玩家,请查阅src/ctex-fontset-custom.def文件并安装好对应字体,否则请首先阅读“编译使用详解”章节的“本地环境准备”部分。}之后,可以使用下面的命令快速体验{\dk}的输出过程和简单用法,如果不确定你的本地环境是否符合要求,请暂时越过本章节,找到“编译使用详解”章节的“本地环境准备”部分跟着说明逐步进行。

\begin{dkcode}*{bash}[打开控制台依次执行]
# 进入项目根目录
cd latex-dukang

# 编译生成你的大作
make

# 编译生成本文档
make doc
\end{dkcode}

整个编译的过程,会输出大量的编译信息和告警内容,如果看到Warning提示,先别着急,在{\LaTeX}的世界里,有众多的叫做“宏包”的贡献者在共同的支撑着你的创作,并不是所有的告警信息都意味着你“哪里做错了”,有些只是“善意的提醒”,比如{\dk}当前版本会有“未找到斜体字形”、“hbox overfull/underfull”等告警,虽然这类告警一定意味着“编译质量不怎么高”,但实际情形并不影响最终的输出结果,这些告警在后续的章节会有详细的说明。

回到上面的命令,如果执行顺利的话,第一个make执行完毕后,都会有类似下面的内容:

\vspace{1em}
\begin{center}
  \dkcodebox*{Latexmk: All targets (src/main.xdv src/main.pdf) are up-to-date}
\end{center}
\vspace{1em}

这说明main.tex文件已经成功被编译输出到了main.pdf,可以打开这个PDF文件看看实际的效果。第二个make是用来生成{\dk}文档(也就是本文)的,也跑一跑编译吧,确保你看到下面的提示信息。

\vspace{1em}
\begin{center}
  \dkcodebox*{Latexmk: All targets (src/dukang-doc.xdv src/dukang-doc.pdf) are up-to-date}
\end{center}
\vspace{1em}

如果顺利编译完成,你会在\dkbutton{./src/}文件夹中看到main.pdf和dukang-doc.pdf两个文件,这说明你已经可以正常使用{\dk}进行创作了。

如果,你碰到了光标停在一个问号后面,画面信息不再滚动的情形,那直接就是“错误”而不是“告警”了,这时候需要你向上查看具体哪里出了问题,或者输入大写的“X”并回车退出编译过程。这时候一般是没有main.pdf或dukang-doc.pdf输出的,请仔细检查输出日志中的错误提示,尝试着解决问题,并在\dkcodebox*{make cleanall}清理完所有临时文件之后再次尝试make。


\clearpage
