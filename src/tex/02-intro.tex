\cvsection{结构设计总体介绍}
由于\dk~尽量使用非侵入式的方式与Awesome-CV进行集成,自然也沿用了其简洁方便的结构设计,除了总体编译文件、项目说明文件、项目图标以及文档类本体以外,其他的内容相关文件都放在\dkbutton{src/}中。其中为了更方便的操作,部分文件使用了软链接\footnote{使用make以及软链接,对Windows平台和使用IDE的用户来说可能需要做一些适应性调整,这部分问题目前不在\dk~的代码设计范围之内,也许未来版本会考虑加入跨平台的设计要素。}。

\begin{dkcomment}{项目结构说明}{\faFolder}
\dirtree{%
.1 \dk.
.2 \faFileCode~awesome-cv.cls\DTcomment{Awesome-CV的类文件}.
.2 \faFileImage~icon.png\DTcomment{项目图标}.
.2 \faFileCode~Makefile\DTcomment{控制编译命令的make文件}.
.2 \faMarkdown~README.md\DTcomment{项目说明文件}.
.2 \faFolder~src\DTcomment{内容相关文件}.
.3 \faLink~awesome-cv.cls\DTcomment{Awesome-CV类文件的软链接}.
.3 \faFileCode~ctex-fontset-custom.def\DTcomment{ctex的自定义字体文件}.
.3 \faFileCode~dukang.sty\DTcomment{\dk~的主文件}.
.3 \faFileCode~dukang-doc.tex\DTcomment{本文档的主文件}.
.3 \faFileCode~main.tex\DTcomment{最简化的起始文件}.
.2 \faFolder~dukang-doc\DTcomment{存放本文档的章节内容文件}.
.2 \faFolder~resource\DTcomment{存放图形、表格、外部pdf或tex等资源文件}.
.3 \faLink~ctex-fontset-custom.def\DTcomment{ctex自定义字体文件的软链接}.
.3 \faFileCode~Makefile\DTcomment{对应resource文件夹的make文件}.
.3 \faFileImage~profile.png\DTcomment{文档首页头信息中引用的图片文件}.
.3 \faFileCode~r-*.png\DTcomment{所有以\dkbutton{r-*.tex}形式命名的文件都会支持自动编译和清理}.
}
\end{dkcomment}

\begin{cvskills}

%---------------------------------------------------------
  \cvskill
    {DevOps} % Category
    {AWS, Docker, Kubernetes, Rancher, Vagrant, Packer, Terraform, Jenkins, CircleCI} % Skills

%---------------------------------------------------------
  \cvskill
    {Back-end} % Category
    {Koa, Express, Django, REST API} % Skills

%---------------------------------------------------------
  \cvskill
    {Front-end} % Category
    {Hugo, Redux, React, HTML5, LESS, SASS} % Skills

%---------------------------------------------------------
  \cvskill
    {Programming} % Category
    {Node.js, Python, JAVA, OCaml, LaTeX} % Skills

%---------------------------------------------------------
  \cvskill
    {Languages} % Category
    {Korean, English, Japanese} % Skills

%---------------------------------------------------------
\end{cvskills}

\clearpage
