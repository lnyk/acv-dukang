\cvsection{编译使用详解}
\cvsubsection{Makefile及编译}
\dk~使用Makefile\footnote{前半部分的指令都有注释进行说明,make的用法也不在本文的讨论范围之内,感兴趣的朋友可以自行学习。}将几个最常用的{\LaTeX}编译相关操作定义成了对应的控制台命令,用以对编译、输出、清理等过程进行简化处理,以及联动resource文件夹下的Makefile文件。

\dkcodefile*{../Makefile}{makefile}[Makefile]

\begin{dkcomment}
  以下所有make命令,都必须在项目根目录(就是你能看到\dkbutton{README.md}的地方)下执行。
\end{dkcomment}

\begin{cvhonors}*
  \cvhonor
  {make}
  {自动编译输出main.tex,等同于make main}
  {编译输出}
  \cvhonor
  {make main}
  {同上}
  {编译输出}
  \cvhonor
  {make doc}
  {自动编译输出\dk~文档,也就是本文档的输出命令}
  {编译输出}
  \cvhonor
  {make resource}
  {使用resource文件夹下的Makefile编译输出所有支持的资源文件,等同于进入resource文件夹下进行make或make all}
  {关联编译}
  \cvhonor
  {make all}
  {一次性编译main.tex和\dk~文档,等同于\dkbutton{make main \&\& make doc}}
  {编译输出}
  \cvhonor
  {make clean}
  {自动清理所有临时文件和文件夹,包括主目录和所有子目录}
  {自动清理}
  \cvhonor
  {make cleanall}
  {在自动清理所有临时文件和文件夹的基础上,还会删除掉所有主目录下的PDF文件,并联动resource文件夹下所有生成的资源PDF}
  {自动清理}
\end{cvhonors}

\begin{dkcomment}
  \dkbutton{make cleanall}只会删除符合资源文件命名规则的PDF文件,其他文件不受影响。
\end{dkcomment}

\cvsubsection{选项设定}
\dk~提供了用于立刻开始创作的初始文件,位置是src/main.tex,该文件导言区\footnote{相信你知道什么是导言区,如果不是太清楚,还是抓紧自学一下吧,或者简单的认为,在大多数情况下,\\\textbackslash begin\{document\}前面的部分,就是导言区。}部分的设定与本文档源码src/dukang-doc.tex完全相同,只需要根据需要修改一些文档名称、作者姓名、首页需要哪些字段、是否需要首行缩进等信息和选项,就可以正式开始为你的大作添加正文了。

打开main.tex之后,首先会看到导言区一堆设定和备注信息,大多数情况下,阅读这些备注信息就足够掌握如何修改了,这里对所有选项进行深入说明。

\begin{cventries}
\cventry
[\Verb{\documentclass[12pt, a4paper, final]{awesome-cv-dukang}}]
[基本配置]
{
  \item 正式引入Awesome-CV文档类。
  \item 这里的字号设定(12pt)基本没什么卵用,因为几乎每个文档部件都定义了自己的字体风格。
}
\cventry
[\Verb{\geometry{left=1.4cm, top=.8cm, right=1.4cm, bottom=1.8cm, footskip=.5cm}}]
[基本配置]
{
  \item 使用geometry宏包定义纸张的页边距以及页脚距离
}
\cventry
[\Verb{\colorlet{awesome}{awesome-red}}]
[Awesome-CV]
[Awesome-CV的颜色设定]
[必选]
{
  \item 可以指定Awesome-CV预制好的几个配色集,包括awesome-emerald, awesome-skyblue, awesome-red, awesome-pink, awesome-orange, awesome-nephritis, awesome-concrete, awesome-darknight
  \item 也可以使用\Verb{\definecolor}指定自己喜欢的颜色,总共有awesome, darktext, text, graytext, lighttext, sectiondivider这几个颜色名称可供定义。
  \item {\color{awesome}{\dk}提供的所有增强部件都可以根据颜色设定进行风格自适应哦\faKissWinkHeart}
}
\cventry
[\Verb{\setbool{acvSectionColorHighlight}{true}}]
[Awesome-CV]
[指定是否使用配色凸显章节标题后紧跟的分割线]
[必选]
{
  \item 如果设定为true,责章节名称后面的长分割线会有颜色,否则为黑色。
}
\cventry
[\Verb{\renewcommand{\acvHeaderSocialSep}{\quad\textbar\quad}}]
[Awesome-CV]
[封面头部Logo右侧社交媒体帐号之间的分隔符定义]
[可选]
{
  \item 默认为管道符:\dkbutton{<空格>|<空格>}
}
\cventry[个人信息部分][Awesome-CV][该部分用来定义一些个人信息或文档信息][部分可选]
{
  \item \Verb{\photo[rectangle,noedge,left]{./src/resource/dukang-logo}}用于定义首页的Logo图片,文件扩展名默认为.png,可用的裁剪选项为circle(圆形)和rectangle(正方形),可用的边框选项为edge(有边框)和noedge(无边框),可用的位置选项为left(靠左)和right(靠右)。
  \item 该部分除了\Verb{\name}和\Verb{\dukangPDFTitle}是必选的以外,其他设定不需要的均可以注释掉,首页中相应的部分会自适应。
  \item {\color{awesome}由于\Verb{\name}在Awesome-CV文档类中有多处引用,所以必须指定,不能删掉。}
  \item {\color{awesome}\Verb{\dukangPDFTitle}用来生成PDF文件书签中的主标题,所以必须指定,不能删掉。}
}
\cventry[社交媒体信息部分][Awesome-CV][该部分用来定义社交媒体帐号或联系方式][部分可选]
{
  \item 该部分有若干社交媒体选项,可以根据需要进行定义,不需要的可以注释掉。
  \item 上面定义的分隔符\Verb{\acvHeaderSocialSep}就是用来分割这些帐号的。
  \item {\color{awesome}至少要保留一条,否则编译出错!}
}
\cventry[cvletter环境基本信息][Awesome-CV][cvletter环境一般用于定义首页的内容][必选]
{
  \item 由于使用了ctex宏包,\Verb{\today}默认为大写中文日期格式。
  \item {\color{awesome}该部分的定义一个都不能少!}
}
\cventry[dukang导言区设定部分][\dk][此部分包含{\dk}及相关宏包提供的若干增强设定][必选]
{
  \item {\color{awesome}由于{\dk}采用无侵入方式对Awesome-CV内容进行修改,因此该部分设定(包括引入宏包)都要出现在上述Awesome-CV相关定义完毕之后的导言区中,也就是\Verb{\begin{document}}之前!}
  \item \Verb{\usepackage{dukang}},首先引入{\dk}宏包,启用中文化及诸多增强功能。
  \item \Verb{\setbool{dukangParIndent}{true}},由于Awesome-CV大多数风格都使用了部件化(自定义命令或环境)来实现,没有使用chapter/section等标准结构,这直接导致了在引入ctex宏包进行中文化的时候,需要对每个部件进行单独的设定,比如首行缩进两字符对于有些部件要么不起作用,要么显示错乱,这里提供一个全局开关,会自动根据部件的具体情况有选择的开启首行缩进,以达到风格统一、显示美观的效果。
  \item \Verb{\setbool{dukangBookmarkLeadingNumber}{true}},Awesome-CV当前版本并不支持给输出的PDF文件按照文档结构自动添加书签(导航栏),{\dk}提供了这方面的支持,这个全局开关用来指定所添加的书签标题前,是否包含阿拉伯数字的章节编号。
  \item \Verb{\hypersetup},该部分用来为生成的PDF文件提供若干属性字段。
}
\cventry
[Awesome-CV文档区设定部分]
[Awesome-CV]
[该部分设定出现在文档区,也就是\Verb{\begin{document}和\end{document}之间。}]
[可选]
{
  \item \Verb{\makecvheader[R]},{\color{awesome}这不是页眉!Awesome-CV没有页眉。}这是首页包括Logo在内的抬头(Header)部分,可以注释掉,首页布局会自动从cvletter环境开始。可用选项用来控制对齐方向,L标识左对齐,C标识居中对齐,R标识右对齐。{\color{awesome}都要大写!}
  \item \Verb{\makecvfooter},这个是每页的页脚,分为左中右三个部分,每个部分都可以留空,但{\color{awesome}必须保留大括号\Verb{{}}}
}
\end{cventries}

以上是{\dk}当前版本设定部分的详细说明,需要格外注意的地方都有颜色高亮,右边红色的标签说明该选项来自哪个部分。并且,假如在修改的过程中不小心把main.tex搞乱了也没关系,可以随时打开dukang-doc.tex查看正确的配置,或者干脆把除了正文以外的所有内容复制回来,重新设定一下,就又可以开始创作了。这也是为什么我推荐用\Verb{\input{...}}把文档正文章节和main.tex主文件分开的原因。

\begin{dkcomment}*[温馨提示]
  无论何时,dukang-doc.tex都是你值得参考的示例文档,文档本身和其内容章节文件(特别是源代码)尽量涵盖到了{\dk}的全部功能,包括设定和功能模块等,随时可以回来查看。\faKissWinkHeart
\end{dkcomment}

\cvsubsection{使用流程}
\dk~的编译控制文件(Makefile)提供了适合下面几种场景的编译流程,相信总有一个适合你。首先,对于一般使用来说,需要做的步骤很简单:

\begin{center}
  \dkcodebox{修改main.tex}~\faArrowCircleRight~\dkcodebox{添加内容}~\faArrowCircleRight~\dkcodebox*{make}~\faCheckCircle
\end{center}

这样在src文件夹下就得到了main.pdf,同时在resource文件夹下,如果有符合命名规则的资源文件,也会被联动编译,并生成对应的PDF文件。{\dk}当前版本资源文件的命名规则是\dkbutton{r-*.tex},符合这个规则的.tex文件都会被自动编译和控制。

如果想要保留编译之后的PDF,同时把项目目录清理干净的话:

\begin{center}
  \dkcodebox{修改main.tex}~\faArrowCircleRight~\dkcodebox{添加内容}~\faArrowCircleRight~\dkcodebox*{make}~\faArrowCircleRight~\dkcodebox*{make clean}~\faCheckCircle
\end{center}

最后,如果希望只保留源代码\footnote{比如用于提交源代码,归档,或者你就是个纯粹的代码强迫症患者{\color{awesome}\faHeart}},把其他临时文件连同编译出来的东西一同干掉的话:

\begin{center}
  \dkcodebox{修改main.tex}~\faArrowCircleRight~\dkcodebox{添加内容}~\faArrowCircleRight~\dkcodebox*{make}~\faArrowCircleRight~\dkcodebox*{make cleanall}~\faCheckCircle
\end{center}

\begin{dkcomment}
  如果在编译过程中出现问题,强制退出编译过程之后想要再次编译,最好先执行\dkcodebox*{make cleanall}一遍,清理完所有临时文件之后再开始,否则编译很可能会出错无法继续下去。
\end{dkcomment}


\clearpage
