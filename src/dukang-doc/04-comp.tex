\cvsection{组件使用详解}
组件共分为两类,一类是Awesome-CV原生定义的组件,用于结构化排版文档,另一类是{\dk}为Awesome-CV添加的自定义组件\footnote{所谓组件,其本质上只是对一些宏包功能的再封装,以达到方便使用的目的,真正要感谢的是那些宏包的作者。}可以在作品中使用。以上两种组件,如果你是一路看到这里,相信已经见过它们中的绝大多数了。

由于语言环境不同,许多Awesome-CV的原生组件在进行中文化的过程中,需要对一些细节进行处理,{\dk}尽量以无侵入(重定义)的方式在不碰类文件的情况下,对这些原生组件进行了修改和部分增强。同时,{\dk}增加了许多自定义组件,有的是方便引入资源的,有的是生成表格的,有些提供了代码高亮,有些生成小按钮风格,它们都有一个共同的能力,就是可以随着文档定义的Awesome-CV主色调自动适应配色。

下面分两个部分分别对Awesome-CV原生组件和{\dk}自定义组件进行详细介绍。

\cvsubsection{Awesome-CV原生组件}
Awesome-CV没有使用{\LaTeX}传统的chapter/section组织结构,而是完全自定义了自己的组件用于支撑文档结构。这样做对于简历类型的文档当然更加灵活,但是如果想要用来进行文章或书籍的创作,就有些不够用了。而且对于中文环境排版来说,我们有着更加复杂的习惯和要求,比如首行缩进、行间距、断句、对齐、字体等,这些是Awesome-CV原生环境装进中文时一定会遇到的问题,虽然一部分能够被ctex宏包自动修正,但由于没有chapter/section等结构,面对非标准化的自定义环境和命令,ctex宏包的强大能力也无处施展。因此,{\dk}在这方面着重下了一番功夫,对绝大部分Awesome-CV原生组件进行了调整,并修复了一些我感觉像bug的地方\footnote{其实应该也不算bug,只是修改完之后在使用方面会更方便灵活},比如某些组件不能紧挨着,某些组件调用顺序不对会搞乱行间距或编译错误等等。

下面我们逐一列出这些经过修改和增强之后的Awesome-CV原生组件。

\begin{cventries}
  \cventry
  [cvletter \& lettersection]
  [信件环境]
  {
    \item 主要用于首页信件环境的风格定义,也就是看似一封信的显示效果。
    \item 由于{\Verb{\lettersection}}的显示效果与{\Verb{\cvsection}}类似,且只用在信件环境中,扩展意义不大,所以尽管在Awesome-CV的示例文件中使用了,但{\dk}推荐直接在{\Verb{\cvletter}}中书写正文,或者包含{\Verb{\cvsection}}部分。
    \item 使用方法可参考\dkbutton{./src/dukang-doc/00-cover.tex}
  }
  \cventry
  [cvparagraph]
  [段落环境]
  {
    \item 这是段落的环境封装,在正文中用的不多,因为{\dk}已经定义好了正文段落的样式,这个cvparagraph环境可以不用,直接书写正文就好。
  }
  \cventry
  [cvsection \& cvsubsection]
  [章节命令]
  {
    \item 两者是Awesome-CV自定义的章节命令,由于没有chapter,所以cvsection就是一级标题,其实对于简历风格的模板来说,这样设计也是合情合理的,只不过用来创作文章或书籍的话,章节层级就要分明一些。
    \item {\dk}修改了这两个命令的样式,增加了决定是否启用首行缩进的开关,在文档的主文件(main.tex)中可以指定,具体可以参考“dukang导言区设定部分”的备注说明。
    \item {\dk}为这两个命令增加了PDF书签功能,并可以根据全局开关设置是否显示书签标题前的章节编号。
  }
  \cventry
  [cventries \& cventry]
  [组件环境]
  {
    \item cventries环境包含若干cventry命令,本部分就是使用该组件进行书写。
    \item {\dk}使用可变参数重新封装了Awesome-CV的原生cventry,现在该组件共包含五个部分,命令格式为:
    \begin{center}
      \dkbutton{\textbackslash cventry[第一行左侧][第一行右侧][第二行左侧][第二行右侧]\{正文\}}
    \end{center}
    \item 参数中除“正文”以外都可选,但需要按顺序给定,否则无法判断是第几个参数,例如只想显示第二行右侧的文字,需要将前面三个参数都标记出来并留空,有内容的参数后面的空参数可以不标记,例如:
    \begin{center}
      \dkbutton{\textbackslash cventry[][][][第二行右侧]\{正文文字\}}
      或者
      \dkbutton{\textbackslash cventry[][第一行右侧]\{正文文字\}}
    \end{center}
  }
  \cventry
  [cvhonors \& cvhonor]
  [组件环境]
  {
    \item cvhonors环境包含若干cvhonor命令,其本质是表格。
    \item {\dk}重新封装了原生组件,现在cvhonors环境有两个可选参数,格式为:
    \begin{center}
      \dkbutton{\textbackslash begin\{cvhonors\}<*>[LLR]...\textbackslash end\{cvhonors\}}
    \end{center}
    \item 其中第一个参数星号“*”为底色开关,带星号标识启用奇偶行底色,不带星号为无底色,切换底色会少许改变外边距,组件会自动进行调整。
    \item 第二个参数“LLR”的三个字母表示接下来所有cvhonor组件左中右三部分的对齐方式,L表示左对齐,C标识居中对齐,R表示右对齐,不给出该参数的话,默认为“LLR”。
    \item cvhonor命令分为三个必选参数,可以留空但不能不写,格式为:
    \begin{center}
      \dkbutton{\textbackslash cvhonor\{左\}\{中\}\{右\}}
    \end{center}
  }
  \cventry
  [cvskills \& cvskill]
  [组件环境]
  {
    \item cvskills环境包含若干cvskill命令,其本质也是2列的表格。
    \item 这个组件应用场景不是很多,所以就没有进行全面修改,依然沿用了Awesome-CV原生的使用方法:
    \begin{center}
      \dkbutton{\textbackslash begin\{cvskills\}...\textbackslash end\{cvskills\}}
    \end{center}
    在环境内部包含若干\dkbutton{\textbackslash cvskill\{左边部分\}\{右边部分\}}
  }
\end{cventries}

以上是{\dk}当前版本进行过优化的所有Awesome-CV文档组件,其实其原生组件也大致就这么多了,下一小节我们看一下来自{\dk}的自定义组件。

\cvsubsection{ACV-Dukang自定义组件}
在Awesome-CV原生组件的基础上,{\dk}将几个比较强大的宏包封装成了一些较为通用的自定义组件,一是更加灵活使用方便,二是丰富了ACV在“简历”场景以外的功能,这些组件适合应用在特别是科技类或IT类文章和书籍场景中。先看一下{\dk}自定义组件的全家福:

\begin{cvhonors}*
  \cvhonor
    {\Verb{\dkbutton}}
    {Inline风格的代码小盒子,可选两种显示风格}
    {小按钮}
  \cvhonor
    {\Verb{\dkresource}}
    {引入外部图片或PDF的命令}
    {外部资源}
  \cvhonor
    {\Verb{\dkcode}}
    {可以对大段代码进行高亮显示的环境,可选两种显示风格}
    {代码高亮}
  \cvhonor
    {\Verb{\dkcodefile}}
    {读取外部文件并将内容进行代码高亮显示的环境,可选两种显示风格}
    {代码高亮}
  \cvhonor
    {\Verb{\dkcomment}}
    {可完全自定义显示的备注框,同样有两种显示风格可选}
    {备注框}
  \cvhonor
    {\Verb{\dkcodebox}}
    {另一种风格的Inline代码小盒子,可切换是否显示命令提示符}
    {深色背景}
  \cvhonor
    {\Verb{\dkmeasure}}
    {打印长度数值,主要用于在构建页面时对长度变量进行测量}
    {开发小工具}
  \cvhonor
    {\Verb{\dirtree}}
    {显示文件夹树状结构的宏包命令,{\dk}进行了引入,无需再封装}
    {直接使用}
\end{cvhonors}

这些组件都有自己的用法和适合的使用环境,以及一些注意事项,其中最重要的一条共性原则是:必选参数都使用大括号|{}|指定,可选参数都是方括号|[]|指定,由于{\LaTeXe}没有所谓“命名参数”的概念,因此所有的可选参数必须从右至左省略,举个例子,参数$1,2,3,4$,要想指定参数$3$的内容,必须连同$1,2$一起给出,否则引擎无法判断你到底给出了第几个参数。

\cvsubsubsection{dkbutton}
首先介绍这个小按钮,是因为在本文档中它用的最多,几乎随处可见,调用格式为:

|\dkbutton<*>[color]{text}|

\begin{center}
  \dkbutton{\textbackslash dkbutton\{这里有个小按钮\}} \& \dkbutton*{\textbackslash dkbutton*\{这里有圆形的小按钮\}}
\end{center}

color默认为自动适应在主文档中指定的Awesome-CV的主配色,也可以自行指定,比如:

\begin{center}
  \dkbutton[blue]{\textbackslash dkbutton[blue]\{蓝色小按钮\}} \& \dkbutton*[green]{\textbackslash dkbutton*[green]\{绿色圆形小按钮\}}
\end{center}

在使用方面,除了尽量以行内方式使用外几乎没有限制,同一行中别用太多就行,否则影响自动断行。当然如果想要一个超大的按钮也是可以的,只是推荐把它放在单独一个段落中,比如:

\begin{center}
  \vspace{1em}
  \dkbutton*{\parbox[c][5em]{20em}{\centering 一个巨大的圆形按钮}}
  \vspace{1em}
\end{center}

\cvsubsubsection{dkresource}
这个组件的作用是将外部资源(比如另一个PDF文档或PNG图片)引入当前位置居中显示,并可手动指定宽度自动缩放,调用格式为:

|\dkresource<[caption]>{file}<[width_factor]><[htbp!]>|

\begin{cvskills}*
  \cvskill
  {caption}
  {可选参数,资源下方的名称或标签。}
  \cvskill
  {file}
  {必选参数,需要引入的资源,可以是图片、PDF文件或其它$figure$环境支持的文件类型,只需给出路径及名称,无需加扩展名或后缀,相对路径的默认起点为\dkbutton{./src/}.}
  \cvskill
  {width\_factor}
  {可选参数,相对于整个行宽的占比,取值为0-1,默认为0.9倍行宽。}
  \cvskill
  {htbp!}
  {可选参数,用来控制资源在页面的浮动显示位置,其中$h$表示当前位置或代码所处的上下文位置,$t$表示Top(页面顶端),$b$表示Bottom(页面底部),$p$表示Page(单独成页),$!$表示在判定位置时忽略限制选项,默认为$htb$.}
\end{cvskills}

使用示例:

|\dkresource[这里有个酒坛子]{resource/dukang-logo}[0.25]|\footnote{resource文件夹的联动编译功能最主要目的也是为了支持dkresource.}

\dkresource[这里有个酒坛子]{resource/dukang-logo}[0.25]

\begin{dkcomment}*[关于浮动体的说明]
  \hspace{2em}$dkresource$实际上是基于{\LaTeXe} $figure$环境的简单再封装,让人比较晕的是$htbp!$浮动体控制参数的意义,简单理解的话就是,在生成PDF文件的时候,{\LaTeX}编译器会根据浮动体控制参数来自动判断并确定浮动体在页面中的显示位置,这里的浮动体指的就是$dkresource$所引入的资源,排版位置的选取与参数里符号的顺序并无关系,编译器总是以$h \to t \to b \to p$的顺序来检查参数。

  \hspace{2em}|!|的作用是忽略限制,这里的默认限制总共有两条,超出该限制会强制将浮动体拖入下一页再判断,这两条分别是:

  \begin{itemize}
    \item 个数:除$p$参数(单独成页)外,默认每页不超过3个浮动体,其中顶部$t$不超过2个,底部$b$不超过1个。
    \item 空间占比:默认顶部$t$不超过页面高度的70\%,底部$b$不超过30\%.
  \end{itemize}
\end{dkcomment}

\cvsubsubsection{dkcode}
这个组件通过封装$tcolorbox$宏包,使用$minted$引擎对计算机语言代码进行高亮显示,支持中文并调整行距字体显得更加美观,同时支持组件标题栏风格、代码高亮风格以及主配色的手动指定。用法为:

|\begin{dkcode}<*>{lang}<[title]><[color]><[minted_style]>...\end{dkcode}|

\begin{dkcomment}*[特别说明]
  $dkcode$环境以及$dkcodefile$命令默认使用$tcolorbox$的$minted$高质量引擎进行代码渲染,这意味着你需要提前准备好Python和Pygments库,详细安装说明在之前“编译使用详解”章节的“本地环境准备”小节中已经进行了阐述,遇到问题请随时检查本地环境是否满足编译条件。
\end{dkcomment}

\begin{cvskills}*
  \cvskill
  {*}
  {可选参数,不带星号为普通标题栏,带星号为瘦标题栏,两者只是标题风格不同。}
  \cvskill
  {lang}
  {必选参数,指定$minted$引擎支持的语言名称,例如$bash, python3, java, c, go$等,可选项请运行\dkbutton{pygmentize -L lexers}命令查看。}
  \cvskill
  {title}
  {可选参数,代码块的标题栏内容,如果没有该参数,则不显示标题栏,并自动忽略*参数。}
  \cvskill
  {color}
  {可选参数,指定主色调或省略进行自适应,默认为$awesome$}
  \cvskill
  {minted\_style}
  {可选参数,指定$minted$引擎预定义的高亮显示风格,默认为$tango$,$minted$预定义的高亮样式,默认为tango,可选项请运行\dkbutton{pygmentize -L styles}命令查看。}
\end{cvskills}

使用示例:

\begin{dkcode}*{latex}[带标题栏默认颜色]
\begin{dkcode}*{latex}[带标题栏默认颜色]
  ...
\end{dkcode}

\begin{dkcode}{latex}[带瘦标题栏蓝色vim风格][blue][vim]
\begin{dkcode}{latex}[带瘦标题栏蓝色vim风格][blue][vim]
  ...
\end{dkcode}

\begin{dkcode}{latex}
\begin{dkcode}{latex}
  % 无标题
\end{dkcode}

\cvsubsubsection{dkcodefile}
这个组件的功能和设定与|\dkcode|类似,都是代码高亮块,风格也一样,只不过多了一个必选参数,就是从外部文件中读取内容,比较适合多行代码,或者联动编译\footnote{本文档“编译使用详解”一节,就是在编译时使用dkcodefile直接读取的真实Makefile文件。}。具体用法是:

|\dkcodefile<*>{file}{lang}<[title]><[color]><[minted_style]>|

\begin{cvskills}*
  \cvskill
  {*}
  {可选参数,不带星号为普通标题栏,带星号为瘦标题栏,两者只是标题风格不同。}
  \cvskill
  {file}
  {必选参数,指定外部文件的位置,可以使用相对路径,起点跟$dkresource$的$file$参数一样,都是\dkbutton{./src/}.}
  \cvskill
  {lang}
  {必选参数,指定$minted$引擎支持的语言名称,例如$bash, python3, java, c, go$等,可选项请运行\dkbutton{pygmentize -L lexers}命令查看。}
  \cvskill
  {title}
  {可选参数,代码块的标题栏内容,如果没有该参数,则不显示标题栏,并自动忽略*参数。}
  \cvskill
  {color}
  {可选参数,指定主色调或省略进行自适应,默认为$awesome$}
  \cvskill
  {minted\_style}
  {可选参数,指定$minted$引擎预定义的高亮显示风格,默认为$tango$,$minted$预定义的高亮样式,默认为tango,可选项请运行\dkbutton{pygmentize -L styles}命令查看。}
\end{cvskills}

\begin{dkcomment}*[对于初学者来说容易犯的一个错误]
dkcode是环境,有开始标记|\begin|和结束标记|\end|,dkcodefile是命令,只有一行,没有开始和结束标记。
\end{dkcomment}

具体应用示例在这里就不展示了,感兴趣的话可以参考本文档“编译与使用详解”一节的源码,位置在:

\dkbutton*{./src/dukang-doc/03-usage.tex}

\cvsubsubsection{dkcomment}
该组件样式风格与$dkcode$和$dkcodefile$类似,只不过不带代码高亮能力,而是多了一个可以指定标题栏(如果有标题的话)图标的参数,主要用来当作备注栏,用法为:

|\begin{dkcomment}<*><[title]><[color]><[fontawesome]>...\end{dkcomment}|

\begin{cvskills}*
  \cvskill
  {*}
  {可选参数,不带星号为普通标题栏,带星号为瘦标题栏,两者只是标题风格不同。}
  \cvskill
  {title}
  {可选参数,代码块的标题栏内容,如果没有该参数,则不显示标题栏,并自动忽略*参数。}
  \cvskill
  {color}
  {可选参数,指定主色调或省略进行自适应,默认为$awesome$}
  \cvskill
  {fontawesome}
  {可选参数,$FontAwesome$宏包的图标代码或名称,默认为$\textbackslash faTheaterMasks$ \faTheaterMasks}
\end{cvskills}

关于$FontAwesome$宏包以及图标代码,可以访问\dkbutton*{\hyperref{https://fontawesome.com/}{}{}{Font Awesome}}官网,或者如果你的套装里有安装\dkbutton{texdoc},可以使用命令\dkcodebox*{texdoc fontawesome5}直接打开本地文档查阅。

使用示例:

\begin{dkcomment}*[瘦标题栏备注框]
  |\begin{dkcomment}*[瘦标题栏备注框]|\\
    ...\\
  |\end{dkcomment}|

  瘦标题栏风格是不带图标的
\end{dkcomment}

\begin{dkcomment}[默认标题栏备注框,指定颜色和图标][green][\faTree]
  |\begin{dkcomment}[默认标题栏备注框,指定颜色和图标][green][\faTree]|\\
    ...\\
  |\end{dkcomment}|

  指定颜色|green|和图标|\faTree|
\end{dkcomment}

\begin{dkcomment}
  |\begin{dkcomment}|\\
    ...\\
  |\end{dkcomment}|

  省略标题参数的话,就没有标题栏
\end{dkcomment}

\begin{dkcomment}[%
  \foreach \x in {39,36,...,3}{%
    \fontsize{\x}{1em}\faTree%
  }%
  \foreach \x in {3,6,...,39}{%
    \fontsize{\x}{1em}\faTree%
  }]%
  [cyan]%
  [\fontsize{42pt}{1em}\faTree]%
  一个发挥想象力的用法{\faHeart}

  下面的$foreach$循环控制命令需要用到$pgffor$宏包,{\dk}已经帮你引用好了。

  \begin{dkcode}*{latex}[代码在这里][cyan]
    \begin{dkcomment}[%
      \foreach \x in {39,36,...,3}{%
        \fontsize{\x}{1em}\faTree%
      }%
      \foreach \x in {3,6,...,39}{%
        \fontsize{\x}{1em}\faTree%
      }]%
      [cyan]%
      [\fontsize{42pt}{1em}\faTree]%
      一个发挥想象力的用法
    \end{dkcomment}
  \end{dkcode}
\end{dkcomment}

\cvsubsubsection{dkcodebox}

\cvsubsubsection{dkmeasure}

\cvsubsubsection{dirtree}

以上所有组件,基本可以满足所有日常创作,特别是科技类和IT类作品的创作需要,当然我会持续不断完善和扩展{\dk}的功能,后续组件会更加丰富。


\clearpage
