\cvsection{编译使用详细说明}
\dk~使用Makefile\footnote{前半部分的指令都有注释进行说明,make的用法也不在本文的讨论范围之内,感兴趣的朋友可以自行学习。}将几个最常用的{\LaTeX}编译相关操作定义成了对应的控制台命令,用以对编译、输出、清理等过程进行简化处理,以及联动resource文件夹下的Makefile文件。

\dkcodefile*{../Makefile}{makefile}[Makefile]

\begin{cvhonors}*
  \cvhonor
  {make}
  {自动编译输出main.tex,等同于make main}
  {编译输出}
  \cvhonor
  {make main}
  {同上}
  {编译输出}
  \cvhonor
  {make doc}
  {自动编译输出\dk~文档,也就是本文档的输出命令}
  {编译输出}
  \cvhonor
  {make resource}
  {使用resource文件夹下的Makefile编译输出所有支持的资源文件,等同于进入resource文件夹下进行make或make all}
  {关联编译}
  \cvhonor
  {make all}
  {一次性编译main.tex和\dk~文档,等同于\dkbutton{make main \&\& make doc}}
  {编译输出}
  \cvhonor
  {make clean}
  {自动清理所有临时文件和文件夹,包括主目录和所有子目录}
  {自动清理}
  \cvhonor
  {make cleanall}
  {在自动清理所有临时文件和文件夹的基础上,还会删除掉所有主目录下的PDF文件,并联动resource文件夹下所有生成的资源PDF}
  {自动清理}
\end{cvhonors}

\begin{dkcomment}
  \dkbutton{make cleanall}只会删除符合资源文件命名规则的PDF文件,其他文件不受影响。
\end{dkcomment}

\cvsubsection{make}
直接用于编译正文

\clearpage
