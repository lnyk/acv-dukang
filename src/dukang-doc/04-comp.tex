\cvsection{组件使用详解}
组件共分为两类,一类是Awesome-CV原生定义的组件,用于结构化排版文档,另一类是{\dk}为Awesome-CV添加的自定义组件\footnote{所谓组件,其本质上只是对一些宏包功能的再封装,以达到方便使用的目的,真正要感谢的是那些宏包的作者。}可以在作品中使用。以上两种组件,如果你是一路看到这里,相信已经见过它们中的绝大多数了。

由于语言环境不同,许多Awesome-CV的原生组件在进行中文化的过程中,需要对一些细节进行处理,{\dk}尽量以无侵入(重定义)的方式在不碰类文件的情况下,对这些原生组件进行了修改和部分增强。同时,{\dk}增加了许多自定义组件,有的是方便引入资源的,有的是生成表格的,有些提供了代码高亮,有些生成小按钮风格,它们都有一个共同的能力,就是可以随着文档定义的Awesome-CV主色调自动适应配色。

下面分两个部分分别对Awesome-CV原生组件和{\dk}自定义组件进行详细介绍。

\cvsubsection{Awesome-CV原生组件}
Awesome-CV没有使用{\LaTeX}传统的chapter/section组织结构,而是完全自定义了自己的组件用于支撑文档结构。这样做对于简历类型的文档当然更加灵活,但是如果想要用来进行文章或书籍的创作,就有些不够用了。而且对于中文环境排版来说,我们有着更加复杂的习惯和要求,比如首行缩进、行间距、断句、对齐、字体等,这些是Awesome-CV原生环境装进中文时一定会遇到的问题,虽然一部分能够被ctex宏包自动修正,但由于没有chapter/section等结构,面对非标准化的自定义环境和命令,ctex宏包的强大能力也无处施展。因此,{\dk}在这方面着重下了一番功夫,对绝大部分Awesome-CV原生组件进行了调整,并修复了一些我感觉像bug的地方\footnote{其实应该也不算bug,只是修改完之后在使用方面会更方便灵活},比如某些组件不能紧挨着,某些组件调用顺序不对会搞乱行间距或编译错误等等。

下面我们逐一列出这些经过修改和增强之后的Awesome-CV原生组件。

\begin{cventries}
  \cventry
    [cvletter \& lettersection]
    [信件环境]
    {
      \item 主要用于首页信件环境的风格定义,也就是看似一封信的显示效果。
      \item 由于{\Verb{\lettersection}}的显示效果与{\Verb{\cvsection}}类似,且只用在信件环境中,扩展意义不大,所以尽管在Awesome-CV的示例文件中使用了,但{\dk}推荐直接在{\Verb{\cvletter}}中书写正文,或者包含{\Verb{\cvsection}}部分。
      \item 使用方法可参考\dkbutton{./src/dukang-doc/00-cover.tex}
    }
\end{cventries}

\cvsubsection{ACV-Dukang自定义组件}
{\dk}当前版本提供的所有组件及功能说明列表如下:

\begin{cvhonors}*
  \cvhonor
    {\Verb{\dkbutton}}
    {inline风格的代码小盒子,可选两种显示风格}
    {小按钮}
  \cvhonor
    {\Verb{\dkresource}}
    {引入外部pdf或图片的命令}
    {外部资源}
  \cvhonor
    {\Verb{\dkcode}}
    {可以对大段代码进行高亮显示的环境,可选两种显示风格}
    {代码高亮}
  \cvhonor
    {\Verb{\dkcodefile}}
    {读取外部文件并将内容进行代码高亮显示的环境,可选两种显示风格}
    {代码高亮}
  \cvhonor
    {\Verb{\dkcomment}}
    {可完全自定义显示的备注框,同样有两种显示风格可选}
    {备注框}
  \cvhonor
    {\Verb{\dkcodebox}}
    {另一种风格的inline代码小盒子,可切换是否显示命令提示符}
    {深色背景}
  \cvhonor
    {\Verb{\dirtree}}
    {显示文件夹树状结构的宏包命令,{\dk}进行了引入,无需再封装}
    {直接使用}
\end{cvhonors}

以上所有组件,基本可以满足所有日常创作,特别是科技类和IT类作品的创作需要,当然我会持续不断完善和扩展{\dk}的功能,后续组件会更加丰富。


\clearpage
