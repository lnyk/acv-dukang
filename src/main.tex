%!TEX TS-program = xelatex
%!TEX encoding = UTF-8 Unicode

%-------------------------------------------------------------------------------
% 基本配置
%-------------------------------------------------------------------------------
% 默认为A4纸张,12pt字号
\documentclass[12pt, a4paper, final]{awesome-cv}

% 使用geometry定义纸张边距
\geometry{left=1.4cm, top=.8cm, right=1.4cm, bottom=1.8cm, footskip=.5cm}

% 高亮颜色配置
% Awesome Colors:
%   awesome-emerald, awesome-skyblue, awesome-red, awesome-pink, awesome-orange
%   awesome-nephritis, awesome-concrete, awesome-darknight
\colorlet{awesome}{awesome-red}
% 下面部分可启用自定义颜色
% \definecolor{awesome}{HTML}{CA63A8}

% Colors for text
% Uncomment if you would like to specify your own color
% \definecolor{darktext}{HTML}{414141}
% \definecolor{text}{HTML}{333333}
% \definecolor{graytext}{HTML}{5D5D5D}
% \definecolor{lighttext}{HTML}{999999}
% \definecolor{sectiondivider}{HTML}{5D5D5D}

% 全局开关变量定义
% 是否使用高亮颜色配置来突出section标题
\setbool{acvSectionColorHighlight}{true}

% Header中社交媒体行的分隔符
% 目前指定为管道符 |
\renewcommand{\acvHeaderSocialSep}{\quad\textbar\quad}


%-------------------------------------------------------------------------------
%	个人信息
%	不需要的部分可以注释掉
%-------------------------------------------------------------------------------
% 首页的图片Logo,默认为./src/resource/profile<.png>
% 可用选项为circle|rectangle,edge/noedge,left/right
\photo[circle,noedge,left]{./src/resource/profile}
\name{\LaTeX~Dukang}{手册}
\position{%
  \hyperref{https://github.com/posquit0/Awesome-CV}{}{}{Awesome-CV}{\enskip\faWineBottle\enskip}\dk
}
\address{基于\hyperref{https://github.com/posquit0/Awesome-CV}{}{}{Awesome-CV}进行中文化适应及无侵入增强的面向\LaTeX{}新人的快速开始项目}

%-------------------------------------------------------------------------------
% 以下部分至少有一条要保留
% \mobile{(+82) 10-9030-1843}
\email{me@williamyao.com}
% \dateofbirth{January 1st, 1970}
\homepage{WilliamYao.com}
\github{lnyk}
% \linkedin{posquit0}
% \gitlab{gitlab-id}
% \stackoverflow{SO-id}{SO-name}
% \twitter{@twit}
% \skype{skype-id}
% \reddit{reddit-id}
% \medium{madium-id}
% \kaggle{kaggle-id}
% \googlescholar{googlescholar-id}{name-to-display}
%% \firstname and \lastname will be used
% \googlescholar{googlescholar-id}{}
\extrainfo{{\faWineBottle}~衷心感谢\hyperref{https://github.com/posquit0}{}{}{Byungjin Park}等童鞋的开源奉献}

\quote{“执着而不计成本,不为索取只为陶醉”——~Carl Zeiss}


%-------------------------------------------------------------------------------
%	信件环境基本信息
%	所有内容必须不能缺少
%-------------------------------------------------------------------------------
% 收件方
\recipient
  {亲爱的\dk 使用者}
  {\hskip2em 本文档旨在用具体的示例全面介绍和展示\dk 宏包的使用、配置和附加功能,包括针对Awesome-CV模板进行的所有中文化处理和修改,并尽力以入门者的角度进行详细说明,希望通过阅读本文,能与诸君分享更多知识和收获。}
% 信件日期
\letterdate{\today}
% 信件标题
\lettertitle{关于快速开始}
% 信件称谓
\letteropening{如果迫不及待的想要使用\dk 进行创作,请直接按照下图查找并阅读项目文件吧\faKissWinkHeart}
% 信件结尾词
\letterclosing{怎么样,难道还意犹未尽吗?}
% 信件结尾词附言
\letterenclosure[亲爱的使用者]{继续发挥想像吧!}


%-------------------------------------------------------------------------------
% dukang导言区设定
%-------------------------------------------------------------------------------
% 引入dukang宏包
\usepackage{dukang}
%-------------------------------------------------------------------------------
% 打开该开关会启用某些正文内容的首行缩进
\setbool{dukangParIndent}{true}
% 是否为PDF书签添加标题前的标号
\setbool{dukangBookmarkLeadingNumber}{true}
%-------------------------------------------------------------------------------
% 定义编译之后的PDF相关属性
\hypersetup{%
  pdftitle={\LaTeX~Dukang~手册},
  pdfauthor={William Yao},
  pdfcreator={William Yao},
  pdfsubject={引用Awesome-CV模板,继承并加强},
  pdfkeywords={tex,latex,pdf,awesome,cv,resume,book,article,william,yao,dukang}
}
%-------------------------------------------------------------------------------


%-------------------------------------------------------------------------------
\begin{document}
%-------------------------------------------------------------------------------
% dukang文档区设定
%-------------------------------------------------------------------------------
% 首页的抬头
% 可选
% 可用对齐选项为C: 居中,L: 左对齐,R: 右对齐
\makecvheader[R]
%-------------------------------------------------------------------------------
% 每页的页脚定义,分为左中右三个部分,分别对应每个{}
% 各部分都可留空,但不能删掉{}
\makecvfooter
  {\hyperref{http://williamyao.com}{}{}{\LaTeX Dukang}}% 左边部分
  {\faWineBottle}% 中间部分
  {\thepage}% 右边部分


%-------------------------------------------------------------------------------
% 正文内容部分
% 独立引用每个文件,或者直接书写正文
%-------------------------------------------------------------------------------
% 信件环境的开头部分
% 可以注释掉,不显示该部分
\makelettertitle

%-------------------------------------------------------------------------------
%	信件环境正文
%-------------------------------------------------------------------------------
\begin{cvletter}

\lettersection{看一看总体结构}
\dkresource[htb]{resource/r-arch}{0.9}{}
\dk 沿用了Awesome-CV简洁方便的结构设计,除了负责总体编译命令的\dkboxr{Makefile}文件、$README.md$说明文件、$icon.png$项目图标以及$awesome-cv.cls$文档类本体以外,与文档写作有关的所有其他文档都放在$src/$文件夹中。其中,$awesome-cv.cls$文件本体以相对软链接\dkbox{ln -rs}的形式链接到$src/awesome-cv.cls$。

\begin{dkcomment}{测试dkcomment}{\faTree}
这里是dkcomment的正文,这里是dkcomment的正文,这里是dkcomment的正文,这里是dkcomment的正文,这里是dkcomment的正文,这里是dkcomment的正文,这里是dkcomment的正文,这里是dkcomment的正文,这里是dkcomment的正文,这里是dkcomment的正文,这里是dkcomment的正文,这里是dkcomment的正文,这里是dkcomment的正文,这里是dkcomment的正文,这里是dkcomment的正文,这里是dkcomment的正文,这里是dkcomment的正文,这里是dkcomment的正文,这里是dkcomment的正文,这里是dkcomment的正文。
\end{dkcomment}

\begin{dkcodeh}{python}{tango}{Test脚本}
import os
print("仍然没有成功解决CTeX环境下中英文之间的多余空格!")
print("Sigh... Need to solve the bug...")
pass
\end{dkcodeh}


\lettersection{Why Google?}
Suspendisse commodo, massa eu congue tincidunt, elit mauris pellentesque orci, cursus tempor odio nisl euismod augue. Aliquam adipiscing nibh ut odio sodales et pulvinar tortor laoreet. Mauris a accumsan ligula. Class aptent taciti sociosqu ad litora torquent per conubia nostra, per inceptos himenaeos. Suspendisse vulputate sem vehicula ipsum varius nec tempus dui dapibus. Phasellus et est urna, ut auctor erat. Sed tincidunt odio id odio aliquam mattis. Donec sapien nulla, feugiat eget adipiscing sit amet, lacinia ut dolor. Phasellus tincidunt, leo a fringilla consectetur, felis diam aliquam urna, vitae aliquet lectus orci nec velit. Vivamus dapibus varius blandit.

\lettersection{Why Me?}
Duis sit amet magna ante, at sodales diam. Aenean consectetur porta risus et sagittis. Ut interdum, enim varius pellentesque tincidunt, magna libero sodales tortor, ut fermentum nunc metus a ante. Vivamus odio leo, tincidunt eu luctus ut, sollicitudin sit amet metus. Nunc sed orci lectus. Ut sodales magna sed velit volutpat sit amet pulvinar diam venenatis.

\end{cvletter}


%-------------------------------------------------------------------------------
% 信件环境结尾部分
% 可以注释掉,不显示该部分
% \makeletterclosing

% 信件环境结束后开新页
\clearpage

\cvsection{结构设计总体介绍}
\begin{cvparagraph}
由于\dk~尽量使用非侵入式的方式与Awesome-CV进行集成,自然也沿用了其简洁方便的结构设计,除了控制总体编译的\dkbutton{Makefile}文件、\dkbutton{README.md}说明文件、\dkbutton{icon.png}项目图标以及\dkbutton{awesome-cv.cls}文档类本体以外,其他项目相关文件都放在\dkbutton{src/}中。其中为了更方便的操作,还使用了软链接\footnote{使用make以及软链接,对Windows平台和使用IDE的用户来说可能需要做一些适应性调整,这部分问题目前不在\dk~的代码设计范围之内,也许未来版本会考虑加入跨平台的设计要素。},比如:

\begin{dkcode}{bash}{tango}{使用软链接}
ln -rs ./awesome-cv.cls ./src/awesome-cv.cls
\end{dkcode}

\dkcodefile{bash}{tango}{使用软链接形式}{resource/test.py}
\end{cvparagraph}

\begin{dkcomment}{目录结构}{\faFolder}
这里是测试
\dirtree{%
  .1 TRT/.
  .2 \faFile\space main.tex\DTcomment{主文档的定义文件}.
  .2 ctex-fontset-custom.def\DTcomment{字体定义文件}.
  .2 customize.tex\DTcomment{自定义命令及环境}.
  .2 data/\DTcomment{存放所有主文档内容}.
  .3 cover.tex\DTcomment{封面、摘要及关键字}.
  .3 denotation.tex\DTcomment{主要符号对照表}.
  .3 ack.tex\DTcomment{致谢内容}.
  .3 resume.tex\DTcomment{个人简历}.
  .3 chap[*].tex\DTcomment{章节内容}.
  .3 appendix[*].tex\DTcomment{附录}.
  .2 docs/\DTcomment{TRT的相关文档}.
  .3 trt.pdf\DTcomment{TRT使用说明(本文档)}.
  .3 dirtree.pdf\DTcomment{dirtree命令官方文档}.
  .3 fontawesome.pdf\DTcomment{FontAwesome官方文档}.
  .2 fonts/\DTcomment{TRT用到的字体文件}.
  .2 figures/\DTcomment{存放所有图表文件}.
  .2 ref/refs.bib\DTcomment{参考文献}.
}
\end{dkcomment}

\begin{cvskills}

%---------------------------------------------------------
  \cvskill
    {DevOps} % Category
    {AWS, Docker, Kubernetes, Rancher, Vagrant, Packer, Terraform, Jenkins, CircleCI} % Skills

%---------------------------------------------------------
  \cvskill
    {Back-end} % Category
    {Koa, Express, Django, REST API} % Skills

%---------------------------------------------------------
  \cvskill
    {Front-end} % Category
    {Hugo, Redux, React, HTML5, LESS, SASS} % Skills

%---------------------------------------------------------
  \cvskill
    {Programming} % Category
    {Node.js, Python, JAVA, OCaml, LaTeX} % Skills

%---------------------------------------------------------
  \cvskill
    {Languages} % Category
    {Korean, English, Japanese} % Skills

%---------------------------------------------------------
\end{cvskills}

\clearpage


\end{document}
