%!TEX TS-program = xelatex
%!TEX encoding = UTF-8 Unicode


%-------------------------------------------------------------------------------
% 基本配置
%-------------------------------------------------------------------------------
% 以下设定需要保持在documentclass之前出现
% 传递table参数到xcolor包,用于tabular颜色支持
\PassOptionsToPackage{table}{xcolor}
%-------------------------------------------------------------------------------
% 默认为A4纸张,12pt字号
\documentclass[12pt, a4paper, final]{awesome-cv}

% 使用geometry定义纸张边距
\geometry{left=1.4cm, top=.8cm, right=1.4cm, bottom=1.8cm, footskip=.5cm}
%-------------------------------------------------------------------------------
% Aweosme-CV导言区设定部分
%-------------------------------------------------------------------------------
% 颜色配置,从下列套装中选择喜欢的配色
% Awesome Colors:
%   awesome-emerald, awesome-skyblue, awesome-red, awesome-pink, awesome-orange
%   awesome-nephritis, awesome-concrete, awesome-darknight
\colorlet{awesome}{awesome-red}
% 可启用自定义颜色
% \definecolor{awesome}{HTML}{CA63A8}
% 字体颜色,如果上面套装的字体颜色不喜欢,可以单独指定
% \definecolor{darktext}{HTML}{414141}
% \definecolor{text}{HTML}{333333}
% \definecolor{graytext}{HTML}{5D5D5D}
% \definecolor{lighttext}{HTML}{999999}
% \definecolor{sectiondivider}{HTML}{5D5D5D}
%-------------------------------------------------------------------------------
% 全局开关变量定义
% 是否使用高亮颜色配置来突出section标题
\setbool{acvSectionColorHighlight}{true}
%-------------------------------------------------------------------------------
% Header中社交媒体行的分隔符
% 目前指定为管道符 |
\renewcommand{\acvHeaderSocialSep}{\quad\textbar\quad}
%-------------------------------------------------------------------------------


%-------------------------------------------------------------------------------
% 文档内容信息部分
%-------------------------------------------------------------------------------
%	个人信息
%	不需要的部分可以注释掉
%-------------------------------------------------------------------------------
% 首页的图片Logo
% 可用选项为circle|rectangle,edge/noedge,left/right
\photo[rectangle,noedge,left]{./src/resource/dukang-logo}
\name{\LaTeX~Dukang}{手册}
\position{%
  \hyperref{https://github.com/posquit0/Awesome-CV}{}{}{Awesome-CV}{\enskip\faWineBottle\enskip}\dk
}
\address{基于\hyperref{https://github.com/posquit0/Awesome-CV}{}{}{Awesome-CV}进行中文化适应及无侵入增强的面向\LaTeX{}新人的快速开始项目}

%-------------------------------------------------------------------------------
% 以下部分至少有一条要保留
% \mobile{(+82) 10-9030-1843}
\email{me@williamyao.com}
% \dateofbirth{January 1st, 1970}
\homepage{WilliamYao.com}
\github{lnyk}
% \linkedin{posquit0}
% \gitlab{gitlab-id}
% \stackoverflow{SO-id}{SO-name}
% \twitter{@twit}
% \skype{skype-id}
% \reddit{reddit-id}
% \medium{madium-id}
% \kaggle{kaggle-id}
% \googlescholar{googlescholar-id}{name-to-display}
%% \firstname and \lastname will be used
% \googlescholar{googlescholar-id}{}
\extrainfo{{\faWineBottle}~衷心感谢\hyperref{https://github.com/posquit0}{}{}{Byungjin Park}等童鞋的开源奉献}

\quote{“执着而不计成本,不为索取只为陶醉”——~Carl Zeiss}
%-------------------------------------------------------------------------------
% 信件环境基本信息
% 所有内容都是必选,一个也不能少
%-------------------------------------------------------------------------------
% 收件方
\recipient
  {偶然到访或是兴趣所致的你}
  {{\LaTeX}, awesome-cv, {\dk}, template\\github, opensource}
% 信件日期
\letterdate{\today}
% 信件标题
\lettertitle{{\dk}}
% 信件称谓
\letteropening{亲爱的朋友:}
% 信件结尾词
\letterclosing{{\dk}项目Logo和名称的含义,相信你懂的,生活不易,何以解忧?\faKissWinkHeart}
% 信件结尾词附言
\letterenclosure[最后]{对你的关注,再次表示由衷的感谢!}
%-------------------------------------------------------------------------------


%-------------------------------------------------------------------------------
% dukang导言区设定部分
%-------------------------------------------------------------------------------
% 引入dukang宏包
\usepackage{dukang}
%-------------------------------------------------------------------------------
% 打开该开关会启用某些正文内容的首行缩进
\setbool{dukangParIndent}{true}
% 是否为PDF书签添加标题前的标号
\setbool{dukangBookmarkLeadingNumber}{true}
%-------------------------------------------------------------------------------
% 定义编译之后的PDF相关属性
\hypersetup{%
  pdftitle={\LaTeX~Dukang~手册},
  pdfauthor={William Yao},
  pdfcreator={William Yao},
  pdfsubject={引用Awesome-CV模板,继承并加强},
  pdfkeywords={tex,latex,pdf,awesome,cv,resume,book,article,william,yao,dukang}
}
%-------------------------------------------------------------------------------


%-------------------------------------------------------------------------------
\begin{document}
%-------------------------------------------------------------------------------
% Awesome-CV文档区设定部分
%-------------------------------------------------------------------------------
% 文档开头的抬头部分,可以注释掉
% 可用对齐选项为C: 居中,L: 左对齐,R: 右对齐
\makecvheader[R]
%-------------------------------------------------------------------------------
% 每页的页脚定义,分为左中右三个部分,分别对应每个{}
% 各部分都可留空,但不能删掉{}
\makecvfooter
  {\hyperref{http://williamyao.com}{}{}{\LaTeX Dukang}}% 左边部分
  {\faWineBottle}% 中间部分
  {\thepage}% 右边部分
%-------------------------------------------------------------------------------


%-------------------------------------------------------------------------------
% 正文内容部分
% 独立引用每个文件,或者直接书写正文
%-------------------------------------------------------------------------------
% 信件环境的开头部分
% 可以注释掉,不显示该部分
\makelettertitle\newline
\begin{cvletter}
{\hspace{2em}}无论你是偶然访问到这里,还是正在搜索你感兴趣的项目,我在这里都表示由衷的感谢。

{\hspace{2em}}Awesome-CV-Dukang项目,简称{\dk},是基于一款叫做\hyperref{https://github.com/posquit0/Awesome-CV}{}{}{Awesome-CV}的{\LaTeX}模板,进行中文化适应、额外提供许多新功能扩展,并面向入门使用者打包构建的快速开始项目。此篇文档旨在尽量以入门者的角度详细介绍{\dk}项目涵盖的所有内容,比如使用、配置和附加功能,希望通过阅读本文,能与诸君分享更多知识和收获。
\end{cvletter}

% 信件环境结尾部分
% 可以注释掉,不显示该部分
\makeletterclosing\newline
\clearpage

\cvsection{总体介绍}
Awesome-CV本身设计非常优秀,但由于语言习惯等差别,其对中文环境的支持和适配需要用户自行调整很多东西,同时其原作者的本意主要是聚焦在构建“简历”模板,鄙人以为,如此漂亮的设计,理应延伸至文章甚至书籍创作领域。在这个层面上,需要更加宏观的对项目中每个部件的细节进行调整,并引入适合文章或书籍创作的功能模块,而且相对于漂亮的设计和丰富的功能,{\LaTeX}过高的使用门槛却一直阻挡在追求严谨排版和高质量输出,却不是特别习惯看似古怪的语法、复杂难记的命令环境和众多宏包的朋友面前,“劝退”很多有志之士踏进{\LaTeX}的神奇世界。

尝试使用ctex宏包对Awesome-CV进行中文环境适配,对相应的部件和元素进行调整,对一些代码上的设计进行改进,并增加一些封装好的开箱即用的实用部件,是{\dk}项目的主要目的,将{\LaTeX}方面的一些知识表述清楚,将维护该项目学习到的一些技巧分享出去,也是我的初衷。

罗嗦这么多,还是赶快让我们进入主题吧。

\cvsubsection{主体结构}
{\dk}的结构大致上分为三个逻辑部分,第一部分是来自Awesome-CV项目的文档类,定义了所生成文件的最重要的内容,第二部分是本项目新增的宏包、扩展内容、资源文件夹以及编译控制文件,第三部分是提供给使用者作为快速开始模板的文件,当然还有一些项目维护所涉及的例如README、LICENSE、Git相关文件等。

下图列举了项目主体结构和主要的文件所在位置,其中来自本项目的所有文本类文件,都有详细的备注和说明信息,如果想要{\color{awesome}快速开始},不妨按照下图找到这些文件,阅读里面的源代码,寻找自己感兴趣的部分。

\dkresource{resource/r-arch}[0.9]

\begin{dkcomment}
  项目其他子文件夹和一些辅助性的文件,比如LOGO等没有在上图中体现。
\end{dkcomment}

\cvsubsection{快速体验与兼容性测试}

在深入介绍以前,有必要先体验一下编译输出的过程,同时也可以达到测试兼容性的目的。

\begin{dkcode}*{bash}[打开控制台依次执行]
# 进入项目根目录
cd latex-dukang

# 编译生成你的大作
make

# 编译生成本文档
make doc
\end{dkcode}

整个编译的过程,会输出大量的编译信息和告警内容,如果看到\dkbutton{Warning},先别着急,在{\LaTeX}的世界里,有众多的叫做“宏包”的贡献者在共同的支撑着你的创作,并不是所有的告警信息都意味着你“哪里做错了”,有些只是“善意的提醒”,比如{\dk}当前版本会有“未找到斜体字形”、“hbox overfull/underfull”等告警,虽然这类告警一定意味着“编译质量不怎么高”,但实际情形并不影响最终的输出结果,这些告警在后续的章节会有详细的说明。

回到上面的命令,如果执行顺利的话,第一个make执行完毕后,都会有类似下面的内容:

\vspace{1em}
\begin{center}
  \dkcodebox*{Latexmk: All targets (src/main.xdv src/main.pdf) are up-to-date}
\end{center}
\vspace{1em}

这说明main.tex文件已经成功被编译输出到了main.pdf,可以打开这个PDF文件看看实际的效果。第二个make是用来生成{\dk}文档(也就是本文)的,也跑一跑编译吧,确保你看到下面的提示信息。

\vspace{1em}
\begin{center}
  \dkcodebox*{Latexmk: All targets (src/dukang-doc.xdv src/dukang-doc.pdf) are up-to-date}
\end{center}
\vspace{1em}

如果,你碰到了光标停在一个问号后面,画面信息不再滚动的情形,那直接就是“错误”而不是“告警”了,这时候需要你向上查看具体哪里出了问题,或者输入大写的“X”并回车退出编译过程。这时候一般是没有main.pdf或dukang-doc.pdf输出出来的,请仔细检查输出日志中的错误提示,尝试着解决问题,并在\dkcodebox*{make cleanall}清理完所有临时文件之后再次尝试make。

\begin{dkcomment}[关于兼容性]
  {\hspace{2em}}{\dk}使用TexLive 2022套件在Ubuntu 20.04平台上开发、维护以及编译测试,使用其他版本套件或编译组件进行操作的话,目前还没有更加详细的测试结果,也衷心希望能够得到你的使用情况反馈,请随时邮件我。
\end{dkcomment}


\clearpage

\cvsection{结构设计}
由于{\dk}尽量使用非侵入式的方式与Awesome-CV进行集成,自然也沿用了其简洁方便的结构设计,除了总体编译文件、项目说明文件、项目图标以及文档类本体以外,其他的内容相关文件都放在\dkbutton{./src/}中。其中为了更方便的操作,部分文件使用了软链接\footnote{使用GNU Make以及软链接,对Windows系统和使用IDE的用户来说可能需要做一些适应性调整,详情请阅读“编译使用详解”章节的“本地环境准备”部分。}。

\begin{dkcomment}*[项目结构说明]
\dirtree{%
.1 \dk.
.2 \faFileCode~awesome-cv.cls\DTcomment{Awesome-CV的类文件}.
.2 \faFileImage~icon.png\DTcomment{项目图标}.
.2 \faFileCode~Makefile\DTcomment{控制编译命令的make文件}.
.2 \faMarkdown~README.md\DTcomment{项目说明文件}.
.2 \faFolder~src\DTcomment{内容相关文件}.
.3 \faLink~awesome-cv.cls\DTcomment{Awesome-CV类文件的软链接}.
.3 \faFileCode~ctex-fontset-custom.def\DTcomment{ctex的自定义字体文件}.
.3 \faFileCode~acv-dukang.cls\DTcomment{\dk~的主文件}.
.3 \faFileCode~dukang-doc.tex\DTcomment{本文档的主文件}.
.3 \faFolder~dukang-doc\DTcomment{存放本文档的章节内容文件}.
.3 \faFileCode~main.tex\DTcomment{从此文件入手开始创作}.
.3 \faFolder~main\DTcomment{存放main.tex引用的正文内容}.
.3 \faFolder~resource\DTcomment{存放图形、表格、外部pdf或tex等资源文件}.
.4 \faLink~ctex-fontset-custom.def\DTcomment{ctex自定义字体文件的软链接}.
.4 \faFileCode~Makefile\DTcomment{对应resource文件夹的make文件}.
.4 \faFileImage~logo.png\DTcomment{文档首页头信息中引用的图片文件}.
.4 \faFileCode~r-*.tex\DTcomment{所有以\dkbutton{r-*.tex}形式命名的文件都会支持联动编译控制}.
}
\end{dkcomment}

\begin{cventries}
\cventry[awesome-cv.cls][文档类]{
  \item {该文件属于Awesome-CV的文档类文件,{\dk}主要也是通过对该文件进行重定义来达到中文支持的目的。}
  \item {由于是无侵入方式,在没有发生重大变化的前提下,应该可以\dkbutton*{\hyperref{https://github.com/posquit0/Awesome-CV}{}{}{下载}}该文件的最新版本后直接覆盖来升级。}
}
\cventry[Makefile][编译控制]{
  \item {控制编译过程、简化项目使用操作的GNU Make文件。}
  \item {在Awesome-CV原有基础上,根据{\dk}所引入的部分新功能进行了增强。}
  \item {\color{awesome}不建议修改此文件,除非你知道自己在做什么。}
}
\cventry[src/awesome-cv.cls和src/resource/ctex-fontset-custom.def][软链接]{
  \item {这两个文件都是Linux平台的相对位置软链接。}
  \item {软链接使用\dkbutton{ln -rs}定义。}
}
\cventry[src/ctex-fontset-custom.def][字体定义]{
  \item {配合ctex宏包的\dkbutton{fontset=custom}选项来使用,可以修改这个文件中的字体设定。}
  \item {根据ctex的字体自定义设置,这里的\dkbutton{custom}对应ctex-fontset-\dkbutton{custom}.def,可以根据需要创建自己的自定义文件。}
}
\cventry[src/acv-dukang.cls][{\dk}主文件]{
  \item {该文件是{\dk}的文档类主文件,在文档中使用\dkbutton{\textbackslash documentclass}进行引用。}
  \item {所有修改以及增强的内容在源代码中都有详细的注释,想要更深入底层了解的话可以直接看该文件源码。}
  \item
  {之所以没有使用Doc和DocStrip构建Class文件,是因为我不会{\faMailchimp}或者只是单纯的不想把简单问题复杂化,毕竟{\dk}本身没有多少代码,内容也不复杂,况且已经配有详细的文档说明,没必要再生成一套“干货”,又不是通用宏包,不会提交给CTAN{\faMehBlank}}
}
\cventry[src/dukang-doc.tex和src/dukang-doc/][{\dk}说明文档]{
  \item {{\dk}说明文档(本文)主文件和存放章节内容文件的目录,同时也包含了用法示例。}
  \item {由于受Makefile编译控制,dukang-doc.tex的位置和文件名不要随便更改,但是dukang-doc文件夹中的内容文件都是由\dkbutton{\textbackslash input\{...\}}引入到主文档中,所以只要对应好文件位置,这个文件夹是可以自由修改的。}
  \item {\color{awesome}由于上述文件只用于生成{\dk}的说明文档(本文),与用户作品无关,所以不建议改动。}
}
\cventry[src/main.tex和src/main/][用户作品文件]{
  \item {用户作品的起始模板文件,可以直接从该文件开始进行创作。}
  \item {文件夹\dkbutton{./src/main/}与编译过程无关,仅用于存放起始模板文件所引用的内容文件,可以更改,甚至不用也行。}
  \item {如果作品的篇幅较长,还是建议使用\dkbutton{\textbackslash input\{...\}}的方式,可以保证主文档的内容干净整洁。}
}
\cventry[src/resource/][资源文件夹]{
  \item {用来统一存放图形、表格、外部pdf或tex等资源文件,并支持部分文件的联动编译。}
  \item {Makefile是与外层Makefile联动的编译控制文件,同时提供了部分针对资源目录中不同类型文件的编译命令。}
  \item {logo.png是封面上部显示的图片,在主文档的导言区中使用\dkbutton{\textbackslash photo\{...\}}进行定义。}
  \item {该文件夹被dukang-doc和用户作品共用。}
  \item {所有以\dkbutton{r-*.tex}形式命名的文件都会被联动编译和控制,本文档封面中Tikz图的源文件\dkbutton{r-arch.tex}就在这里。}
  \item {\color{awesome}不要更改该文件夹的位置,否则联动编译控制将找不到文件。}
}
\end{cventries}

\clearpage

\cvsection{编译使用详细说明}
\dk~使用\dkbutton{Makefile}将几个最常用的{\LaTeX}编译相关操作定义成了对应的控制台命令,用以对编译、输出、清理等过程进行简化处理,以及联动\dkbutton{Resource}文件夹下的\dkbutton{Makefile}文件。

{\today}版的Makefile\footnote{前半部分的指令都有注释进行说明,make的用法也不在本文的讨论范围之内,感兴趣的朋友可以自行学习。}文件源码如下:

\dkcodefile{makefile}{tango}{Makefile}{../Makefile}

\begin{cvhonors}*
  \cvhonor
  {make}
  {自动编译输出main.tex,等同于make main}
  {编译输出}
  \cvhonor
  {make main}
  {同上}
  {编译输出}
  \cvhonor
  {make doc}
  {自动编译输出\dk~文档,也就是本文档的输出命令}
  {编译输出}
  \cvhonor
  {make resource}
  {使用resource文件夹下的Makefile编译输出所有支持的资源文件,等同于进入resource文件夹下进行make或make all}
  {关联编译}
  \cvhonor
  {make all}
  {一次性编译main.tex和\dk~文档,等同于\dkbutton{make main \&\& make doc}}
  {编译输出}
\end{cvhonors}

\cvsubsection{make}
直接用于编译正文


\clearpage

%-------------------------------------------------------------------------------


\end{document}
