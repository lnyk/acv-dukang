\cvsection{结构设计总体介绍}
由于\dk~尽量使用非侵入式的方式与Awesome-CV进行集成,自然也沿用了其简洁方便的结构设计,除了总体编译文件、项目说明文件、项目图标以及文档类本体以外,其他的内容相关文件都放在\dkbutton{src/}中。其中为了更方便的操作,部分文件使用了软链接\footnote{使用make以及软链接,对Windows平台和使用IDE的用户来说可能需要做一些适应性调整,这部分问题目前不在\dk~的代码设计范围之内,也许未来版本会考虑加入跨平台的设计要素。}。

\begin{dkcomment}{项目结构说明}{\faFolder}
\dirtree{%
.1 \dk.
.2 \faFileCode~awesome-cv.cls\DTcomment{Awesome-CV的类文件}.
.2 \faFileImage~icon.png\DTcomment{项目图标}.
.2 \faFileCode~Makefile\DTcomment{控制编译命令的make文件}.
.2 \faMarkdown~README.md\DTcomment{项目说明文件}.
.2 \faFolder~src\DTcomment{内容相关文件}.
.3 \faLink~awesome-cv.cls\DTcomment{Awesome-CV类文件的软链接}.
.3 \faFileCode~ctex-fontset-custom.def\DTcomment{ctex的自定义字体文件}.
.3 \faFileCode~dukang.sty\DTcomment{\dk~的主文件}.
.3 \faFileCode~dukang-doc.tex\DTcomment{本文档的主文件}.
.3 \faFileCode~main.tex\DTcomment{从此文件入手开始创作}.
.3 \faFolder~dukang-doc\DTcomment{存放本文档的章节内容文件}.
.3 \faFolder~resource\DTcomment{存放图形、表格、外部pdf或tex等资源文件}.
.4 \faLink~ctex-fontset-custom.def\DTcomment{ctex自定义字体文件的软链接}.
.4 \faFileCode~Makefile\DTcomment{对应resource文件夹的make文件}.
.4 \faFileImage~profile.png\DTcomment{文档首页头信息中引用的图片文件}.
.4 \faFileCode~r-*.png\DTcomment{所有以\dkbutton{r-*.tex}形式命名的文件都会支持自动编译和清理}.
.3 \faFolder~tex\DTcomment{存放被\dkbutton{main.tex}引用的内容}.
}
\end{dkcomment}

先快速简要的说明一下部分文件和位置的注意事项,详细内容会在后续章节中进行详细说明。

\begin{cventries}

%---------------------------------------------------------
  \cventry
    {ctex-fontset-custom.def}
    {字体定义}
    {}
    {}
    {
      \begin{cvitems} % Description(s) of tasks/responsibilities
        \item {配合ctex的\dkbutton{fontset=custom}设定来使用,可以根据需要自行调整字体文件。}
        \item {Built fully automated CI/CD pipelines on CircleCI for containerized applications using Docker, AWS ECR and Rancher.}
        \item {Designed an overall service architecture and pipelines of the Machine Learning based Fashion Tagging API SaaS product with the micro-services architecture.}
        \item {Implemented several API microservices in Node.js Koa and in the serverless AWS Lambda functions.}
        \item {Deployed a centralized logging environment(ELK, Filebeat, CloudWatch, S3) which gather log data from docker containers and AWS resources.}
        \item {Deployed a centralized monitoring environment(Grafana, InfluxDB, CollectD) which gather system metrics as well as docker run-time metrics.}
      \end{cvitems}
    }

%---------------------------------------------------------
  \cventry
    {Freelance Penetration Tester} % Job title
    {SAMSUNG Electronics} % Organization
    {S.Korea} % Location
    {Sep. 2013, Mar. 2011 - Oct. 2011} % Date(s)
    {
      % \begin{cvitems} % Description(s) of tasks/responsibilities
      %   \item {Conducted penetration testing on SAMSUNG KNOX, which is solution for enterprise mobile security.}
      %   \item {Conducted penetration testing on SAMSUNG Smart TV.}
      % \end{cvitems}
      \begin{cvsubentries}
       \cvsubentry{}{KNOX(Solution for Enterprise Mobile Security) Penetration Testing}{Sep. 2013}{}
       \cvsubentry{}{Smart TV Penetration Testing}{Mar. 2011 - Oct. 2011}{}
      \end{cvsubentries}
    }

%---------------------------------------------------------
\end{cventries}

\begin{cvskills}
\cvskill
{ctex-fontset-custom.def}
{配合ctex的\dkbutton{fontset=custom}设定来使用,可以根据需要自行调整字体文件。}

\cvskill
{Back-end} % Category
{Koa, Express, Django, REST API} % Skills

\cvskill
{Front-end} % Category
{Hugo, Redux, React, HTML5, LESS, SASS} % Skills

\cvskill
{Programming} % Category
{Node.js, Python, JAVA, OCaml, LaTeX} % Skills

\cvskill
{Languages} % Category
{Korean, English, Japanese} % Skills

\cvskill
{Back-end} % Category
{Koa, Express, Django, REST API} % Skills

\cvskill
{Front-end} % Category
{Hugo, Redux, React, HTML5, LESS, SASS} % Skills

\cvskill
{Programming} % Category
{Node.js, Python, JAVA, OCaml, LaTeX} % Skills

\cvskill
{Languages} % Category
{Korean, English, Japanese} % Skills
\end{cvskills}

\clearpage
