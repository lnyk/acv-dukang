% !TeX encoding = UTF-8
% !TeX program = xelatex
% !TeX spellcheck = en_US

%% dukang-doc.tex
%% Copyright 2022 William Yao <lnyk@me.com>
%
% This work may be distributed and/or modified under the
% conditions of the LaTeX Project Public License, either version 1.3
% of this license or (at your option) any later version.
% The latest version of this license is in
%   http://www.latex-project.org/lppl.txt
% and version 1.3 or later is part of all distributions of LaTeX
% version 2005/12/01 or later.
%
% This work has the LPPL maintenance status `maintained'.
%
% The Current Maintainer of this work is William Yao <lnyk@me.com>
%
% This work consists of the files acv-dukang.cls and dukang-doc.tex
% and content files dukang-doc/* and associated project files such as
% Makefiles.


%-------------------------------------------------------------------------------
% 基本配置
%-------------------------------------------------------------------------------
% 默认为A4纸张,12pt字号,其实这里的字体设置基本没有效果,Awesome-CV每个部件都有自己的定义
\documentclass[12pt, a4paper, final]{acv-dukang}
%-------------------------------------------------------------------------------
% 使用geometry定义纸张边距
\geometry{left=1.4cm, top=.8cm, right=1.4cm, bottom=1.8cm, footskip=14.5pt}
%-------------------------------------------------------------------------------
% Aweosme-CV导言区设定部分
%-------------------------------------------------------------------------------
% 颜色配置,从下列套装中选择喜欢的配色
% Awesome Colors:
%   awesome-emerald, awesome-skyblue, awesome-red, awesome-pink, awesome-orange
%   awesome-nephritis, awesome-concrete, awesome-darknight
\colorlet{awesome}{awesome-red}
% 如果上面套装的字体颜色不喜欢,可以单独指定
% \definecolor{awesome}{HTML}{CA63A8}
% \definecolor{darktext}{HTML}{414141}
% \definecolor{text}{HTML}{333333}
% \definecolor{graytext}{HTML}{5D5D5D}
% \definecolor{lighttext}{HTML}{999999}
% \definecolor{sectiondivider}{HTML}{5D5D5D}
%-------------------------------------------------------------------------------
% 全局开关变量定义
% 是否使用高亮颜色配置来突出section标题
\setbool{acvSectionColorHighlight}{true}
%-------------------------------------------------------------------------------
% Header中社交媒体行的分隔符
% 目前指定为管道符 |
\renewcommand{\acvHeaderSocialSep}{\quad\textbar\quad}
%-------------------------------------------------------------------------------


%-------------------------------------------------------------------------------
% 文档内容信息部分
%-------------------------------------------------------------------------------
%	个人信息
%	除了\name和\dukangPDFTitle必选以外,其余部分不需要的都可以注释掉
%-------------------------------------------------------------------------------
% 这个必须留着!
\name{ACV-Dukang}{手册}
\dukangPDFTitle{ACV-Dukang手册}
% 首页的图片Logo
% 可用选项为circle|rectangle,edge/noedge,left/right
\photo[rectangle,noedge,left]{./src/resource/dukang-logo}
\position{%
  \hyperref{https://github.com/posquit0/Awesome-CV}{}{}{Awesome-CV}{\enskip\faWineBottle\enskip}\dk
}
\address{基于\hyperref{https://github.com/posquit0/Awesome-CV}{}{}{Awesome-CV}进行中文化适应及无侵入增强的面向\LaTeX{}新人的快速开始项目}
%-------------------------------------------------------------------------------
% 社交媒体信息
% 至少有一条要保留
%-------------------------------------------------------------------------------
% \mobile{(+82) 10-9030-1843}
\email{lnyk@me.com}
% \dateofbirth{January 1st, 1970}
% \homepage{github.com/lnyk}
\github{lnyk}
% \linkedin{posquit0}
% \gitlab{gitlab-id}
% \stackoverflow{SO-id}{SO-name}
% \twitter{@twit}
% \skype{skype-id}
% \reddit{reddit-id}
% \medium{madium-id}
% \kaggle{kaggle-id}
% \googlescholar{googlescholar-id}{name-to-display}
%% 如果留空,会自动使用\firstname和\lastname
% \googlescholar{googlescholar-id}{}
\extrainfo{{\faWineBottle}~衷心感谢\hyperref{https://github.com/posquit0}{}{}{Byungjin Park}等童鞋的开源奉献}
\quote{“执着而不计成本,不为索取只为陶醉”——~Carl Zeiss}
%-------------------------------------------------------------------------------
% cvletter环境基本信息
% 所有内容都是必选,一个也不能少
%-------------------------------------------------------------------------------
% 收件方
\recipient
  {偶然到访或是兴趣所致的你}
  {{\LaTeX}, awesome-cv, {\dk}, template\\github, opensource}
% 信件日期
\letterdate{\today}
% 信件标题
\lettertitle{{\dk}}
% 信件称谓
\letteropening{亲爱的朋友:}
% 信件结尾词
\letterclosing{{\dk}项目Logo来自互联网,如果冒犯请随时联系我纠正。生活不易,何以解忧?\faKissWinkHeart}
% 信件结尾词附言
\letterenclosure[最后]{对你的关注,再次表示由衷的感谢!}
%-------------------------------------------------------------------------------


%-------------------------------------------------------------------------------
% dukang导言区设定部分
%-------------------------------------------------------------------------------
% 打开该开关会启用某些正文内容的首行缩进
\setbool{dukangParIndent}{true}
% 是否为PDF书签添加标题前的标号
\setbool{dukangBookmarkLeadingNumber}{true}
%-------------------------------------------------------------------------------
% 定义编译之后的PDF相关属性
\hypersetup{%
  pdftitle={Awesome-CV-Dukang手册},
  pdfauthor={William Yao},
  pdfcreator={William Yao},
  pdfsubject={引用Awesome-CV模板,继承并加强},
  pdfkeywords={tex,latex,pdf,awesome,cv,resume,book,article,william,yao,dukang}
}
%-------------------------------------------------------------------------------


%-------------------------------------------------------------------------------
\begin{document}
%-------------------------------------------------------------------------------
% Awesome-CV文档区设定部分
%-------------------------------------------------------------------------------
% 文档开头的抬头部分,可以注释掉
% 可用对齐选项为C: 居中,L: 左对齐,R: 右对齐
\makecvheader[R]
%-------------------------------------------------------------------------------
% 每页的页脚定义,分为左中右三个部分,分别对应每个{}
% 各部分都可留空,但不能删掉{}
\makecvfooter
  {\hyperref{http://williamyao.com}{}{}{\dk}}% 左边部分
  {\faWineBottle}% 中间部分
  {\thepage}% 右边部分
%-------------------------------------------------------------------------------


%-------------------------------------------------------------------------------
% 正文内容部分
% 独立引用每个文件,或者直接书写正文
%-------------------------------------------------------------------------------
% 信件环境的开头部分
% 可以注释掉,不显示该部分
\makelettertitle\newline
\begin{cvletter}
{\hspace{2em}}无论你是偶然访问到这里,还是正在搜索你感兴趣的项目,我在这里都表示由衷的感谢。

{\hspace{2em}}Awesome-CV-Dukang项目,简称{\dk},是基于一款叫做\hyperref{https://github.com/posquit0/Awesome-CV}{}{}{Awesome-CV}的{\LaTeX}模板,进行中文化适应、额外提供许多新功能扩展,并面向入门使用者打包构建的快速开始项目。此篇文档旨在尽量以入门者的角度详细介绍{\dk}项目涵盖的所有内容,比如使用、配置和附加功能,希望通过阅读本文,能与诸君分享更多知识和收获。
\end{cvletter}

% 信件环境结尾部分
% 可以注释掉,不显示该部分
\makeletterclosing\newline
\clearpage

\cvsection{总体介绍}
Awesome-CV本身设计非常优秀,但由于语言习惯等差别,其对中文环境的支持和适配需要用户自行调整很多东西,同时其原作者的本意主要是聚焦在构建“简历”模板,鄙人以为,如此漂亮的设计,理应延伸至文章甚至书籍创作领域。在这个层面上,需要更加宏观的对项目中每个部件的细节进行调整,并引入适合文章或书籍创作的功能模块,而且相对于漂亮的设计和丰富的功能,{\LaTeX}过高的使用门槛却一直阻挡在追求严谨排版和高质量输出,却不是特别习惯看似古怪的语法、复杂难记的命令环境和众多宏包的朋友面前,“劝退”很多有志之士踏进{\LaTeX}的神奇世界。

尝试使用ctex宏包对Awesome-CV进行中文环境适配,对相应的部件和元素进行调整,对一些代码上的设计进行改进,并增加一些封装好的开箱即用的实用部件,是{\dk}项目的主要目的,将{\LaTeX}方面的一些知识表述清楚,将维护该项目学习到的一些技巧分享出去,也是我的初衷。

罗嗦这么多,还是赶快让我们进入主题吧。

\cvsubsection{主体结构}
{\dk}的结构大致上分为三个逻辑部分,第一部分是来自Awesome-CV项目的文档类,定义了所生成文件的最重要的内容,第二部分是本项目新增的宏包、扩展内容、资源文件夹以及编译控制文件,第三部分是提供给使用者作为快速开始模板的文件,当然还有一些项目维护所涉及的例如README、LICENSE、Git相关文件等。

下图列举了项目主体结构和主要的文件所在位置,其中来自本项目的所有文本类文件,都有详细的备注和说明信息,如果想要{\color{awesome}快速开始},不妨按照下图找到这些文件,阅读里面的源代码,寻找自己感兴趣的部分。

\dkresource{resource/r-arch}[0.9]

\begin{dkcomment}
  项目其他子文件夹和一些辅助性的文件,比如LOGO等没有在上图中体现。
\end{dkcomment}

\cvsubsection{快速体验与兼容性测试}

在深入介绍以前,有必要先体验一下编译输出的过程,同时也可以达到测试兼容性的目的。

\begin{dkcode}*{bash}[打开控制台依次执行]
# 进入项目根目录
cd latex-dukang

# 编译生成你的大作
make

# 编译生成本文档
make doc
\end{dkcode}

整个编译的过程,会输出大量的编译信息和告警内容,如果看到\dkbutton{Warning},先别着急,在{\LaTeX}的世界里,有众多的叫做“宏包”的贡献者在共同的支撑着你的创作,并不是所有的告警信息都意味着你“哪里做错了”,有些只是“善意的提醒”,比如{\dk}当前版本会有“未找到斜体字形”、“hbox overfull/underfull”等告警,虽然这类告警一定意味着“编译质量不怎么高”,但实际情形并不影响最终的输出结果,这些告警在后续的章节会有详细的说明。

回到上面的命令,如果执行顺利的话,第一个make执行完毕后,都会有类似下面的内容:

\vspace{1em}
\begin{center}
  \dkcodebox*{Latexmk: All targets (src/main.xdv src/main.pdf) are up-to-date}
\end{center}
\vspace{1em}

这说明main.tex文件已经成功被编译输出到了main.pdf,可以打开这个PDF文件看看实际的效果。第二个make是用来生成{\dk}文档(也就是本文)的,也跑一跑编译吧,确保你看到下面的提示信息。

\vspace{1em}
\begin{center}
  \dkcodebox*{Latexmk: All targets (src/dukang-doc.xdv src/dukang-doc.pdf) are up-to-date}
\end{center}
\vspace{1em}

如果,你碰到了光标停在一个问号后面,画面信息不再滚动的情形,那直接就是“错误”而不是“告警”了,这时候需要你向上查看具体哪里出了问题,或者输入大写的“X”并回车退出编译过程。这时候一般是没有main.pdf或dukang-doc.pdf输出出来的,请仔细检查输出日志中的错误提示,尝试着解决问题,并在\dkcodebox*{make cleanall}清理完所有临时文件之后再次尝试make。

\begin{dkcomment}[关于兼容性]
  {\hspace{2em}}{\dk}使用TexLive 2022套件在Ubuntu 20.04平台上开发、维护以及编译测试,使用其他版本套件或编译组件进行操作的话,目前还没有更加详细的测试结果,也衷心希望能够得到你的使用情况反馈,请随时邮件我。
\end{dkcomment}


\clearpage

\cvsection{结构设计}
由于{\dk}尽量使用非侵入式的方式与Awesome-CV进行集成,自然也沿用了其简洁方便的结构设计,除了总体编译文件、项目说明文件、项目图标以及文档类本体以外,其他的内容相关文件都放在\dkbutton{./src/}中。其中为了更方便的操作,部分文件使用了软链接\footnote{使用GNU Make以及软链接,对Windows系统和使用IDE的用户来说可能需要做一些适应性调整,详情请阅读“编译使用详解”章节的“本地环境准备”部分。}。

\begin{dkcomment}*[项目结构说明]
\dirtree{%
.1 \dk.
.2 \faFileCode~awesome-cv.cls\DTcomment{Awesome-CV的类文件}.
.2 \faFileImage~icon.png\DTcomment{项目图标}.
.2 \faFileCode~Makefile\DTcomment{控制编译命令的make文件}.
.2 \faMarkdown~README.md\DTcomment{项目说明文件}.
.2 \faFolder~src\DTcomment{内容相关文件}.
.3 \faLink~awesome-cv.cls\DTcomment{Awesome-CV类文件的软链接}.
.3 \faFileCode~ctex-fontset-custom.def\DTcomment{ctex的自定义字体文件}.
.3 \faFileCode~acv-dukang.cls\DTcomment{\dk~的主文件}.
.3 \faFileCode~dukang-doc.tex\DTcomment{本文档的主文件}.
.3 \faFolder~dukang-doc\DTcomment{存放本文档的章节内容文件}.
.3 \faFileCode~main.tex\DTcomment{从此文件入手开始创作}.
.3 \faFolder~main\DTcomment{存放main.tex引用的正文内容}.
.3 \faFolder~resource\DTcomment{存放图形、表格、外部pdf或tex等资源文件}.
.4 \faLink~ctex-fontset-custom.def\DTcomment{ctex自定义字体文件的软链接}.
.4 \faFileCode~Makefile\DTcomment{对应resource文件夹的make文件}.
.4 \faFileImage~logo.png\DTcomment{文档首页头信息中引用的图片文件}.
.4 \faFileCode~r-*.tex\DTcomment{所有以\dkbutton{r-*.tex}形式命名的文件都会支持联动编译控制}.
}
\end{dkcomment}

\begin{cventries}
\cventry[awesome-cv.cls][文档类]{
  \item {该文件属于Awesome-CV的文档类文件,{\dk}主要也是通过对该文件进行重定义来达到中文支持的目的。}
  \item {由于是无侵入方式,在没有发生重大变化的前提下,应该可以\dkbutton*{\hyperref{https://github.com/posquit0/Awesome-CV}{}{}{下载}}该文件的最新版本后直接覆盖来升级。}
}
\cventry[Makefile][编译控制]{
  \item {控制编译过程、简化项目使用操作的GNU Make文件。}
  \item {在Awesome-CV原有基础上,根据{\dk}所引入的部分新功能进行了增强。}
  \item {\color{awesome}不建议修改此文件,除非你知道自己在做什么。}
}
\cventry[src/awesome-cv.cls和src/resource/ctex-fontset-custom.def][软链接]{
  \item {这两个文件都是Linux平台的相对位置软链接。}
  \item {软链接使用\dkbutton{ln -rs}定义。}
}
\cventry[src/ctex-fontset-custom.def][字体定义]{
  \item {配合ctex宏包的\dkbutton{fontset=custom}选项来使用,可以修改这个文件中的字体设定。}
  \item {根据ctex的字体自定义设置,这里的\dkbutton{custom}对应ctex-fontset-\dkbutton{custom}.def,可以根据需要创建自己的自定义文件。}
}
\cventry[src/acv-dukang.cls][{\dk}主文件]{
  \item {该文件是{\dk}的文档类主文件,在文档中使用\dkbutton{\textbackslash documentclass}进行引用。}
  \item {所有修改以及增强的内容在源代码中都有详细的注释,想要更深入底层了解的话可以直接看该文件源码。}
  \item
  {之所以没有使用Doc和DocStrip构建Class文件,是因为我不会{\faMailchimp}或者只是单纯的不想把简单问题复杂化,毕竟{\dk}本身没有多少代码,内容也不复杂,况且已经配有详细的文档说明,没必要再生成一套“干货”,又不是通用宏包,不会提交给CTAN{\faMehBlank}}
}
\cventry[src/dukang-doc.tex和src/dukang-doc/][{\dk}说明文档]{
  \item {{\dk}说明文档(本文)主文件和存放章节内容文件的目录,同时也包含了用法示例。}
  \item {由于受Makefile编译控制,dukang-doc.tex的位置和文件名不要随便更改,但是dukang-doc文件夹中的内容文件都是由\dkbutton{\textbackslash input\{...\}}引入到主文档中,所以只要对应好文件位置,这个文件夹是可以自由修改的。}
  \item {\color{awesome}由于上述文件只用于生成{\dk}的说明文档(本文),与用户作品无关,所以不建议改动。}
}
\cventry[src/main.tex和src/main/][用户作品文件]{
  \item {用户作品的起始模板文件,可以直接从该文件开始进行创作。}
  \item {文件夹\dkbutton{./src/main/}与编译过程无关,仅用于存放起始模板文件所引用的内容文件,可以更改,甚至不用也行。}
  \item {如果作品的篇幅较长,还是建议使用\dkbutton{\textbackslash input\{...\}}的方式,可以保证主文档的内容干净整洁。}
}
\cventry[src/resource/][资源文件夹]{
  \item {用来统一存放图形、表格、外部pdf或tex等资源文件,并支持部分文件的联动编译。}
  \item {Makefile是与外层Makefile联动的编译控制文件,同时提供了部分针对资源目录中不同类型文件的编译命令。}
  \item {logo.png是封面上部显示的图片,在主文档的导言区中使用\dkbutton{\textbackslash photo\{...\}}进行定义。}
  \item {该文件夹被dukang-doc和用户作品共用。}
  \item {所有以\dkbutton{r-*.tex}形式命名的文件都会被联动编译和控制,本文档封面中Tikz图的源文件\dkbutton{r-arch.tex}就在这里。}
  \item {\color{awesome}不要更改该文件夹的位置,否则联动编译控制将找不到文件。}
}
\end{cventries}

\clearpage

\cvsection{编译使用详细说明}
\dk~使用\dkbutton{Makefile}将几个最常用的{\LaTeX}编译相关操作定义成了对应的控制台命令,用以对编译、输出、清理等过程进行简化处理,以及联动\dkbutton{Resource}文件夹下的\dkbutton{Makefile}文件。

{\today}版的Makefile\footnote{前半部分的指令都有注释进行说明,make的用法也不在本文的讨论范围之内,感兴趣的朋友可以自行学习。}文件源码如下:

\dkcodefile{makefile}{tango}{Makefile}{../Makefile}

\begin{cvhonors}*
  \cvhonor
  {make}
  {自动编译输出main.tex,等同于make main}
  {编译输出}
  \cvhonor
  {make main}
  {同上}
  {编译输出}
  \cvhonor
  {make doc}
  {自动编译输出\dk~文档,也就是本文档的输出命令}
  {编译输出}
  \cvhonor
  {make resource}
  {使用resource文件夹下的Makefile编译输出所有支持的资源文件,等同于进入resource文件夹下进行make或make all}
  {关联编译}
  \cvhonor
  {make all}
  {一次性编译main.tex和\dk~文档,等同于\dkbutton{make main \&\& make doc}}
  {编译输出}
\end{cvhonors}

\cvsubsection{make}
直接用于编译正文


\clearpage

\cvsection{组件使用详解}
组件共分为两类,一类是Awesome-CV原生定义的组件,用于结构化排版文档,另一类是{\dk}为Awesome-CV添加的自定义组件\footnote{所谓组件,其本质上只是对一些宏包功能的再封装,以达到方便使用的目的,真正要感谢的是那些宏包的作者。}可以在作品中使用。以上两种组件,如果你是一路看到这里,相信已经见过它们中的绝大多数了。

由于语言环境不同,许多Awesome-CV的原生组件在进行中文化的过程中,需要对一些细节进行处理,{\dk}尽量以无侵入(重定义)的方式在不碰类文件的情况下,对这些原生组件进行了修改和部分增强。同时,{\dk}增加了许多自定义组件,有的是方便引入资源的,有的是生成表格的,有些提供了代码高亮,有些生成小按钮风格,它们都有一个共同的能力,就是可以随着文档定义的Awesome-CV主色调自动适应配色。

下面分两个部分分别对Awesome-CV原生组件和{\dk}自定义组件进行详细介绍。

\cvsubsection{Awesome-CV原生组件}
Awesome-CV没有使用{\LaTeX}传统的chapter/section组织结构,而是完全自定义了自己的组件用于支撑文档结构。这样做对于简历类型的文档当然更加灵活,但是如果想要用来进行文章或书籍的创作,就有些不够用了。而且对于中文环境排版来说,我们有着更加复杂的习惯和要求,比如首行缩进、行间距、断句、对齐、字体等,这些是Awesome-CV原生环境装进中文时一定会遇到的问题,虽然一部分能够被ctex宏包自动修正,但由于没有chapter/section等结构,面对非标准化的自定义环境和命令,ctex宏包的强大能力也无处施展。因此,{\dk}在这方面着重下了一番功夫,对绝大部分Awesome-CV原生组件进行了调整,并修复了一些我感觉像bug的地方\footnote{其实应该也不算bug,只是修改完之后在使用方面会更方便灵活},比如某些组件不能紧挨着,某些组件调用顺序不对会搞乱行间距或编译错误等等。

下面我们逐一列出这些经过修改和增强之后的Awesome-CV原生组件。

\begin{cventries}
  \cventry
  [cvletter \& lettersection]
  [信件环境]
  {
    \item 主要用于首页信件环境的风格定义,也就是看似一封信的显示效果。
    \item 由于{\Verb{\lettersection}}的显示效果与{\Verb{\cvsection}}类似,且只用在信件环境中,扩展意义不大,所以尽管在Awesome-CV的示例文件中使用了,但{\dk}推荐直接在{\Verb{\cvletter}}中书写正文,或者包含{\Verb{\cvsection}}部分。
    \item 使用方法可参考\dkbutton{./src/dukang-doc/00-cover.tex}
  }
  \cventry
  [cvparagraph]
  [段落环境]
  {
    \item 这是段落的环境封装,在正文中用的不多,因为{\dk}已经定义好了正文段落的样式,这个cvparagraph环境可以不用,直接书写正文就好。
  }
  \cventry
  [cvsection \& cvsubsection]
  [章节命令]
  {
    \item 两者是Awesome-CV自定义的章节命令,由于没有chapter,所以cvsection就是一级标题,其实对于简历风格的模板来说,这样设计也是合情合理的,只不过用来创作文章或书籍的话,章节层级就要分明一些。
    \item {\dk}修改了这两个命令的样式,增加了决定是否启用首行缩进的开关,在文档的主文件(main.tex)中可以指定,具体可以参考“dukang导言区设定部分”的备注说明。
    \item {\dk}为这两个命令增加了PDF书签功能,并可以根据全局开关设置是否显示书签标题前的章节编号。
  }
  \cventry
  [cventries \& cventry]
  [组件环境]
  {
    \item cventries环境包含若干cventry命令,本部分就是使用该组件进行书写。
    \item {\dk}使用可变参数重新封装了Awesome-CV的原生cventry,现在该组件共包含五个部分,命令格式为:
    \begin{center}
      \dkbutton{\textbackslash cventry[第一行左侧][第一行右侧][第二行左侧][第二行右侧]\{正文\}}
    \end{center}
    \item 参数中除“正文”以外都可选,但需要按顺序给定,否则无法判断是第几个参数,例如只想显示第二行右侧的文字,需要将前面三个参数都标记出来并留空,有内容的参数后面的空参数可以不标记,例如:
    \begin{center}
      \dkbutton{\textbackslash cventry[][][][第二行右侧]\{正文文字\}}
      或者
      \dkbutton{\textbackslash cventry[][第一行右侧]\{正文文字\}}
    \end{center}
  }
  \cventry
  [cvhonors \& cvhonor]
  [组件环境]
  {
    \item cvhonors环境包含若干cvhonor命令,其本质是表格。
    \item {\dk}重新封装了原生组件,现在cvhonors环境有两个可选参数,格式为:
    \begin{center}
      \dkbutton{\textbackslash begin\{cvhonors\}<*>[LLR]...\textbackslash end\{cvhonors\}}
    \end{center}
    \item 其中第一个参数星号“*”为底色开关,带星号标识启用奇偶行底色,不带星号为无底色,切换底色会少许改变外边距,组件会自动进行调整。
    \item 第二个参数“[LLR]”的三个字母表示接下来所有cvhonor组件左中右三部分的对齐方式,L表示左对齐,C标识居中对齐,R表示右对齐,不给出该参数的话,默认为“[LLR]”。
    \item cvhonor命令分为三个必选参数,可以留空但不能不写,格式为:
    \begin{center}
      \dkbutton{\textbackslash cvhonor\{左\}\{中\}\{右\}}
    \end{center}
  }
  \cventry
  [cvskills \& cvskill]
  [组件环境]
  {
    \item cvskills环境包含若干cvskill命令,其本质也是2列的表格。
    \item 这个组件应用场景不是很多,所以就没有进行全面修改,依然沿用了Awesome-CV原生的使用方法:
    \begin{center}
      \dkbutton{\textbackslash begin\{cvskills\}...\textbackslash end\{cvskills\}}
    \end{center}
    在环境内部包含若干\dkbutton{\textbackslash cvskill\{左边部分\}\{右边部分\}}
  }
\end{cventries}

以上是{\dk}当前版本进行过优化的所有Awesome-CV文档组件,其实其原生组件也大致就这么多了,下一小节我们看一下来自{\dk}的自定义组件。

\cvsubsection{ACV-Dukang自定义组件}
在Awesome-CV原生组件的基础上,{\dk}将几个比较强大的宏包封装成了一些较为通用的自定义组件,一是更加灵活使用方便,二是丰富了ACV在“简历”场景以外的功能,这些组件适合应用在特别是科技类或IT类文章和书籍场景中。先看一下{\dk}自定义组件的全家福:

\begin{cvhonors}*
  \cvhonor
    {\Verb{\dkbutton}}
    {inline风格的代码小盒子,可选两种显示风格}
    {小按钮}
  \cvhonor
    {\Verb{\dkresource}}
    {引入外部pdf或图片的命令}
    {外部资源}
  \cvhonor
    {\Verb{\dkcode}}
    {可以对大段代码进行高亮显示的环境,可选两种显示风格}
    {代码高亮}
  \cvhonor
    {\Verb{\dkcodefile}}
    {读取外部文件并将内容进行代码高亮显示的环境,可选两种显示风格}
    {代码高亮}
  \cvhonor
    {\Verb{\dkcomment}}
    {可完全自定义显示的备注框,同样有两种显示风格可选}
    {备注框}
  \cvhonor
    {\Verb{\dkcodebox}}
    {另一种风格的inline代码小盒子,可切换是否显示命令提示符}
    {深色背景}
  \cvhonor
    {\Verb{\dirtree}}
    {显示文件夹树状结构的宏包命令,{\dk}进行了引入,无需再封装}
    {直接使用}
\end{cvhonors}

这些组件都有自己的用法和适合的使用环境,以及一些注意事项,接下来逐一介绍一下它们的细节。

\cvsubsubsection{dkbutton}
首先介绍这个小按钮,是因为在本文档中它用的最多,几乎随处可见,调用格式为:

|\dkbutton<*>[color]{text}|

包括星号在内共有两个可选参数,其中星号控制两种风格的切换:

\begin{center}
  \dkbutton{\textbackslash dkbutton\{这里有个小按钮\}} \& \dkbutton*{\textbackslash dkbutton*\{这里有圆形的小按钮\}}
\end{center}

而|[color]|可选参数默认为自动适应在主文档中指定的Awesome-CV的主配色,如果想要自选配色,可以使用这个参数进行指定,比如:

\begin{center}
  \dkbutton[blue]{\textbackslash dkbutton[blue]\{蓝色小按钮\}} \& \dkbutton*[green]{\textbackslash dkbutton*[green]\{绿色圆形小按钮\}}
\end{center}

在使用方面,除了尽量以“Inline”(也就是在正文的行内)的方式使用外几乎没有限制,当然如果想要一个超大的按钮也是可以的,只是推荐把它放在单独一个段落中,比如:

\begin{center}
  \vspace{1em}
  \dkbutton*{\parbox[c][5em]{20em}{\centering 一个巨大的圆形按钮}}
  \vspace{1em}
\end{center}

需要注意的是,同一行(或段落)中不要使用的太多,不然会影响自动断行,出现超宽或过窄的情况。

\cvsubsubsection{dkresource}
这个组件的作用是将外部资源(比如另一个PDF文档或PNG图片)引入当前位置居中显示,并可手动指定宽度自动缩放,调用格式为:

|\dkresource<[caption]>{file}<[width_factor]><[htbp!]>|

其中,各参数的含义为:

\begin{cvskills}*
  \cvskill
  {caption}
  {可选参数,资源下方需要显式的名称或标签}
  \cvskill
  {file}
  {必选参数,需要引入的资源,可以是图片、tex文件或任意$figure$环境支持的文件类型,只需给出路径及名称,无需加扩展名或后缀,相对路径的默认起点为\dkbutton{./src/}}
  \cvskill
  {width\_factor}
  {可选参数,相对于整个行宽的占比,取值为0-1,默认为0.9倍行宽}
  \cvskill
  {htbp!}
  {可选参数,用来控制资源在页面的浮动显示位置,其中$h$表示当前位置或代码所处的上下文位置,$t$表示Top(页面顶端),$b$表示Bottom(页面底部),$p$表示Page(单独成页),$!$表示在判定位置时忽略限制选项,默认为$htb$}
\end{cvskills}

\begin{dkcomment}
  \hspace{2em}$dkresource$实际上是基于{\LaTeX} $figure$环境的简单再封装,让人比较晕的是$htbp!$浮动体控制参数的意义,简单理解的话就是,在生成PDF文件的时候,{\LaTeX}编译器会根据浮动体控制参数来自动判断并确定浮动体在页面中的显示位置,这里的浮动体指的就是$dkresource$所引入的资源,排版位置的选取与参数里符号的顺序并无关系,编译器总是以$h \to t \to b \to p$的顺序来检查参数。

  \hspace{2em}|!|的作用是忽略限制,这里的默认限制总共有两条,超出该限制会强制将浮动体拖入下一页再判断,这两条分别是:

  \hspace{2em}个数:除$p$参数(单独成页)外,默认每页不超过3个浮动体,其中顶部$t$不超过2个,底部$b$不超过1个。

  \hspace{2em}空间占比:默认顶部$t$不超过页面高度的70\%,底部$b$不超过30\%
\end{dkcomment}

以上所有组件,基本可以满足所有日常创作,特别是科技类和IT类作品的创作需要,当然我会持续不断完善和扩展{\dk}的功能,后续组件会更加丰富。


\clearpage

\cvsection{更新与维护}


\clearpage

%-------------------------------------------------------------------------------


\end{document}
