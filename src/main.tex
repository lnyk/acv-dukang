%!TEX TS-program = xelatex
%!TEX encoding = UTF-8 Unicode

%-------------------------------------------------------------------------------
% CONFIGURATIONS
%-------------------------------------------------------------------------------
% A4 paper size by default, use 'letterpaper' for US letter
\documentclass[11pt, a4paper]{awesome-cv}
\usepackage[fontset=adobe]{ctex}

% Configure page margins with geometry
\geometry{left=1.4cm, top=.8cm, right=1.4cm, bottom=1.8cm, footskip=.5cm}

% Color for highlights
% Awesome Colors: awesome-emerald, awesome-skyblue, awesome-red, awesome-pink, awesome-orange
%                 awesome-nephritis, awesome-concrete, awesome-darknight
\colorlet{awesome}{awesome-red}
% Uncomment if you would like to specify your own color
% \definecolor{awesome}{HTML}{CA63A8}

% Colors for text
% Uncomment if you would like to specify your own color
% \definecolor{darktext}{HTML}{414141}
% \definecolor{text}{HTML}{333333}
% \definecolor{graytext}{HTML}{5D5D5D}
% \definecolor{lighttext}{HTML}{999999}
% \definecolor{sectiondivider}{HTML}{5D5D5D}

% Set false if you don't want to highlight section with awesome color
\setbool{acvSectionColorHighlight}{true}

% Set false if you don't want to indent the content
\setbool{dukangParIndent}{true}

% If you would like to change the social information separator from a pipe (|) to something else
\renewcommand{\acvHeaderSocialSep}{\quad\textbar\quad}


%-------------------------------------------------------------------------------
%	PERSONAL INFORMATION
%	Comment any of the lines below if they are not required
%-------------------------------------------------------------------------------
% Available options: circle|rectangle,edge/noedge,left/right
\photo[circle,noedge,left]{./src/profile}
\name{Dukang}{项目手册}
\position{%
  Kubenetes{\enskip\faAdn\enskip}Rook-Ceph{\enskip\cdotp\enskip}Kubesphere{\enskip\cdotp\enskip}开源{\enskip\cdotp\enskip}容器云
}
\address{42-8, Bangbae-ro 15-gil, Seocho-gu, Seoul, 00681, Rep. of KOREA}

\mobile{(+82) 10-9030-1843}
\email{posquit0.bj@gmail.com}
%\dateofbirth{January 1st, 1970}
\homepage{www.posquit0.com}
\github{posquit0}
\linkedin{posquit0}
% \gitlab{gitlab-id}
% \stackoverflow{SO-id}{SO-name}
% \twitter{@twit}
% \skype{skype-id}
% \reddit{reddit-id}
% \medium{madium-id}
% \kaggle{kaggle-id}
% \googlescholar{googlescholar-id}{name-to-display}
%% \firstname and \lastname will be used
% \googlescholar{googlescholar-id}{}
% \extrainfo{extra information}

\quote{``Be the change that you want to see in the world."}


%-------------------------------------------------------------------------------
%	LETTER INFORMATION
%	All of the below lines must be filled out
%-------------------------------------------------------------------------------
% The company being applied to
\recipient
  {Company Recruitment Team}
  {Google Inc.\\1600 Amphitheatre Parkway\\Mountain View, CA 94043}
% The date on the letter, default is the date of compilation
\letterdate{\today}
% The title of the letter
\lettertitle{Job Application for Software Engineer}
% How the letter is opened
\letteropening{Dear Mr./Ms./Dr. LastName,}
% How the letter is closed
\letterclosing{Sincerely,}
% Any enclosures with the letter
\letterenclosure[Attached]{Curriculum Vitae}


%-------------------------------------------------------------------------------
\begin{document}

% Print the header with above personal information
% Give optional argument to change alignment(C: center, L: left, R: right)
\makecvheader[R]

% Print the footer with 3 arguments(<left>, <center>, <right>)
% Leave any of these blank if they are not needed
\makecvfooter
  {\today}
  {Claud D. Park~~~·~~~Cover Letter}
  {\thepage}

% Print the title with above letter information
\makelettertitle

%-------------------------------------------------------------------------------
%	LETTER CONTENT
%-------------------------------------------------------------------------------
\begin{cvletter}

\lettersection{测试中文}
这里怎么说也是一堆中文嘛这里怎么说也是一堆中文嘛这里怎么说也是一堆中文嘛这里怎么说也是一堆中文嘛这里怎么说也是一堆中文嘛这里怎么说也是一堆中文嘛这里怎么说也是一堆中文嘛这里怎么说也是一堆中文嘛这里怎么说也是一堆中文嘛这里怎么说也是一堆中文嘛这里怎么说也是一堆中文嘛这里怎么说也是一堆中文嘛这里怎么说也是一堆中文嘛这里怎么说也是一堆中文嘛

\lettersection{Why Google?}
Suspendisse commodo, massa eu congue tincidunt, elit mauris pellentesque orci, cursus tempor odio nisl euismod augue. Aliquam adipiscing nibh ut odio sodales et pulvinar tortor laoreet. Mauris a accumsan ligula. Class aptent taciti sociosqu ad litora torquent per conubia nostra, per inceptos himenaeos. Suspendisse vulputate sem vehicula ipsum varius nec tempus dui dapibus. Phasellus et est urna, ut auctor erat. Sed tincidunt odio id odio aliquam mattis. Donec sapien nulla, feugiat eget adipiscing sit amet, lacinia ut dolor. Phasellus tincidunt, leo a fringilla consectetur, felis diam aliquam urna, vitae aliquet lectus orci nec velit. Vivamus dapibus varius blandit.

\lettersection{Why Me?}
Duis sit amet magna ante, at sodales diam. Aenean consectetur porta risus et sagittis. Ut interdum, enim varius pellentesque tincidunt, magna libero sodales tortor, ut fermentum nunc metus a ante. Vivamus odio leo, tincidunt eu luctus ut, sollicitudin sit amet metus. Nunc sed orci lectus. Ut sodales magna sed velit volutpat sit amet pulvinar diam venenatis.

\end{cvletter}


%-------------------------------------------------------------------------------
% Print the signature and enclosures with above letter information
\makeletterclosing


%-------------------------------------------------------------------------------
%	CV/RESUME CONTENT
%	Each section is imported separately, open each file in turn to modify content
%-------------------------------------------------------------------------------
\input{cv/education.tex}
\input{cv/skills.tex}
\input{cv/experience.tex}
\input{cv/extracurricular.tex}
\input{cv/honors.tex}
\input{cv/presentation.tex}
\input{cv/writing.tex}
\input{cv/committees.tex}


%-------------------------------------------------------------------------------
%	CV/RESUME CONTENT
%	Each section is imported separately, open each file in turn to modify content
%-------------------------------------------------------------------------------
\input{resume/summary.tex}
\input{resume/experience.tex}
\input{resume/honors.tex}
\input{resume/presentation.tex}
\input{resume/writing.tex}
\input{resume/committees.tex}
\input{resume/education.tex}
\input{resume/extracurricular.tex}


\end{document}
