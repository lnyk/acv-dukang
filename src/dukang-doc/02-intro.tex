\cvsection{结构设计总体介绍}
\begin{cvparagraph}
  \dk 沿用了Awesome-CV简洁方便的结构设计,除了控制总体编译的\dkbutton{Makefile}文件、\dkbutton{README.md}说明文件、\dkbutton{icon.png}项目图标以及\dkbutton{awesome-cv.cls}文档类本体以外,与文档写作有关的所有其他文档都放在\dkbutton{src/}文件夹中。其中,\dkbutton{awesome-cv.cls}文件本体以相对软链接\dkbutton{ln -rs}的形式链接到\dkbutton{src/awesome-cv.cls}。
\end{cvparagraph}

\begin{dkcomment}{目录结构}{\faFolder}
这里是测试
\dirtree{%
  .1 TRT/.
  .2 \faFile\space main.tex\DTcomment{主文档的定义文件}.
  .2 ctex-fontset-custom.def\DTcomment{字体定义文件}.
  .2 customize.tex\DTcomment{自定义命令及环境}.
  .2 data/\DTcomment{存放所有主文档内容}.
  .3 cover.tex\DTcomment{封面、摘要及关键字}.
  .3 denotation.tex\DTcomment{主要符号对照表}.
  .3 ack.tex\DTcomment{致谢内容}.
  .3 resume.tex\DTcomment{个人简历}.
  .3 chap[*].tex\DTcomment{章节内容}.
  .3 appendix[*].tex\DTcomment{附录}.
  .2 docs/\DTcomment{TRT的相关文档}.
  .3 trt.pdf\DTcomment{TRT使用说明(本文档)}.
  .3 dirtree.pdf\DTcomment{dirtree命令官方文档}.
  .3 fontawesome.pdf\DTcomment{FontAwesome官方文档}.
  .2 fonts/\DTcomment{TRT用到的字体文件}.
  .2 figures/\DTcomment{存放所有图表文件}.
  .2 ref/refs.bib\DTcomment{参考文献}.
}
\end{dkcomment}

\begin{cvskills}

%---------------------------------------------------------
  \cvskill
    {DevOps} % Category
    {AWS, Docker, Kubernetes, Rancher, Vagrant, Packer, Terraform, Jenkins, CircleCI} % Skills

%---------------------------------------------------------
  \cvskill
    {Back-end} % Category
    {Koa, Express, Django, REST API} % Skills

%---------------------------------------------------------
  \cvskill
    {Front-end} % Category
    {Hugo, Redux, React, HTML5, LESS, SASS} % Skills

%---------------------------------------------------------
  \cvskill
    {Programming} % Category
    {Node.js, Python, JAVA, OCaml, LaTeX} % Skills

%---------------------------------------------------------
  \cvskill
    {Languages} % Category
    {Korean, English, Japanese} % Skills

%---------------------------------------------------------
\end{cvskills}

\clearpage
