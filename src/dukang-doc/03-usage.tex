\cvsection{编译使用详细说明}
\dk~使用\dkbutton{Makefile}将几个最常用的{\LaTeX}编译相关操作定义成了对应的控制台命令,用以对编译、输出、清理等过程进行简化处理,以及联动\dkbutton{Resource}文件夹下的\dkbutton{Makefile}文件。

{\today}版的Makefile\footnote{前半部分的指令都有注释进行说明,make的用法也不在本文的讨论范围之内,感兴趣的朋友可以自行学习。}文件源码如下:

\dkcodefile{makefile}{tango}{Makefile}{../Makefile}

\begin{cvhonors}*
  \cvhonor
  {make}
  {自动编译输出main.tex,等同于make main}
  {编译输出}
  \cvhonor
  {make main}
  {同上}
  {编译输出}
  \cvhonor
  {make doc}
  {自动编译输出\dk~文档,也就是本文档的输出命令}
  {编译输出}
  \cvhonor
  {make resource}
  {使用resource文件夹下的Makefile编译输出所有支持的资源文件,等同于进入resource文件夹下进行make或make all}
  {关联编译}
  \cvhonor
  {make all}
  {一次性编译main.tex和\dk~文档,等同于\dkbutton{make main \&\& make doc}}
  {编译输出}
\end{cvhonors}

\cvsubsection{make}
直接用于编译正文


\clearpage
