%!TEX TS-program = xelatex
%!TEX encoding = UTF-8 Unicode

%-------------------------------------------------------------------------------
% 基本配置
%-------------------------------------------------------------------------------
% 默认为A4纸张,12pt字号
\documentclass[12pt, a4paper, final]{awesome-cv}

% 使用geometry定义纸张边距
\geometry{left=1.4cm, top=.8cm, right=1.4cm, bottom=1.8cm, footskip=.5cm}

% 高亮颜色配置
% Awesome Colors:
%   awesome-emerald, awesome-skyblue, awesome-red, awesome-pink, awesome-orange
%   awesome-nephritis, awesome-concrete, awesome-darknight
\colorlet{awesome}{awesome-red}
% 下面部分可启用自定义颜色
% \definecolor{awesome}{HTML}{CA63A8}

% Colors for text
% Uncomment if you would like to specify your own color
% \definecolor{darktext}{HTML}{414141}
% \definecolor{text}{HTML}{333333}
% \definecolor{graytext}{HTML}{5D5D5D}
% \definecolor{lighttext}{HTML}{999999}
% \definecolor{sectiondivider}{HTML}{5D5D5D}

% 全局开关变量定义
% 是否使用高亮颜色配置来突出section标题
\setbool{acvSectionColorHighlight}{true}

% Header中社交媒体行的分隔符
% 目前指定为管道符 |
\renewcommand{\acvHeaderSocialSep}{\quad\textbar\quad}


%-------------------------------------------------------------------------------
%	个人信息
%	不需要的部分可以注释掉
%-------------------------------------------------------------------------------
% 首页的图片Logo,默认为./src/resource/profile<.png>
% 可用选项为circle|rectangle,edge/noedge,left/right
\photo[circle,noedge,left]{./src/resource/profile}
\name{\LaTeX~Dukang}{手册}
\position{%
  \hyperref{https://github.com/posquit0/Awesome-CV}{}{}{Awesome-CV}{\enskip\faWineBottle\enskip}\dk
}
\address{基于\hyperref{https://github.com/posquit0/Awesome-CV}{}{}{Awesome-CV}进行中文化适应及无侵入增强的面向\LaTeX{}新人的快速开始项目}

%-------------------------------------------------------------------------------
% 以下部分至少有一条要保留
% \mobile{(+82) 10-9030-1843}
\email{me@williamyao.com}
% \dateofbirth{January 1st, 1970}
\homepage{WilliamYao.com}
\github{lnyk}
% \linkedin{posquit0}
% \gitlab{gitlab-id}
% \stackoverflow{SO-id}{SO-name}
% \twitter{@twit}
% \skype{skype-id}
% \reddit{reddit-id}
% \medium{madium-id}
% \kaggle{kaggle-id}
% \googlescholar{googlescholar-id}{name-to-display}
%% \firstname and \lastname will be used
% \googlescholar{googlescholar-id}{}
\extrainfo{{\faWineBottle}~衷心感谢\hyperref{https://github.com/posquit0}{}{}{Byungjin Park}等童鞋的开源奉献}

\quote{“执着而不计成本,不为索取只为陶醉”——~Carl Zeiss}


%-------------------------------------------------------------------------------
%	信件环境基本信息
%	所有内容必须不能缺少
%-------------------------------------------------------------------------------
% 收件方
\recipient
  {亲爱的\dk 使用者}
  {\hskip2em 本文档旨在用具体的示例全面介绍和展示\dk 宏包的使用、配置和附加功能,包括针对Awesome-CV模板进行的所有中文化处理和修改,并尽力以入门者的角度进行详细说明,希望通过阅读本文,能与诸君分享更多知识和收获。}
% 信件日期
\letterdate{\today}
% 信件标题
\lettertitle{关于快速开始}
% 信件称谓
\letteropening{如果迫不及待的想要使用\dk 进行创作,请直接按照下图查找并阅读项目文件吧\faKissWinkHeart}
% 信件结尾词
\letterclosing{怎么样,难道还意犹未尽吗?}
% 信件结尾词附言
\letterenclosure[亲爱的使用者]{继续发挥想像吧!}


%-------------------------------------------------------------------------------
% dukang导言区设定
%-------------------------------------------------------------------------------
% 引入dukang宏包
\usepackage{dukang}
%-------------------------------------------------------------------------------
% 打开该开关会启用某些正文内容的首行缩进
\setbool{dukangParIndent}{true}
% 是否为PDF书签添加标题前的标号
\setbool{dukangBookmarkLeadingNumber}{true}
%-------------------------------------------------------------------------------
% 定义编译之后的PDF相关属性
\hypersetup{%
  pdftitle={\LaTeX~Dukang~手册},
  pdfauthor={William Yao},
  pdfcreator={William Yao},
  pdfsubject={引用Awesome-CV模板,继承并加强},
  pdfkeywords={tex,latex,pdf,awesome,cv,resume,book,article,william,yao,dukang}
}
%-------------------------------------------------------------------------------


%-------------------------------------------------------------------------------
\begin{document}
%-------------------------------------------------------------------------------
% dukang文档区设定
%-------------------------------------------------------------------------------
% 首页的抬头
% 可选
% 可用对齐选项为C: 居中,L: 左对齐,R: 右对齐
\makecvheader[R]
%-------------------------------------------------------------------------------
% 每页的页脚定义,分为左中右三个部分,分别对应每个{}
% 各部分都可留空,但不能删掉{}
\makecvfooter
  {\hyperref{http://williamyao.com}{}{}{\LaTeX Dukang}}% 左边部分
  {\faWineBottle}% 中间部分
  {\thepage}% 右边部分


%-------------------------------------------------------------------------------
% 正文内容部分
% 独立引用每个文件,或者直接书写正文
%-------------------------------------------------------------------------------
% 信件环境的开头部分
% 可以注释掉,不显示该部分
\makelettertitle

%-------------------------------------------------------------------------------
%	信件环境正文
%-------------------------------------------------------------------------------
\begin{cvletter}

\lettersection{它们就在这些地方}
\dkresource[htb]{resource/r-arch}{0.9}{}

\end{cvletter}


%-------------------------------------------------------------------------------
% 信件环境结尾部分
% 可以注释掉,不显示该部分
% \makeletterclosing

% 信件环境结束后开新页
\clearpage

\cvsection{结构设计总体介绍}
由于\dk~尽量使用非侵入式的方式与Awesome-CV进行集成,自然也沿用了其简洁方便的结构设计,除了总体编译文件、项目说明文件、项目图标以及文档类本体以外,其他的内容相关文件都放在\dkbutton{src/}中。其中为了更方便的操作,部分文件使用了软链接\footnote{使用make以及软链接,对Windows平台和使用IDE的用户来说可能需要做一些适应性调整,这部分问题目前不在\dk~的代码设计范围之内,也许未来版本会考虑加入跨平台的设计要素。}。

\begin{dkcomment}{项目结构说明}{\faFolder}
\dirtree{%
.1 \dk.
.2 \faFileCode~awesome-cv.cls\DTcomment{Awesome-CV的类文件}.
.2 \faFileImage~icon.png\DTcomment{项目图标}.
.2 \faFileCode~Makefile\DTcomment{控制编译命令的make文件}.
.2 \faMarkdown~README.md\DTcomment{项目说明文件}.
.2 \faFolder~src\DTcomment{内容相关文件}.
.3 \faLink~awesome-cv.cls\DTcomment{Awesome-CV类文件的软链接}.
.3 \faFileCode~ctex-fontset-custom.def\DTcomment{ctex的自定义字体文件}.
.3 \faFileCode~dukang.sty\DTcomment{\dk~的主文件}.
.3 \faFileCode~dukang-doc.tex\DTcomment{本文档的主文件}.
.3 \faFileCode~main.tex\DTcomment{最简化的起始文件}.
.2 \faFolder~dukang-doc\DTcomment{存放本文档的章节内容文件}.
.2 \faFolder~resource\DTcomment{存放图形、表格、外部pdf或tex等资源文件}.
.3 \faLink~ctex-fontset-custom.def\DTcomment{ctex自定义字体文件的软链接}.
.3 \faFileCode~Makefile\DTcomment{对应resource文件夹的make文件}.
.3 \faFileImage~profile.png\DTcomment{文档首页头信息中引用的图片文件}.
.3 \faFileCode~r-*.png\DTcomment{所有以\dkbutton{r-*.tex}形式命名的文件都会支持自动编译和清理}.
}
\end{dkcomment}

\begin{cvskills}

%---------------------------------------------------------
  \cvskill
    {DevOps} % Category
    {AWS, Docker, Kubernetes, Rancher, Vagrant, Packer, Terraform, Jenkins, CircleCI} % Skills

%---------------------------------------------------------
  \cvskill
    {Back-end} % Category
    {Koa, Express, Django, REST API} % Skills

%---------------------------------------------------------
  \cvskill
    {Front-end} % Category
    {Hugo, Redux, React, HTML5, LESS, SASS} % Skills

%---------------------------------------------------------
  \cvskill
    {Programming} % Category
    {Node.js, Python, JAVA, OCaml, LaTeX} % Skills

%---------------------------------------------------------
  \cvskill
    {Languages} % Category
    {Korean, English, Japanese} % Skills

%---------------------------------------------------------
\end{cvskills}

\clearpage


\end{document}
