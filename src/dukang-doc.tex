%!TEX TS-program = xelatex
%!TEX encoding = UTF-8 Unicode


%-------------------------------------------------------------------------------
% 基本配置
%-------------------------------------------------------------------------------
% 以下设定需要保持在documentclass之前出现
% 传递table参数到xcolor包,用于tabular颜色支持
\PassOptionsToPackage{table}{xcolor}
%-------------------------------------------------------------------------------
% 默认为A4纸张,12pt字号
\documentclass[12pt, a4paper, final]{awesome-cv}

% 使用geometry定义纸张边距
\geometry{left=1.4cm, top=.8cm, right=1.4cm, bottom=1.8cm, footskip=.5cm}
%-------------------------------------------------------------------------------
% Aweosme-CV导言区设定部分
%-------------------------------------------------------------------------------
% 颜色配置,从下列套装中选择喜欢的配色
% Awesome Colors:
%   awesome-emerald, awesome-skyblue, awesome-red, awesome-pink, awesome-orange
%   awesome-nephritis, awesome-concrete, awesome-darknight
\colorlet{awesome}{awesome-red}
% 可启用自定义颜色
% \definecolor{awesome}{HTML}{CA63A8}
% 字体颜色,如果上面套装的字体颜色不喜欢,可以单独指定
% \definecolor{darktext}{HTML}{414141}
% \definecolor{text}{HTML}{333333}
% \definecolor{graytext}{HTML}{5D5D5D}
% \definecolor{lighttext}{HTML}{999999}
% \definecolor{sectiondivider}{HTML}{5D5D5D}
%-------------------------------------------------------------------------------
% 全局开关变量定义
% 是否使用高亮颜色配置来突出section标题
\setbool{acvSectionColorHighlight}{true}
%-------------------------------------------------------------------------------
% Header中社交媒体行的分隔符
% 目前指定为管道符 |
\renewcommand{\acvHeaderSocialSep}{\quad\textbar\quad}
%-------------------------------------------------------------------------------


%-------------------------------------------------------------------------------
% 文档内容信息部分
%-------------------------------------------------------------------------------
%	个人信息
%	不需要的部分可以注释掉
%-------------------------------------------------------------------------------
% 首页的图片Logo,默认为./src/resource/profile<.png>
% 可用选项为circle|rectangle,edge/noedge,left/right
\photo[circle,noedge,left]{./src/resource/profile}
\name{\LaTeX~Dukang}{手册}
\position{%
  \hyperref{https://github.com/posquit0/Awesome-CV}{}{}{Awesome-CV}{\enskip\faWineBottle\enskip}\dk
}
\address{基于\hyperref{https://github.com/posquit0/Awesome-CV}{}{}{Awesome-CV}进行中文化适应及无侵入增强的面向\LaTeX{}新人的快速开始项目}

%-------------------------------------------------------------------------------
% 以下部分至少有一条要保留
% \mobile{(+82) 10-9030-1843}
\email{me@williamyao.com}
% \dateofbirth{January 1st, 1970}
\homepage{WilliamYao.com}
\github{lnyk}
% \linkedin{posquit0}
% \gitlab{gitlab-id}
% \stackoverflow{SO-id}{SO-name}
% \twitter{@twit}
% \skype{skype-id}
% \reddit{reddit-id}
% \medium{madium-id}
% \kaggle{kaggle-id}
% \googlescholar{googlescholar-id}{name-to-display}
%% \firstname and \lastname will be used
% \googlescholar{googlescholar-id}{}
\extrainfo{{\faWineBottle}~衷心感谢\hyperref{https://github.com/posquit0}{}{}{Byungjin Park}等童鞋的开源奉献}

\quote{“执着而不计成本,不为索取只为陶醉”——~Carl Zeiss}
%-------------------------------------------------------------------------------
% 信件环境基本信息
% 所有内容都是必选,一个也不能少
%-------------------------------------------------------------------------------
% 收件方
\recipient
  {亲爱的\dk 使用者}
  {\hskip2em 本文档旨在用具体的示例全面介绍和展示\dk 宏包的使用、配置和附加功能,包括针对Awesome-CV模板进行的所有中文化处理和修改,并尽力以入门者的角度进行详细说明,希望通过阅读本文,能与诸君分享更多知识和收获。}
% 信件日期
\letterdate{\today}
% 信件标题
\lettertitle{关于快速开始}
% 信件称谓
\letteropening{如果迫不及待的想要使用\dk 进行创作,请直接按照下图查找并阅读项目文件吧\faKissWinkHeart}
% 信件结尾词
\letterclosing{怎么样,难道还意犹未尽吗?}
% 信件结尾词附言
\letterenclosure[亲爱的使用者]{继续发挥想像吧!}
%-------------------------------------------------------------------------------


%-------------------------------------------------------------------------------
% dukang导言区设定部分
%-------------------------------------------------------------------------------
% 引入dukang宏包
\usepackage{dukang}
%-------------------------------------------------------------------------------
% 打开该开关会启用某些正文内容的首行缩进
\setbool{dukangParIndent}{true}
% 是否为PDF书签添加标题前的标号
\setbool{dukangBookmarkLeadingNumber}{true}
%-------------------------------------------------------------------------------
% 定义编译之后的PDF相关属性
\hypersetup{%
  pdftitle={\LaTeX~Dukang~手册},
  pdfauthor={William Yao},
  pdfcreator={William Yao},
  pdfsubject={引用Awesome-CV模板,继承并加强},
  pdfkeywords={tex,latex,pdf,awesome,cv,resume,book,article,william,yao,dukang}
}
%-------------------------------------------------------------------------------


%-------------------------------------------------------------------------------
\begin{document}
%-------------------------------------------------------------------------------
% Awesome-CV文档区设定部分
%-------------------------------------------------------------------------------
% 文档开头的抬头部分,可以注释掉
% 可用对齐选项为C: 居中,L: 左对齐,R: 右对齐
\makecvheader[R]
%-------------------------------------------------------------------------------
% 每页的页脚定义,分为左中右三个部分,分别对应每个{}
% 各部分都可留空,但不能删掉{}
\makecvfooter
  {\hyperref{http://williamyao.com}{}{}{\LaTeX Dukang}}% 左边部分
  {\faWineBottle}% 中间部分
  {\thepage}% 右边部分
%-------------------------------------------------------------------------------


%-------------------------------------------------------------------------------
% 正文内容部分
% 独立引用每个文件,或者直接书写正文
%-------------------------------------------------------------------------------
% 信件环境的开头部分
% 可以注释掉,不显示该部分
\makelettertitle

%-------------------------------------------------------------------------------
%	信件环境正文
%-------------------------------------------------------------------------------
\begin{cvletter}

\lettersection{看一看总体结构}
\dkresource[htb]{resource/r-arch}{0.9}{}
\dk 沿用了Awesome-CV简洁方便的结构设计,除了负责总体编译命令的\dkboxr{Makefile}文件、$README.md$说明文件、$icon.png$项目图标以及$awesome-cv.cls$文档类本体以外,与文档写作有关的所有其他文档都放在$src/$文件夹中。其中,$awesome-cv.cls$文件本体以相对软链接\dkbox{ln -rs}的形式链接到$src/awesome-cv.cls$。

\begin{dkcomment}{测试dkcomment}{\faTree}
这里是dkcomment的正文,这里是dkcomment的正文,这里是dkcomment的正文,这里是dkcomment的正文,这里是dkcomment的正文,这里是dkcomment的正文,这里是dkcomment的正文,这里是dkcomment的正文,这里是dkcomment的正文,这里是dkcomment的正文,这里是dkcomment的正文,这里是dkcomment的正文,这里是dkcomment的正文,这里是dkcomment的正文,这里是dkcomment的正文,这里是dkcomment的正文,这里是dkcomment的正文,这里是dkcomment的正文,这里是dkcomment的正文,这里是dkcomment的正文。
\end{dkcomment}

\begin{dkcodeh}{python}{tango}{Test脚本}
import os
print("仍然没有成功解决CTeX环境下中英文之间的多余空格!")
print("Sigh... Need to solve the bug...")
pass
\end{dkcodeh}


\lettersection{Why Google?}
Suspendisse commodo, massa eu congue tincidunt, elit mauris pellentesque orci, cursus tempor odio nisl euismod augue. Aliquam adipiscing nibh ut odio sodales et pulvinar tortor laoreet. Mauris a accumsan ligula. Class aptent taciti sociosqu ad litora torquent per conubia nostra, per inceptos himenaeos. Suspendisse vulputate sem vehicula ipsum varius nec tempus dui dapibus. Phasellus et est urna, ut auctor erat. Sed tincidunt odio id odio aliquam mattis. Donec sapien nulla, feugiat eget adipiscing sit amet, lacinia ut dolor. Phasellus tincidunt, leo a fringilla consectetur, felis diam aliquam urna, vitae aliquet lectus orci nec velit. Vivamus dapibus varius blandit.

\lettersection{Why Me?}
Duis sit amet magna ante, at sodales diam. Aenean consectetur porta risus et sagittis. Ut interdum, enim varius pellentesque tincidunt, magna libero sodales tortor, ut fermentum nunc metus a ante. Vivamus odio leo, tincidunt eu luctus ut, sollicitudin sit amet metus. Nunc sed orci lectus. Ut sodales magna sed velit volutpat sit amet pulvinar diam venenatis.

\end{cvletter}


%-------------------------------------------------------------------------------
% 信件环境结尾部分
% 可以注释掉,不显示该部分
% \makeletterclosing

% 信件环境结束后开新页
\clearpage

\cvsection{结构设计总体介绍}
\begin{cvparagraph}
由于\dk~尽量使用非侵入式的方式与Awesome-CV进行集成,自然也沿用了其简洁方便的结构设计,除了控制总体编译的\dkbutton{Makefile}文件、\dkbutton{README.md}说明文件、\dkbutton{icon.png}项目图标以及\dkbutton{awesome-cv.cls}文档类本体以外,其他项目相关文件都放在\dkbutton{src/}中。其中为了更方便的操作,还使用了软链接\footnote{使用make以及软链接,对Windows平台和使用IDE的用户来说可能需要做一些适应性调整,这部分问题目前不在\dk~的代码设计范围之内,也许未来版本会考虑加入跨平台的设计要素。},比如:

\begin{dkcode}{bash}{tango}{使用软链接}
ln -rs ./awesome-cv.cls ./src/awesome-cv.cls
\end{dkcode}

\dkcodefile{bash}{tango}{使用软链接形式}{resource/test.py}
\end{cvparagraph}

\begin{dkcomment}{目录结构}{\faFolder}
这里是测试
\dirtree{%
  .1 TRT/.
  .2 \faFile\space main.tex\DTcomment{主文档的定义文件}.
  .2 ctex-fontset-custom.def\DTcomment{字体定义文件}.
  .2 customize.tex\DTcomment{自定义命令及环境}.
  .2 data/\DTcomment{存放所有主文档内容}.
  .3 cover.tex\DTcomment{封面、摘要及关键字}.
  .3 denotation.tex\DTcomment{主要符号对照表}.
  .3 ack.tex\DTcomment{致谢内容}.
  .3 resume.tex\DTcomment{个人简历}.
  .3 chap[*].tex\DTcomment{章节内容}.
  .3 appendix[*].tex\DTcomment{附录}.
  .2 docs/\DTcomment{TRT的相关文档}.
  .3 trt.pdf\DTcomment{TRT使用说明(本文档)}.
  .3 dirtree.pdf\DTcomment{dirtree命令官方文档}.
  .3 fontawesome.pdf\DTcomment{FontAwesome官方文档}.
  .2 fonts/\DTcomment{TRT用到的字体文件}.
  .2 figures/\DTcomment{存放所有图表文件}.
  .2 ref/refs.bib\DTcomment{参考文献}.
}
\end{dkcomment}

\begin{cvskills}

%---------------------------------------------------------
  \cvskill
    {DevOps} % Category
    {AWS, Docker, Kubernetes, Rancher, Vagrant, Packer, Terraform, Jenkins, CircleCI} % Skills

%---------------------------------------------------------
  \cvskill
    {Back-end} % Category
    {Koa, Express, Django, REST API} % Skills

%---------------------------------------------------------
  \cvskill
    {Front-end} % Category
    {Hugo, Redux, React, HTML5, LESS, SASS} % Skills

%---------------------------------------------------------
  \cvskill
    {Programming} % Category
    {Node.js, Python, JAVA, OCaml, LaTeX} % Skills

%---------------------------------------------------------
  \cvskill
    {Languages} % Category
    {Korean, English, Japanese} % Skills

%---------------------------------------------------------
\end{cvskills}

\clearpage

\cvsection{编译使用详解}
在本章节里,我会用很短的篇幅快速介绍一下与{\dk}运行相关的内容,包括本地环境准备、Makefile及编译、选项设定和使用流程,有些内容比较重要,请尽量顺序阅读。

\cvsubsection{本地环境准备}
在真正开始使用{\dk}进行创作之前,要先准备好你的本地编译环境。

\cvsubsubsection{关于兼容性}
我是使用TexLive 2022套件在Ubuntu 20.04平台上对{\dk}进行开发维护和编译测试的,如果你的是Windows操作系统,或其他版本TeX套件,可能会遇到问题,虽然{\dk}的Makefile已经针对Windows系统进行了设计,但对应Windows版本的TexLive套件,以及GNU Make和Python 3可能需要你花些功夫安装调试一番。目前还没有更加详细的测试结果,也衷心希望能够得到你的使用情况反馈。

\cvsubsubsection{关于字体}
由于{\dk}主要使用$ctex$宏包进行中文环境底层支持,而$ctex$默认提供的四套不同平台的字体方案都比较通用,没有个性化的设置,并且{\dk}在$ctex$的基础上增加了几个中文环境下的常用字体,因此需要采用custom方式进行设置,配置文件的相对位置在:\dkbutton{./src/ctex-fontset-custom.def}

完整列表如下:

\begin{cvhonors}*[LLL]
  \cvhonor
  {\heiti\color{black}字体名称}
  {\heiti\color{black}说明}
  {\heiti\color{black}短命令}
  \cvhonor
  {Adobe 仿宋 Std}
  {文档正文默认字体,大小为10pt}
  {\textbackslash fangsong}
  \cvhonor
  {Adobe 黑体 Std}
  {\heiti 一般用于章节名称}
  {\textbackslash heiti}
  \cvhonor
  {Adobe 楷体 Std}
  {\kaishu 脚注、代码块以及中文Mono/Sans环境会用到}
  {\textbackslash kaishu}
  \cvhonor
  {方正小标宋简体}
  {\fzxbs 主要用于章节标题等场景}
  {\textbackslash fzxbs}
  \cvhonor
  {Inconsolata}
  {\sffamily Footnote, Code block and English Mono/Sans Environmet}
  {\textbackslash sffamily}
  \cvhonor
  {Adobe 宋体 Std}
  {\songti 在文档中几乎不用,可以根据需要使用}
  {\textbackslash songti}
  \cvhonor
  {幼圆}
  {\youyuan 艺术创意等场景可能会用到}
  {\textbackslash youyuan}
  \cvhonor
  {隶书}
  {\lishu 诗词歌赋、文学环境比较合适}
  {\textbackslash lishu}
\end{cvhonors}

以上字体一部分是{\dk}基本依赖,一部分是本文档所用字体,建议全部安装,或者根据需要自行修改字体配置文件。短命令的具体用法可以参考本文档源码。

\cvsubsubsection{参考配置}
因为本地环境的构建可以有很多种选择,现在以我的配置为例:

\begin{enumerate}
  \item Ubuntu 20.04
  \item TexLive 2022
  \item GNU Make 4.2.1
  \item Python 3.8.10
  \item Python Pygments
\end{enumerate}

如果你用的是Linux操作系统,很有可能已经自带GNU Make和Python了,使用下面的命令查看一下版本号,确保版本别太旧:

\begin{dkcode}*{bash}
# 这里推荐使用Python3
pytho3 --version

# 查看GNU Make版本,越高越好
make --version
\end{dkcode}

之所以需要Python支持,是因为{\dk}部分组件使用了$minted$引擎调用$Pygments$库进行代码高亮渲染,如果缺少该库,这部分组件将不可用:

\begin{dkcode}*{bash}
# 查看Pygments版本
pygmentize -V

# 安装或升级Pygments库
pip install --upgrade Pygments
\end{dkcode}

\cvsubsection{Makefile及编译}
{\dk}使用Makefile\footnote{前半部分的指令都有注释进行说明,GNU Make的用法也不在本文的讨论范围之内,感兴趣的朋友可以自行学习。}将几个最常用的{\LaTeX}编译相关操作定义成了对应的控制台命令,用以对编译、输出、清理等过程进行简化处理,以及联动resource文件夹下的Makefile文件。

\dkcodefile*{../Makefile}{makefile}[Makefile]

\begin{dkcomment}
  以下所有make命令,都必须在项目根目录(就是你能看到\dkbutton{README.md}的地方)下执行。
\end{dkcomment}

\begin{cvhonors}*
  \cvhonor
  {make}
  {自动编译输出main.tex,等同于make main}
  {编译输出}
  \cvhonor
  {make main}
  {同上}
  {编译输出}
  \cvhonor
  {make doc}
  {自动编译输出\dk~文档,也就是本文档的输出命令}
  {编译输出}
  \cvhonor
  {make resource}
  {使用resource文件夹下的Makefile编译输出所有支持的资源文件,等同于进入resource文件夹下进行make或make all}
  {关联编译}
  \cvhonor
  {make all}
  {一次性编译main.tex和{\dk}文档,等同于make main \&\& make doc}
  {编译输出}
  \cvhonor
  {make clean}
  {自动清理所有临时文件和文件夹,包括主目录和所有子目录}
  {自动清理}
  \cvhonor
  {make cleanall}
  {在自动清理所有临时文件和文件夹的基础上,还会删除掉所有主目录下的PDF文件,并联动resource文件夹下所有生成的资源PDF}
  {自动清理}
\end{cvhonors}

\begin{dkcomment}
  make cleanall只会删除符合资源文件命名规则的PDF文件,其他文件不受影响。
\end{dkcomment}

\cvsubsection{文档选项及设定}
{\dk}提供了用于立刻开始创作的初始文件,位置是src/main.tex,该文件导言区\footnote{相信你知道什么是导言区,如果不是太清楚,还是抓紧自学一下吧,或者简单的认为,在大多数情况下,\\\textbackslash begin\{document\}前面的部分,就是导言区。}部分的设定与本文档源码src/dukang-doc.tex完全相同,只需要根据需要修改一些文档名称、作者姓名、首页需要哪些字段、是否需要首行缩进等信息和选项,就可以正式开始为你的大作添加正文了。

打开main.tex之后,首先会看到导言区一堆设定和备注信息,大多数情况下,阅读这些备注信息就足够掌握如何修改了,这里对所有选项进行深入说明。

\begin{cventries}
\cventry
[\Verb{\documentclass[12pt, a4paper, final]{awesome-cv-dukang}}]
[基本配置]
{
  \item 正式引入Awesome-CV文档类。
  \item 这里的字号设定(12pt)基本没什么卵用,因为几乎每个文档组件都定义了自己的字体风格。
}
\cventry
[\Verb{\geometry{left=1.4cm, top=.8cm, right=1.4cm, bottom=1.8cm, footskip=.5cm}}]
[基本配置]
{
  \item 使用geometry宏包定义纸张的页边距以及页脚距离
}
\cventry
[\Verb{\colorlet{awesome}{awesome-red}}]
[Awesome-CV]
[Awesome-CV的颜色设定]
[必选]
{
  \item 可以指定Awesome-CV预制好的几个配色集,包括awesome-emerald, awesome-skyblue, awesome-red, awesome-pink, awesome-orange, awesome-nephritis, awesome-concrete, awesome-darknight
  \item 也可以使用\Verb{\definecolor}指定自己喜欢的颜色,总共有awesome, darktext, text, graytext, lighttext, sectiondivider这几个颜色名称可供定义。
  \item {\color{awesome}{\dk}提供的所有增强组件都可以根据颜色设定进行风格自适应哦\faKissWinkHeart}
}
\cventry
[\Verb{\setbool{acvSectionColorHighlight}{true}}]
[Awesome-CV]
[指定是否使用配色凸显章节标题后紧跟的分割线]
[必选]
{
  \item 如果设定为true,责章节名称后面的长分割线会有颜色,否则为黑色。
}
\cventry
[\Verb{\renewcommand{\acvHeaderSocialSep}{\quad\textbar\quad}}]
[Awesome-CV]
[封面头部Logo右侧社交媒体帐号之间的分隔符定义]
[可选]
{
  \item 默认为管道符:\dkbutton{<空格>\textbar <空格>}
}
\cventry[个人信息部分][Awesome-CV][该部分用来定义一些个人信息或文档信息][部分可选]
{
  \item \Verb{\photo[rectangle,noedge,left]{./src/resource/dukang-logo}}用于定义首页的Logo图片,文件扩展名默认为.png,可用的裁剪选项为circle(圆形)和rectangle(正方形),可用的边框选项为edge(有边框)和noedge(无边框),可用的位置选项为left(靠左)和right(靠右)。
  \item 该部分除了\Verb{\name}和\Verb{\dukangPDFTitle}是必选的以外,其他设定不需要的均可以注释掉,首页中相应的部分会自适应。
  \item {\color{awesome}由于\Verb{\name}在Awesome-CV文档类中有多处引用,所以必须指定,不能删掉。}
  \item {\color{awesome}\Verb{\dukangPDFTitle}用来生成PDF文件书签中的主标题,所以必须指定,不能删掉。}
}
\cventry[社交媒体信息部分][Awesome-CV][该部分用来定义社交媒体帐号或联系方式][部分可选]
{
  \item 该部分有若干社交媒体选项,可以根据需要进行定义,不需要的可以注释掉。
  \item 上面定义的分隔符\Verb{\acvHeaderSocialSep}就是用来分割这些帐号的。
  \item {\color{awesome}至少要保留一条,否则编译出错!}
}
\cventry[cvletter环境基本信息][Awesome-CV][cvletter环境一般用于定义首页的内容][必选]
{
  \item 由于使用了ctex宏包,\Verb{\today}默认为大写中文日期格式。
  \item {\color{awesome}该部分的定义一个都不能少!}
}
\cventry[dukang导言区设定部分][\dk][此部分包含{\dk}及相关宏包提供的若干增强设定][必选]
{
  \item \Verb{\setbool{dukangParIndent}{true}},由于Awesome-CV大多数风格都使用了组件化(自定义命令或环境)来实现,没有使用chapter/section等标准结构,这直接导致了在引入ctex宏包进行中文化的时候,需要对每个组件进行单独的设定,比如首行缩进两字符对于有些组件要么不起作用,要么显示错乱,这里提供一个全局开关,会自动根据组件的具体情况有选择的开启首行缩进,以达到风格统一、显示美观的效果。
  \item \Verb{\setbool{dukangBookmarkLeadingNumber}{true}},Awesome-CV当前版本并不支持给输出的PDF文件按照文档结构自动添加书签(导航栏),{\dk}提供了这方面的支持,这个全局开关用来指定所添加的书签标题前,是否包含阿拉伯数字的章节编号。
  \item \Verb{\hypersetup},该部分用来为生成的PDF文件提供若干属性字段。
}
\cventry
[Awesome-CV文档区设定部分]
[Awesome-CV]
[该部分设定出现在文档区,也就是\Verb{\begin{document}和\end{document}之间。}]
[可选]
{
  \item \Verb{\makecvheader[R]},{\color{awesome}这不是页眉!Awesome-CV没有页眉。}这是首页包括Logo在内的抬头(Header)部分,可以注释掉,首页布局会自动从cvletter环境开始。可用选项用来控制对齐方向,L标识左对齐,C标识居中对齐,R标识右对齐。{\color{awesome}都要大写!}
  \item \Verb{\makecvfooter},这个是每页的页脚,分为左中右三个部分,每个部分都可以留空,但{\color{awesome}必须保留大括号\Verb{{}}}
}
\end{cventries}

以上是{\dk}当前版本设定部分的详细说明,需要格外注意的地方都有颜色高亮,右边颜色高亮的标签说明该选项来自哪个部分。并且,假如在修改的过程中不小心把main.tex搞乱了也没关系,可以随时打开dukang-doc.tex查看正确的配置,或者干脆把除了正文以外的所有内容复制回来,重新设定一下,就又可以开始创作了。这也是为什么我推荐用\Verb{\input{...}}把文档正文章节和main.tex主文件分开的原因。

\begin{dkcomment}*[温馨提示]
  无论何时,dukang-doc.tex都是你值得参考的示例文档,文档本身和其内容章节文件(特别是源代码)尽量涵盖到了{\dk}的全部功能,包括设定和功能模块等,随时可以回来查看。\faKissWinkHeart
\end{dkcomment}

\cvsubsection{使用流程}
\dk~的编译控制文件(Makefile)提供了适合下面几种场景的编译流程,相信总有一个适合你。首先,对于一般使用来说,需要做的步骤很简单:

\begin{center}
  \dkcodebox{修改main.tex}~\faArrowCircleRight~\dkcodebox{添加内容}~\faArrowCircleRight~\dkcodebox*{make}~\faCheckCircle
\end{center}

这样在src文件夹下就得到了main.pdf,同时在resource文件夹下,如果有符合命名规则的资源文件,也会被联动编译,并生成对应的PDF文件。{\dk}当前版本资源文件的命名规则是\dkbutton{r-*.tex},符合这个规则的.tex文件都会被自动编译和控制。

如果想要保留编译之后的PDF,同时把项目目录清理干净的话:

\begin{center}
  \dkcodebox{修改main.tex}~\faArrowCircleRight~\dkcodebox{添加内容}~\faArrowCircleRight~\dkcodebox*{make}~\faArrowCircleRight~\dkcodebox*{make clean}~\faCheckCircle
\end{center}

最后,如果希望只保留源代码\footnote{比如用于提交源代码,归档,或者你就是个纯粹的代码强迫症患者{\color{awesome}\faHeart}},把其他临时文件连同编译出来的东西一同干掉的话:

\begin{center}
  \dkcodebox{修改main.tex}~\faArrowCircleRight~\dkcodebox{添加内容}~\faArrowCircleRight~\dkcodebox*{make}~\faArrowCircleRight~\dkcodebox*{make cleanall}~\faCheckCircle
\end{center}

\begin{dkcomment}
  如果在编译过程中出现问题,强制退出编译过程之后想要再次编译,最好先执行\dkcodebox*{make cleanall}一遍,清理完所有临时文件之后再开始,否则编译很可能会出错无法继续下去。
\end{dkcomment}


\clearpage

%-------------------------------------------------------------------------------


\end{document}
