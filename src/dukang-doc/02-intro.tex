\cvsection{结构设计总体介绍}
由于\dk~尽量使用非侵入式的方式与Awesome-CV进行集成,自然也沿用了其简洁方便的结构设计,除了总体编译文件、项目说明文件、项目图标以及文档类本体以外,其他的内容相关文件都放在\dkbutton{src/}中。其中为了更方便的操作,部分文件使用了软链接\footnote{使用make以及软链接,对Windows平台和使用IDE的用户来说可能需要做一些适应性调整,这部分问题目前不在\dk~的代码设计范围之内,也许未来版本会考虑加入跨平台的设计要素。}。

\begin{dkcomment}*[项目结构说明]
\dirtree{%
.1 \dk.
.2 \faFileCode~awesome-cv.cls\DTcomment{Awesome-CV的类文件}.
.2 \faFileImage~icon.png\DTcomment{项目图标}.
.2 \faFileCode~Makefile\DTcomment{控制编译命令的make文件}.
.2 \faMarkdown~README.md\DTcomment{项目说明文件}.
.2 \faFolder~src\DTcomment{内容相关文件}.
.3 \faLink~awesome-cv.cls\DTcomment{Awesome-CV类文件的软链接}.
.3 \faFileCode~ctex-fontset-custom.def\DTcomment{ctex的自定义字体文件}.
.3 \faFileCode~dukang.sty\DTcomment{\dk~的主文件}.
.3 \faFileCode~dukang-doc.tex\DTcomment{本文档的主文件}.
.3 \faFileCode~main.tex\DTcomment{从此文件入手开始创作}.
.3 \faFolder~dukang-doc\DTcomment{存放本文档的章节内容文件}.
.3 \faFolder~resource\DTcomment{存放图形、表格、外部pdf或tex等资源文件}.
.4 \faLink~ctex-fontset-custom.def\DTcomment{ctex自定义字体文件的软链接}.
.4 \faFileCode~Makefile\DTcomment{对应resource文件夹的make文件}.
.4 \faFileImage~profile.png\DTcomment{文档首页头信息中引用的图片文件}.
.4 \faFileCode~r-*.tex\DTcomment{所有以\dkbutton{r-*.tex}形式命名的文件都会支持联动编译控制}.
.3 \faFolder~tex\DTcomment{存放被\dkbutton{main.tex}引用的内容}.
}
\end{dkcomment}

先简要说明一下部分文件和位置,详细内容会在后续章节中进行说明。

\begin{cventries}
\cventry
  {awesome-cv.cls}
  {文档类}
  {}
  {}
  {
    \begin{cvitems}
      \item {该文件属于Awesome-CV的文档类文件,\dk~主要也是通过对该文件进行重定义来达到中文化增强的目的。}
      \item {由于是无侵入增强,在没有发生重大变化的前提下,应该可以\dkbuttonr{\hyperref{https://github.com/posquit0/Awesome-CV}{}{}{下载}}该文件的最新版本后直接覆盖来升级。}
    \end{cvitems}
  }
\cventry
  {Makefile}
  {编译控制}
  {}
  {}
  {
    \begin{cvitems}
      \item {控制编译过程、简化项目使用操作的make文件。}
      \item {在Awesome-CV原有基础上,根据\dk~所引入的部分新功能进行了增强。}
      \item {\color{awesome}不建议修改此文件,除非你知道自己在做什么。}
    \end{cvitems}
  }
\cventry
  {src/awesome-cv.cls和src/resource/ctex-fontset-custom.def}
  {软链接}
  {}
  {}
  {
    \begin{cvitems}
      \item {这两个文件都是Linux平台的相对位置软链接,所以对应的文件位置不能改变。}
      \item {软链接使用\dkbutton{ln -rs}定义。}
      \item {在Windows平台上可能会有快捷方式方面的兼容性问题。}
    \end{cvitems}
  }
\cventry
  {src/ctex-fontset-custom.def}
  {字体定义}
  {}
  {}
  {
    \begin{cvitems}
      \item {配合ctex的\dkbutton{fontset=custom}设定来使用,可以修改这个文件中的字体设定。}
      \item {根据ctex的字体自定义设置,这里的\dkbutton{custom}对应ctex-fontset-\dkbutton{custom}.def,可以根据需要创建自己的自定义文件。}
    \end{cvitems}
  }
\cventry
  {src/dukang.sty}
  {\dk~主文件}
  {}
  {}
  {
    \begin{cvitems}
      \item {该文件是\dk~的主文件,在文档的导言区使用\dkbutton{\textbackslash usepackage\{dukang\}}引入。}
      \item {所有修改以及增强的内容在源代码中都有详细的注释,想要用更直接的方式查看用法的朋友可以直接看该文件。}
    \end{cvitems}
  }
\cventry
  {src/dukang-doc.tex和src/dukang-doc/}
  {\dk~说明文档}
  {}
  {}
  {
    \begin{cvitems}
      \item {\dk~说明文档(本文)主文件和存放章节内容文件的目录,同时也尽量包含了一些\dk~的基本用法示例。}
      \item {由于受Makefile编译控制,dukang-doc.tex的位置和文件名不要随便更改,但是dukang-doc文件夹中的内容文件都是由\dkbutton{\textbackslash input\{...\}}引入到主文档中,所以只要对应好文件位置,这个文件夹是可以自由修改的。}
      \item {\color{awesome}但是,由于这两个位置只用于生成\dk~的说明文档(本文),与用户作品无关,所以不建议改动。}
    \end{cvitems}
  }
\cventry
  {src/main.tex和src/tex/}
  {用户作品文件}
  {}
  {}
  {
    \begin{cvitems}
      \item {用户作品的起始模板文件,可以直接从该文件开始进行创作。}
      \item {文件夹\dkbutton{tex/}与编译过程无关,仅用于存放起始模板文件所引用的内容文件,可以更改,甚至不用也行。}
      \item {如果作品的篇幅较长,还是建议使用\dkbutton{\textbackslash input\{...\}}的方式,可以保证主文档的内容干净整洁。}
    \end{cvitems}
  }
\cventry
  {src/resource/}
  {资源文件夹}
  {}
  {}
  {
    \begin{cvitems}
      \item {用来统一存放图形、表格、外部pdf或tex等资源文件,并支持部分文件的联动编译。}
      \item {Makefile是与外层Makefile联动的编译控制文件,同时提供了部分针对资源目录中不同类型文件的编译命令。}
      \item {profile.png是封面上部显示的图片,在主文档的导言区中使用\dkbutton{\textbackslash photo\{...\}}进行定义。}
      \item {该文件夹被dukang-doc和用户作品共用。}
      \item {所有以\dkbutton{r-*.tex}形式命名的文件都会被联动编译和控制,本文档封面中\dkbutton{tikz}图的源文件\dkbutton{r-arch.tex}就在这里。}
      \item {\color{awesome}不要更改该文件夹的位置,否则联动编译控制将找不到文件。}
    \end{cvitems}
  }
\end{cventries}

\clearpage
