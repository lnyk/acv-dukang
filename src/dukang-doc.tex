%!TEX TS-program = xelatex
%!TEX encoding = UTF-8 Unicode


%-------------------------------------------------------------------------------
% 基本配置
%-------------------------------------------------------------------------------
% 以下设定需要保持在documentclass之前出现
% 传递table参数到xcolor包,用于tabular颜色支持
\PassOptionsToPackage{table}{xcolor}
\PassOptionsToPackage{CJKbookmarks}{hyperref}
%-------------------------------------------------------------------------------
% 默认为A4纸张,12pt字号
\documentclass[12pt, a4paper, final]{awesome-cv}

% 使用geometry定义纸张边距
\geometry{left=1.4cm, top=.8cm, right=1.4cm, bottom=1.8cm, footskip=.5cm}
%-------------------------------------------------------------------------------
% Aweosme-CV导言区设定部分
%-------------------------------------------------------------------------------
% 颜色配置,从下列套装中选择喜欢的配色
% Awesome Colors:
%   awesome-emerald, awesome-skyblue, awesome-red, awesome-pink, awesome-orange
%   awesome-nephritis, awesome-concrete, awesome-darknight
\colorlet{awesome}{awesome-red}
% 可启用自定义颜色
% \definecolor{awesome}{HTML}{CA63A8}
% 字体颜色,如果上面套装的字体颜色不喜欢,可以单独指定
% \definecolor{darktext}{HTML}{414141}
% \definecolor{text}{HTML}{333333}
% \definecolor{graytext}{HTML}{5D5D5D}
% \definecolor{lighttext}{HTML}{999999}
% \definecolor{sectiondivider}{HTML}{5D5D5D}
%-------------------------------------------------------------------------------
% 全局开关变量定义
% 是否使用高亮颜色配置来突出section标题
\setbool{acvSectionColorHighlight}{true}
%-------------------------------------------------------------------------------
% Header中社交媒体行的分隔符
% 目前指定为管道符 |
\renewcommand{\acvHeaderSocialSep}{\quad\textbar\quad}
%-------------------------------------------------------------------------------


%-------------------------------------------------------------------------------
% 文档内容信息部分
%-------------------------------------------------------------------------------
%	个人信息
%	不需要的部分可以注释掉
%-------------------------------------------------------------------------------
% 首页的图片Logo
% 可用选项为circle|rectangle,edge/noedge,left/right
\photo[rectangle,noedge,left]{./src/resource/dukang-logo}
\name{\LaTeX~Dukang}{手册}
\position{%
  \hyperref{https://github.com/posquit0/Awesome-CV}{}{}{Awesome-CV}{\enskip\faWineBottle\enskip}\dk
}
\address{基于\hyperref{https://github.com/posquit0/Awesome-CV}{}{}{Awesome-CV}进行中文化适应及无侵入增强的面向\LaTeX{}新人的快速开始项目}

%-------------------------------------------------------------------------------
% 以下部分至少有一条要保留
% \mobile{(+82) 10-9030-1843}
\email{me@williamyao.com}
% \dateofbirth{January 1st, 1970}
\homepage{WilliamYao.com}
\github{lnyk}
% \linkedin{posquit0}
% \gitlab{gitlab-id}
% \stackoverflow{SO-id}{SO-name}
% \twitter{@twit}
% \skype{skype-id}
% \reddit{reddit-id}
% \medium{madium-id}
% \kaggle{kaggle-id}
% \googlescholar{googlescholar-id}{name-to-display}
%% \firstname and \lastname will be used
% \googlescholar{googlescholar-id}{}
\extrainfo{{\faWineBottle}~衷心感谢\hyperref{https://github.com/posquit0}{}{}{Byungjin Park}等童鞋的开源奉献}

\quote{“执着而不计成本,不为索取只为陶醉”——~Carl Zeiss}
%-------------------------------------------------------------------------------
% 信件环境基本信息
% 所有内容都是必选,一个也不能少
%-------------------------------------------------------------------------------
% 收件方
\recipient
  {偶然到访或是兴趣所致的你}
  {{\LaTeX}, awesome-cv, {\dk}, template\\github, opensource}
% 信件日期
\letterdate{\today}
% 信件标题
\lettertitle{{\dk}}
% 信件称谓
\letteropening{亲爱的朋友:}
% 信件结尾词
\letterclosing{{\dk}项目Logo和名称的含义,相信你懂的,生活不易,何以解忧?\faKissWinkHeart}
% 信件结尾词附言
\letterenclosure[最后]{对你的关注,再次表示由衷的感谢!}
%-------------------------------------------------------------------------------


%-------------------------------------------------------------------------------
% dukang导言区设定部分
%-------------------------------------------------------------------------------
% 引入dukang宏包
\usepackage{dukang}
%-------------------------------------------------------------------------------
% 打开该开关会启用某些正文内容的首行缩进
\setbool{dukangParIndent}{true}
% 是否为PDF书签添加标题前的标号
\setbool{dukangBookmarkLeadingNumber}{true}
%-------------------------------------------------------------------------------
% 定义编译之后的PDF相关属性
\hypersetup{%
  pdftitle={\LaTeX~Dukang~手册},
  pdfauthor={William Yao},
  pdfcreator={William Yao},
  pdfsubject={引用Awesome-CV模板,继承并加强},
  pdfkeywords={tex,latex,pdf,awesome,cv,resume,book,article,william,yao,dukang}
}
%-------------------------------------------------------------------------------


%-------------------------------------------------------------------------------
\begin{document}
%-------------------------------------------------------------------------------
% Awesome-CV文档区设定部分
%-------------------------------------------------------------------------------
% 文档开头的抬头部分,可以注释掉
% 可用对齐选项为C: 居中,L: 左对齐,R: 右对齐
\makecvheader[R]
%-------------------------------------------------------------------------------
% 每页的页脚定义,分为左中右三个部分,分别对应每个{}
% 各部分都可留空,但不能删掉{}
\makecvfooter
  {\hyperref{http://williamyao.com}{}{}{\LaTeX Dukang}}% 左边部分
  {\faWineBottle}% 中间部分
  {\thepage}% 右边部分
%-------------------------------------------------------------------------------


%-------------------------------------------------------------------------------
% 正文内容部分
% 独立引用每个文件,或者直接书写正文
%-------------------------------------------------------------------------------
% 信件环境的开头部分
% 可以注释掉,不显示该部分
\makelettertitle

\begin{cvletter}
% \lettersection{它们就在这些地方}
{\hspace{2em}}无论你是偶然访问到这里,还是正在搜索你感兴趣的项目,我在这里都表示由衷的感谢。

{\hspace{2em}}{\dk}项目是基于一款叫做\hyperref{https://github.com/posquit0/Awesome-CV}{}{}{Awesome-CV}的{\LaTeX}模板,进行中文化适应、额外提供许多新功能扩展,并面向入门使用者打包构建的快速开始项目。此篇文档旨在尽量以入门者的角度详细介绍{\dk}项目涵盖的所有内容,比如使用、配置和附加功能,希望通过阅读本文,能与诸君分享更多知识和收获。
\end{cvletter}

% 信件环境结尾部分
% 可以注释掉,不显示该部分
\makeletterclosing

\clearpage

\cvsection{第一章节}
正文


\clearpage % 分页符

\cvsection{结构设计}
由于{\dk}尽量使用非侵入式的方式与Awesome-CV进行集成,自然也沿用了其简洁方便的结构设计,除了总体编译文件、项目说明文件、项目图标以及文档类本体以外,其他的内容相关文件都放在\dkbutton{./src/}中。其中为了更方便的操作,部分文件使用了软链接\footnote{使用GNU Make以及软链接,对Windows系统和使用IDE的用户来说可能需要做一些适应性调整,详情请阅读“编译使用详解”章节的“本地环境准备”部分。}。

\begin{dkcomment}*[项目结构说明]
\dirtree{%
.1 \dk.
.2 \faFileCode~awesome-cv.cls\DTcomment{Awesome-CV的类文件}.
.2 \faFileImage~icon.png\DTcomment{项目图标}.
.2 \faFileCode~Makefile\DTcomment{控制编译命令的make文件}.
.2 \faMarkdown~README.md\DTcomment{项目说明文件}.
.2 \faFolder~src\DTcomment{内容相关文件}.
.3 \faLink~awesome-cv.cls\DTcomment{Awesome-CV类文件的软链接}.
.3 \faFileCode~ctex-fontset-custom.def\DTcomment{ctex的自定义字体文件}.
.3 \faFileCode~acv-dukang.cls\DTcomment{\dk~的主文件}.
.3 \faFileCode~dukang-doc.tex\DTcomment{本文档的主文件}.
.3 \faFolder~dukang-doc\DTcomment{存放本文档的章节内容文件}.
.3 \faFileCode~main.tex\DTcomment{从此文件入手开始创作}.
.3 \faFolder~main\DTcomment{存放main.tex引用的正文内容}.
.3 \faFolder~resource\DTcomment{存放图形、表格、外部pdf或tex等资源文件}.
.4 \faLink~ctex-fontset-custom.def\DTcomment{ctex自定义字体文件的软链接}.
.4 \faFileCode~Makefile\DTcomment{对应resource文件夹的make文件}.
.4 \faFileImage~logo.png\DTcomment{文档首页头信息中引用的图片文件}.
.4 \faFileCode~r-*.tex\DTcomment{所有以\dkbutton{r-*.tex}形式命名的文件都会支持联动编译控制}.
}
\end{dkcomment}

\begin{cventries}
\cventry[awesome-cv.cls][文档类]{
  \item {该文件属于Awesome-CV的文档类文件,{\dk}主要也是通过对该文件进行重定义来达到中文支持的目的。}
  \item {由于是无侵入方式,在没有发生重大变化的前提下,应该可以\dkbutton*{\hyperref{https://github.com/posquit0/Awesome-CV}{}{}{下载}}该文件的最新版本后直接覆盖来升级。}
}
\cventry[Makefile][编译控制]{
  \item {控制编译过程、简化项目使用操作的GNU Make文件。}
  \item {在Awesome-CV原有基础上,根据{\dk}所引入的部分新功能进行了增强。}
  \item {\color{awesome}不建议修改此文件,除非你知道自己在做什么。}
}
\cventry[src/awesome-cv.cls和src/resource/ctex-fontset-custom.def][软链接]{
  \item {这两个文件都是Linux平台的相对位置软链接。}
  \item {软链接使用\dkbutton{ln -rs}定义。}
}
\cventry[src/ctex-fontset-custom.def][字体定义]{
  \item {配合ctex宏包的\dkbutton{fontset=custom}选项来使用,可以修改这个文件中的字体设定。}
  \item {根据ctex的字体自定义设置,这里的\dkbutton{custom}对应ctex-fontset-\dkbutton{custom}.def,可以根据需要创建自己的自定义文件。}
}
\cventry[src/acv-dukang.cls][{\dk}主文件]{
  \item {该文件是{\dk}的文档类主文件,在文档中使用\dkbutton{\textbackslash documentclass}进行引用。}
  \item {所有修改以及增强的内容在源代码中都有详细的注释,想要更深入底层了解的话可以直接看该文件源码。}
  \item
  {之所以没有使用Doc和DocStrip构建Class文件,是因为我不会{\faMailchimp}或者只是单纯的不想把简单问题复杂化,毕竟{\dk}本身没有多少代码,内容也不复杂,况且已经配有详细的文档说明,没必要再生成一套“干货”,又不是通用宏包,不会提交给CTAN{\faMehBlank}}
}
\cventry[src/dukang-doc.tex和src/dukang-doc/][{\dk}说明文档]{
  \item {{\dk}说明文档(本文)主文件和存放章节内容文件的目录,同时也包含了用法示例。}
  \item {由于受Makefile编译控制,dukang-doc.tex的位置和文件名不要随便更改,但是dukang-doc文件夹中的内容文件都是由\dkbutton{\textbackslash input\{...\}}引入到主文档中,所以只要对应好文件位置,这个文件夹是可以自由修改的。}
  \item {\color{awesome}由于上述文件只用于生成{\dk}的说明文档(本文),与用户作品无关,所以不建议改动。}
}
\cventry[src/main.tex和src/main/][用户作品文件]{
  \item {用户作品的起始模板文件,可以直接从该文件开始进行创作。}
  \item {文件夹\dkbutton{./src/main/}与编译过程无关,仅用于存放起始模板文件所引用的内容文件,可以更改,甚至不用也行。}
  \item {如果作品的篇幅较长,还是建议使用\dkbutton{\textbackslash input\{...\}}的方式,可以保证主文档的内容干净整洁。}
}
\cventry[src/resource/][资源文件夹]{
  \item {用来统一存放图形、表格、外部pdf或tex等资源文件,并支持部分文件的联动编译。}
  \item {Makefile是与外层Makefile联动的编译控制文件,同时提供了部分针对资源目录中不同类型文件的编译命令。}
  \item {logo.png是封面上部显示的图片,在主文档的导言区中使用\dkbutton{\textbackslash photo\{...\}}进行定义。}
  \item {该文件夹被dukang-doc和用户作品共用。}
  \item {所有以\dkbutton{r-*.tex}形式命名的文件都会被联动编译和控制,本文档封面中Tikz图的源文件\dkbutton{r-arch.tex}就在这里。}
  \item {\color{awesome}不要更改该文件夹的位置,否则联动编译控制将找不到文件。}
}
\end{cventries}

\clearpage

\cvsection{编译使用详解}
在本章节里,我会用很短的篇幅快速介绍一下与{\dk}运行相关的内容,包括本地环境准备、Makefile及编译、选项设定和使用流程,有些内容比较重要,请尽量顺序阅读。

\cvsubsection{本地环境准备}
在真正开始使用{\dk}进行创作之前,要先准备好你的本地编译环境。

\cvsubsubsection{关于兼容性}
我是使用TexLive 2022套件在Ubuntu 20.04平台上对{\dk}进行开发维护和编译测试的,如果你的是Windows操作系统,或其他版本TeX套件,可能会遇到问题,虽然{\dk}的Makefile已经针对Windows系统进行了设计,但对应Windows版本的TexLive套件,以及GNU Make和Python 3可能需要你花些功夫安装调试一番。目前还没有更加详细的测试结果,也衷心希望能够得到你的使用情况反馈。

\cvsubsubsection{关于字体}
由于{\dk}主要使用$ctex$宏包进行中文环境底层支持,而$ctex$默认提供的四套不同平台的字体方案都比较通用,没有个性化的设置,并且{\dk}在$ctex$的基础上增加了几个中文环境下的常用字体,因此需要采用custom方式进行设置,配置文件的相对位置在:\dkbutton{./src/ctex-fontset-custom.def}

完整列表如下:

\begin{cvhonors}*[LLL]
  \cvhonor
  {\heiti\color{black}字体名称}
  {\heiti\color{black}说明}
  {\heiti\color{black}短命令}
  \cvhonor
  {Adobe 仿宋 Std}
  {文档正文默认字体,大小为10pt}
  {\textbackslash fangsong}
  \cvhonor
  {Adobe 黑体 Std}
  {\heiti 一般用于章节名称}
  {\textbackslash heiti}
  \cvhonor
  {Adobe 楷体 Std}
  {\kaishu 脚注、代码块以及中文Mono/Sans环境会用到}
  {\textbackslash kaishu}
  \cvhonor
  {方正小标宋简体}
  {\fzxbs 主要用于章节标题等场景}
  {\textbackslash fzxbs}
  \cvhonor
  {Inconsolata}
  {\sffamily Footnote, Code block and English Mono/Sans Environmet}
  {\textbackslash sffamily}
  \cvhonor
  {Adobe 宋体 Std}
  {\songti 在文档中几乎不用,可以根据需要使用}
  {\textbackslash songti}
  \cvhonor
  {幼圆}
  {\youyuan 艺术创意等场景可能会用到}
  {\textbackslash youyuan}
  \cvhonor
  {隶书}
  {\lishu 诗词歌赋、文学环境比较合适}
  {\textbackslash lishu}
\end{cvhonors}

以上字体一部分是{\dk}基本依赖,一部分是本文档所用字体,建议全部安装,或者根据需要自行修改字体配置文件。短命令的具体用法可以参考本文档源码。

\cvsubsubsection{参考配置}
因为本地环境的构建可以有很多种选择,现在以我的配置为例:

\begin{enumerate}
  \item Ubuntu 20.04
  \item TexLive 2022
  \item GNU Make 4.2.1
  \item Python 3.8.10
  \item Python Pygments
\end{enumerate}

如果你用的是Linux操作系统,很有可能已经自带GNU Make和Python了,使用下面的命令查看一下版本号,确保版本别太旧:

\begin{dkcode}*{bash}
# 这里推荐使用Python3
pytho3 --version

# 查看GNU Make版本,越高越好
make --version
\end{dkcode}

之所以需要Python支持,是因为{\dk}部分组件使用了$minted$引擎调用$Pygments$库进行代码高亮渲染,如果缺少该库,这部分组件将不可用:

\begin{dkcode}*{bash}
# 查看Pygments版本
pygmentize -V

# 安装或升级Pygments库
pip install --upgrade Pygments
\end{dkcode}

\cvsubsection{Makefile及编译}
{\dk}使用Makefile\footnote{前半部分的指令都有注释进行说明,GNU Make的用法也不在本文的讨论范围之内,感兴趣的朋友可以自行学习。}将几个最常用的{\LaTeX}编译相关操作定义成了对应的控制台命令,用以对编译、输出、清理等过程进行简化处理,以及联动resource文件夹下的Makefile文件。

\dkcodefile*{../Makefile}{makefile}[Makefile]

\begin{dkcomment}
  以下所有make命令,都必须在项目根目录(就是你能看到\dkbutton{README.md}的地方)下执行。
\end{dkcomment}

\begin{cvhonors}*
  \cvhonor
  {make}
  {自动编译输出main.tex,等同于make main}
  {编译输出}
  \cvhonor
  {make main}
  {同上}
  {编译输出}
  \cvhonor
  {make doc}
  {自动编译输出\dk~文档,也就是本文档的输出命令}
  {编译输出}
  \cvhonor
  {make resource}
  {使用resource文件夹下的Makefile编译输出所有支持的资源文件,等同于进入resource文件夹下进行make或make all}
  {关联编译}
  \cvhonor
  {make all}
  {一次性编译main.tex和{\dk}文档,等同于make main \&\& make doc}
  {编译输出}
  \cvhonor
  {make clean}
  {自动清理所有临时文件和文件夹,包括主目录和所有子目录}
  {自动清理}
  \cvhonor
  {make cleanall}
  {在自动清理所有临时文件和文件夹的基础上,还会删除掉所有主目录下的PDF文件,并联动resource文件夹下所有生成的资源PDF}
  {自动清理}
\end{cvhonors}

\begin{dkcomment}
  make cleanall只会删除符合资源文件命名规则的PDF文件,其他文件不受影响。
\end{dkcomment}

\cvsubsection{文档选项及设定}
{\dk}提供了用于立刻开始创作的初始文件,位置是src/main.tex,该文件导言区\footnote{相信你知道什么是导言区,如果不是太清楚,还是抓紧自学一下吧,或者简单的认为,在大多数情况下,\\\textbackslash begin\{document\}前面的部分,就是导言区。}部分的设定与本文档源码src/dukang-doc.tex完全相同,只需要根据需要修改一些文档名称、作者姓名、首页需要哪些字段、是否需要首行缩进等信息和选项,就可以正式开始为你的大作添加正文了。

打开main.tex之后,首先会看到导言区一堆设定和备注信息,大多数情况下,阅读这些备注信息就足够掌握如何修改了,这里对所有选项进行深入说明。

\begin{cventries}
\cventry
[\Verb{\documentclass[12pt, a4paper, final]{awesome-cv-dukang}}]
[基本配置]
{
  \item 正式引入Awesome-CV文档类。
  \item 这里的字号设定(12pt)基本没什么卵用,因为几乎每个文档组件都定义了自己的字体风格。
}
\cventry
[\Verb{\geometry{left=1.4cm, top=.8cm, right=1.4cm, bottom=1.8cm, footskip=.5cm}}]
[基本配置]
{
  \item 使用geometry宏包定义纸张的页边距以及页脚距离
}
\cventry
[\Verb{\colorlet{awesome}{awesome-red}}]
[Awesome-CV]
[Awesome-CV的颜色设定]
[必选]
{
  \item 可以指定Awesome-CV预制好的几个配色集,包括awesome-emerald, awesome-skyblue, awesome-red, awesome-pink, awesome-orange, awesome-nephritis, awesome-concrete, awesome-darknight
  \item 也可以使用\Verb{\definecolor}指定自己喜欢的颜色,总共有awesome, darktext, text, graytext, lighttext, sectiondivider这几个颜色名称可供定义。
  \item {\color{awesome}{\dk}提供的所有增强组件都可以根据颜色设定进行风格自适应哦\faKissWinkHeart}
}
\cventry
[\Verb{\setbool{acvSectionColorHighlight}{true}}]
[Awesome-CV]
[指定是否使用配色凸显章节标题后紧跟的分割线]
[必选]
{
  \item 如果设定为true,责章节名称后面的长分割线会有颜色,否则为黑色。
}
\cventry
[\Verb{\renewcommand{\acvHeaderSocialSep}{\quad\textbar\quad}}]
[Awesome-CV]
[封面头部Logo右侧社交媒体帐号之间的分隔符定义]
[可选]
{
  \item 默认为管道符:\dkbutton{<空格>\textbar <空格>}
}
\cventry[个人信息部分][Awesome-CV][该部分用来定义一些个人信息或文档信息][部分可选]
{
  \item \Verb{\photo[rectangle,noedge,left]{./src/resource/dukang-logo}}用于定义首页的Logo图片,文件扩展名默认为.png,可用的裁剪选项为circle(圆形)和rectangle(正方形),可用的边框选项为edge(有边框)和noedge(无边框),可用的位置选项为left(靠左)和right(靠右)。
  \item 该部分除了\Verb{\name}和\Verb{\dukangPDFTitle}是必选的以外,其他设定不需要的均可以注释掉,首页中相应的部分会自适应。
  \item {\color{awesome}由于\Verb{\name}在Awesome-CV文档类中有多处引用,所以必须指定,不能删掉。}
  \item {\color{awesome}\Verb{\dukangPDFTitle}用来生成PDF文件书签中的主标题,所以必须指定,不能删掉。}
}
\cventry[社交媒体信息部分][Awesome-CV][该部分用来定义社交媒体帐号或联系方式][部分可选]
{
  \item 该部分有若干社交媒体选项,可以根据需要进行定义,不需要的可以注释掉。
  \item 上面定义的分隔符\Verb{\acvHeaderSocialSep}就是用来分割这些帐号的。
  \item {\color{awesome}至少要保留一条,否则编译出错!}
}
\cventry[cvletter环境基本信息][Awesome-CV][cvletter环境一般用于定义首页的内容][必选]
{
  \item 由于使用了ctex宏包,\Verb{\today}默认为大写中文日期格式。
  \item {\color{awesome}该部分的定义一个都不能少!}
}
\cventry[dukang导言区设定部分][\dk][此部分包含{\dk}及相关宏包提供的若干增强设定][必选]
{
  \item \Verb{\setbool{dukangParIndent}{true}},由于Awesome-CV大多数风格都使用了组件化(自定义命令或环境)来实现,没有使用chapter/section等标准结构,这直接导致了在引入ctex宏包进行中文化的时候,需要对每个组件进行单独的设定,比如首行缩进两字符对于有些组件要么不起作用,要么显示错乱,这里提供一个全局开关,会自动根据组件的具体情况有选择的开启首行缩进,以达到风格统一、显示美观的效果。
  \item \Verb{\setbool{dukangBookmarkLeadingNumber}{true}},Awesome-CV当前版本并不支持给输出的PDF文件按照文档结构自动添加书签(导航栏),{\dk}提供了这方面的支持,这个全局开关用来指定所添加的书签标题前,是否包含阿拉伯数字的章节编号。
  \item \Verb{\hypersetup},该部分用来为生成的PDF文件提供若干属性字段。
}
\cventry
[Awesome-CV文档区设定部分]
[Awesome-CV]
[该部分设定出现在文档区,也就是\Verb{\begin{document}和\end{document}之间。}]
[可选]
{
  \item \Verb{\makecvheader[R]},{\color{awesome}这不是页眉!Awesome-CV没有页眉。}这是首页包括Logo在内的抬头(Header)部分,可以注释掉,首页布局会自动从cvletter环境开始。可用选项用来控制对齐方向,L标识左对齐,C标识居中对齐,R标识右对齐。{\color{awesome}都要大写!}
  \item \Verb{\makecvfooter},这个是每页的页脚,分为左中右三个部分,每个部分都可以留空,但{\color{awesome}必须保留大括号\Verb{{}}}
}
\end{cventries}

以上是{\dk}当前版本设定部分的详细说明,需要格外注意的地方都有颜色高亮,右边颜色高亮的标签说明该选项来自哪个部分。并且,假如在修改的过程中不小心把main.tex搞乱了也没关系,可以随时打开dukang-doc.tex查看正确的配置,或者干脆把除了正文以外的所有内容复制回来,重新设定一下,就又可以开始创作了。这也是为什么我推荐用\Verb{\input{...}}把文档正文章节和main.tex主文件分开的原因。

\begin{dkcomment}*[温馨提示]
  无论何时,dukang-doc.tex都是你值得参考的示例文档,文档本身和其内容章节文件(特别是源代码)尽量涵盖到了{\dk}的全部功能,包括设定和功能模块等,随时可以回来查看。\faKissWinkHeart
\end{dkcomment}

\cvsubsection{使用流程}
\dk~的编译控制文件(Makefile)提供了适合下面几种场景的编译流程,相信总有一个适合你。首先,对于一般使用来说,需要做的步骤很简单:

\begin{center}
  \dkcodebox{修改main.tex}~\faArrowCircleRight~\dkcodebox{添加内容}~\faArrowCircleRight~\dkcodebox*{make}~\faCheckCircle
\end{center}

这样在src文件夹下就得到了main.pdf,同时在resource文件夹下,如果有符合命名规则的资源文件,也会被联动编译,并生成对应的PDF文件。{\dk}当前版本资源文件的命名规则是\dkbutton{r-*.tex},符合这个规则的.tex文件都会被自动编译和控制。

如果想要保留编译之后的PDF,同时把项目目录清理干净的话:

\begin{center}
  \dkcodebox{修改main.tex}~\faArrowCircleRight~\dkcodebox{添加内容}~\faArrowCircleRight~\dkcodebox*{make}~\faArrowCircleRight~\dkcodebox*{make clean}~\faCheckCircle
\end{center}

最后,如果希望只保留源代码\footnote{比如用于提交源代码,归档,或者你就是个纯粹的代码强迫症患者{\color{awesome}\faHeart}},把其他临时文件连同编译出来的东西一同干掉的话:

\begin{center}
  \dkcodebox{修改main.tex}~\faArrowCircleRight~\dkcodebox{添加内容}~\faArrowCircleRight~\dkcodebox*{make}~\faArrowCircleRight~\dkcodebox*{make cleanall}~\faCheckCircle
\end{center}

\begin{dkcomment}
  如果在编译过程中出现问题,强制退出编译过程之后想要再次编译,最好先执行\dkcodebox*{make cleanall}一遍,清理完所有临时文件之后再开始,否则编译很可能会出错无法继续下去。
\end{dkcomment}


\clearpage

%-------------------------------------------------------------------------------


\end{document}
