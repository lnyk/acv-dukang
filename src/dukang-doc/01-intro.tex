\cvsection{总体介绍}
\cvsubsection{初衷及由来}
Awesome-CV本身设计非常优秀,但由于语言习惯等差别,其对中文环境的支持和适配需要用户自行调整很多东西,同时其原作者的初衷主要是聚焦在构建“简历”模板,如此漂亮的设计,理应延伸至文章甚至书籍创作领域,在这个层面上,需要更加宏观的调整项目中每个部件的细节,并引入适合文章或书籍创作的功能模块,而且相对于漂亮的设计和丰富的功能,{\LaTeX}过高的使用门槛却一直是追求严谨排版和高质量输出的朋友最不愿意接受的。

尝试解决以上种种问题,是我的初衷,也是{\dk}项目的由来。

\cvsubsection{主体结构}
{\dk}的结构大致上分为三个逻辑部分,第一部分是来自Awesome-CV项目的文档类,定义了所生成文件的最重要的内容,第二部分是本项目新增的宏包、扩展内容、资源文件夹以及编译控制文件,第三部分是提供给使用者作为快速开始模板的文件,当然还有一些项目维护所涉及的例如README、LICENSE、Git相关文件等。

下图列举了项目主体结构和主要的文件所在位置,其中来自本项目的所有文本类文件,都有详细的备注和说明信息,如果想要{\color{awesome}快速开始},不妨按照下图找到这些文件,阅读里面的源代码,寻找自己感兴趣的部分。

\dkresource[项目结构图]{resource/r-arch}[0.9]

\cvsubsection{快速开始}

\begin{dkcode}*{bash}[在{\dk}项目根目录执行]
# 进入项目根目录
cd latex-dukang
# 编译生成你的大作
make
# 编译生成本文档
make doc
\end{dkcode}


\clearpage
