%!TEX encoding = UTF-8 Unicode

%-------------------------------------------------------------------------------
% 为Awesome-CV引入部分额外功能
% Author: William Yao
% Create: 2022-04-29
%-------------------------------------------------------------------------------

%-------------------------------------------------------------------------------
% 定义开始
%-------------------------------------------------------------------------------


%-------------------------------------------------------------------------------
% 基础依赖部分
%
% 所有功能包的引入、依赖和设定都放在这个部分
%-------------------------------------------------------------------------------
% ctex
% 加载中文环境支持
%
% 选项:
%   fontset
%     定义中文环境使用的字体,可选项为:
%       adobe, fandol, founder, mac, macnew,
%       macold, ubuntu, windows, none, <自定义>
%       其中如果指定自定义名称的配置,需配合ctex-fontset-<自定义>.def使用
% 说明:
%   本项目默认使用自定义字体文件进行编译,更多细节可查看ctex-fontset-custom.def
%   所用字体为:
%     Adobe 黑体 Std, Adobe 楷体 Std, Adobe 仿宋 Std, Adobe 宋体 Std,
%     幼圆, 隶书, 方正小标宋, YaHei Consolas Hybrid, Inconsolata
%   上述字体如果系统未安装,请自行下载安装,或根据需要自行变更
\usepackage[fontset=custom]{ctex}
%-------------------------------------------------------------------------------
% minted
% 使用minted实现代码高亮
%
% 选项:
%   cache=true
%     缓存目录
%     如果开启缓存,有可能会出现undefined control sequence错误
%     解决办法,要么在重新编译前,删除_minted-*缓存目录,要么关闭缓存
%     自动删除已添加到Makefile的make clean以及make cleanall命令中,无需手动删除
\usepackage[cache=false]{minted}
%-------------------------------------------------------------------------------
% tcolorbox
% 引入tcolorbox的库
% 由于Awesome-CV宏包已经引入了tcolorbox,因此这里如果想要启用minted库的话,
% 不能使用\usepackage命令,可以使用tcolorbox内置的\tcbuselibrary命令
% 调用对应的库
\tcbuselibrary{skins, listings, xparse, breakable, minted}
%-------------------------------------------------------------------------------
% minted的全局样式设定
% 以下设定即可以放在导言区,也可以在文档区随用随定义

% 指定所有语言的全局高亮样式
%   \usemintedstyle{minted_style}
%
% 单独指定语言的全局高亮样式
%   \usemintedstyle[lang]{minted_style}
%
% 单独指定语言的行内代码样式
%   \newmintinline{lang}{option}
%   其中option选项可为:
%     showspaces 将空格显式的打印出来
%
% 说明:
%   minted_style
%     代码高亮样式,可选项为:
%       algol, friendly, lovelace, pastie, tango, vs, autumn, fruity,
%       monokai, perldoc, trac, borland, igor, murphy, rrt, vim
%-------------------------------------------------------------------------------
% tabular
% 用于支撑tcblisting环境
\usepackage{tabularx}
\newcolumntype{\CeX}{>{\centering\let\newline\\\arraybackslash}X}%
\newcommand{\TwoSymbolsAndText}[3]{%
  \begin{tabularx}{\textwidth}{c\CeX c}%
    #1 & #2 & #3
  \end{tabularx}%
}
%-------------------------------------------------------------------------------
% dirtree
% 用于在文档中以漂亮的方式显示文件目录树
\usepackage{dirtree}
%-------------------------------------------------------------------------------


%-------------------------------------------------------------------------------
% 自定义环境部分
%
% 所有自定义环境的具体实现都放在该部分
% 自定义环境统一以dk开头,用于区分其他宏包的自定义环境
%-------------------------------------------------------------------------------
% dkcodeblue
% 以蓝色调为框架主色的代码高亮块,样式不可变
%
% 用法:
%   \begin{dkcodeblue}{lang}{minted_style}{title}
%     ...
%   \end{dkcodeblue}
%
% 说明:
%   lang
%     语言名称,例如bash, python3, java, c, go等
%   minted_style
%     minted预定义的高亮样式,可选项为:
%       algol, friendly, lovelace, pastie, tango, vs, autumn, fruity,
%       monokai, perldoc, trac, borland, igor, murphy, rrt, vim
%   title
%     代码块标题栏文字
%
% 例如:
%   \begin{dkcodeblue}{python3}{tango}{代码示例}
%     import os
%     pass
%   \end{dkcodeblue}
%-------------------------------------------------------------------------------
\newtcblisting{dkcodeblue}[3]{
  breakable,
  drop shadow,
  listing engine=minted,
  minted style=#2,
  minted language=#1,
  minted options={
    fontsize=\xiaowu,
    linenos,
    numbersep=3mm
  },
  listing only,
  left=6mm,
  enhanced,
  title={#3},
  colframe=blue!50!black,
  colback=blue!10!white,
  colbacktitle=blue!5!yellow!10!white,
  fonttitle=\bfseries,
  coltitle=black,
  attach boxed title to top center={
    yshift=-0.25mm-\tcboxedtitleheight/2,
    yshifttext=2mm-\tcboxedtitleheight/2
  },
  boxed title style={
    enhanced,boxrule=0.5mm,
    frame code={
      \path[tcb fill frame]
      ([xshift=-4mm]frame.west)
      -- (frame.north west)
      -- (frame.north east)
      -- ([xshift=4mm]frame.east)
      -- (frame.south east)
      -- (frame.south west)
      -- cycle;
    },
    interior code={
      \path[tcb fill interior]
      ([xshift=-2mm]interior.west)
      -- (interior.north west)
      -- (interior.north east)
      -- ([xshift=2mm]interior.east)
      -- (interior.south east)
      -- (interior.south west)
      -- cycle;
    }
  },
  overlay={
    \begin{tcbclipinterior}
      \fill[red!20!blue!20!white]
      (frame.south west)
      rectangle
      ([xshift=5mm]frame.north west);
    \end{tcbclipinterior}
  }
}
%-------------------------------------------------------------------------------
% dkcodefileblue
% 以蓝色调为框架主色的代码高亮块,样式不可变
% 该命令从指定文件中读取内容进行高亮显示
%
% 用法:
%   \dkcodefileblue{lang}{minted_style}{title}{file}
%
% 说明:
%   lang
%     语言名称,例如bash, python3, java, c, go等
%   minted_style
%     minted预定义的高亮样式,可选项为:
%       algol, friendly, lovelace, pastie, tango, vs, autumn, fruity,
%       monokai, perldoc, trac, borland, igor, murphy, rrt, vim
%   title
%     代码块标题栏文字
%   file
%     代码块内容的来源文件,必须为内容可读的文本文件
%
% 例如:
%   \dkcodefileblue{python3}{tango}{代码示例}{resource/code.py}
%-------------------------------------------------------------------------------
\newtcbinputlisting{dkcodefileblue}[4]{
  breakable,
  drop shadow,
  listing engine=minted,
  minted style=#2,
  minted language=#1,
  minted options={
    breaklines,
    fontsize=\xiaowu,
    linenos,
    numbersep=3mm
  },
  listing only,
  left=6mm,
  enhanced,
  title={#3},
  listing file={#4},
  colframe=blue!50!black,
  colback=blue!10!white,
  colbacktitle=blue!5!yellow!10!white,
  fonttitle=\bfseries,
  coltitle=black,
  attach boxed title to top center={
    yshift=-0.25mm-\tcboxedtitleheight/2,
    yshifttext=2mm-\tcboxedtitleheight/2
  },
  boxed title style={
    enhanced,boxrule=0.5mm,
    frame code={
      \path[tcb fill frame]
      ([xshift=-4mm]frame.west)
      -- (frame.north west)
      -- (frame.north east)
      -- ([xshift=4mm]frame.east)
      -- (frame.south east)
      -- (frame.south west)
      -- cycle;
    },
    interior code={
      \path[tcb fill interior]
      ([xshift=-2mm]interior.west)
      -- (interior.north west)
      -- (interior.north east)
      -- ([xshift=2mm]interior.east)
      -- (interior.south east)
      -- (interior.south west)
      -- cycle;
    }
  },
  overlay={
    \begin{tcbclipinterior}
      \fill[red!20!blue!20!white]
      (frame.south west)
      rectangle
      ([xshift=5mm]frame.north west);
    \end{tcbclipinterior}
  }
}
%-------------------------------------------------------------------------------
% dkcode
% 可指定框架主色的代码高亮块
%
% 用法:
%   \begin{dkcode}{lang}{minted_style}{title}{frame_color}{bg_color}
%     ...
%   \end{dkcode}
%
% 说明:
%   lang
%     语言名称,例如bash, python3, java, c, go等
%   minted_style
%     minted预定义的高亮样式,可选项为:
%       algol, friendly, lovelace, pastie, tango, vs, autumn, fruity,
%       monokai, perldoc, trac, borland, igor, murphy, rrt, vim
%   title
%     代码块标题栏文字
%   frame_color
%     标题及边框的颜色
%   bg_color
%     代码部分的底色
%
% 例如:
%   \begin{dkcode}{python3}{tango}{代码示例}{green!35!black}{green!5}
%     ...
%   \end{dkcode}
%
% 推荐样式:
%   浅绿样式 {green!35!black}{green!5}
%   黑白样式 {black}{white}
%   黑灰样式 {black!75}{black!5}
%-------------------------------------------------------------------------------
\newtcblisting{dkcode}[5]{
  breakable,
  drop shadow,
  listing engine=minted,
  minted style=#2,
  minted language=#1,
  minted options={
    breaklines,
    fontsize=\xiaowu,
    linenos,
    numbersep=3mm
  },
  listing only,
  left=6mm,
  right=6mm,
  enhanced,
  title=\TwoSymbolsAndText{\faCode}{#3}{\faCode},
  colframe=#4,
  colback=#5,
  coltitle=white,
  fonttitle=\bfseries,
}
%-------------------------------------------------------------------------------
% dkcodefile
% 可指定框架主色的代码高亮块
% 该命令从指定文件中读取内容进行高亮显示
%
% 用法:
%   \dkcodefile{lang}{minted_style}{title}{frame_color}{bg_color}{file}
%
% 说明:
%   lang
%     语言名称,例如bash, python3, java, c, go等
%   minted_style
%     minted预定义的高亮样式,可选项为:
%       algol, friendly, lovelace, pastie, tango, vs, autumn, fruity,
%       monokai, perldoc, trac, borland, igor, murphy, rrt, vim
%   title
%     代码块标题栏文字
%   frame_color
%     标题及边框的颜色
%   bg_color
%     代码部分的底色
%   file
%     代码块内容的来源文件,必须为内容可读的文本文件
%
% 例如:
%   \dkcodefile{python3}{tango}{代码示例}{green!35!black}{green!5}{resource/code.py}
%
% 推荐样式:
%   浅绿样式 {green!35!black}{green!5}
%   黑白样式 {black}{white}
%   黑灰样式 {black!75}{black!5}
%-------------------------------------------------------------------------------
\newtcbinputlisting{dkcodefile}[6]{
  before skip=18bp,
  breakable,
  drop shadow,
  listing engine=minted,
  minted style=#2,
  minted language=#1,
  minted options={
    breaklines,
    fontsize=\xiaowu,
    linenos,
    numbersep=3mm
  },
  listing only,
  left=6mm,
  right=6mm,
  enhanced,
  title=\TwoSymbolsAndText{\faCode}{#3}{\faCode},
  listing file={#6},
  colframe=#4,
  colback=#5,
  coltitle=white,
  fonttitle=\bfseries,
}
%-------------------------------------------------------------------------------
% dkcomment
% 可指定框架主色的备注框
%
% 用法:
%   \begin{dkcomment}{title}{fontawesome_name}{frame_color}{bg_color}
%     ...
%   \end{dkcomment}
%
% 说明:
%   title
%     代码块标题栏文字
%   fontawesome_name
%     FontAwesome字体的代码或名称
%   frame_color
%     标题及边框的颜色
%   bg_color
%     代码部分的底色
%
% 例如:
%   \begin{dkcomment}{特别说明}{\faCheck}{green!35!black}{green!5}
%     ...
%   \end{dkcomment}
%
% 推荐样式:
%   浅绿样式 {green!35!black}{green!5}
%   黑白样式 {black}{white}
%   黑灰样式 {black!75}{black!5}
%-------------------------------------------------------------------------------
\newtcolorbox{dkcomment}[4]{
  before skip=18bp,
  drop shadow,
  size=title,
  top=3mm,
  bottom=3mm,
  left=6mm,
  right=6mm,
  arc=1.5mm,
  breakable,
  enhanced jigsaw,
  colframe=#3,
  coltitle=white,
  boxrule=0.5mm,
  colback=#4,
  coltext=black,
  title=\TwoSymbolsAndText{#2}{#1}{#2}
}
%-------------------------------------------------------------------------------


%-------------------------------------------------------------------------------
% 定义结束
%-------------------------------------------------------------------------------
